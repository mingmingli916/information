\documentclass{article}
\input{~/latex/article-preamble.tex}
\begin{document}

\title{Conjugaison}
\author{Mingming Li}
\date{2022/02/07}
\begin{titlepage}
\maketitle{}  
\end{titlepage}


\tableofcontents{}
\newpage{}

\section{Indicatif Présent}
\subsection{-er}

Il y a 9630 verbes se terminant par -er.

\begin{table}[H]
  \centering
  \begin{tabular}{p{0.2\columnwidth}p{0.8\columnwidth}}
    \toprule[1.5pt]
    \head{sujet} & \head{terminaison} \\
    \midrule[1.5pt]
    je & -e \\
    tu & -es \\
    il/elle/on & -e \\
    nous & -ons \\
    vous & -ez \\
    ils/elles & -ent \\
    \bottomrule[1.5pt]
  \end{tabular}
  \caption{-er}
\end{table}



\begin{table}[H]
  \centering
  \begin{tabular}{p{0.2\columnwidth}p{0.8\columnwidth}}
    \toprule[1.5pt]
    \head{sujet} & \head{conjugaison} \\
    \midrule[1.5pt]
    je & parl\textbf{\underline{e}} \\
    tu & parl\textbf{\underline{es}} \\
    il/elle/on & parl\textbf{\underline{e}} \\
    nous & parl\textbf{\underline{ons}} \\
    vous & parl\textbf{\underline{ez}} \\
    ils/elles & parl\textbf{\underline{ent}} \\
    \bottomrule[1.5pt]
  \end{tabular}
  \caption{parler}
\end{table}

\subsubsection{-cer}

Pour le prononciation de [s]:
\begin{table}[H]
  \centering
  \begin{tabular}{p{0.2\columnwidth}p{0.8\columnwidth}} 
    \toprule[1.5pt]
    \head{sujet} & \head{terminaison} \\
    \midrule[1.5pt]    
    nous & -çons \\
    \bottomrule[1.5pt]
  \end{tabular}
  \caption{-cer}
\end{table}

\subsubsection{-ger}

Pour le prononciation de \textipa{[Z]}:
\begin{table}[H]
  \centering
  \begin{tabular}{p{0.2\columnwidth}p{0.8\columnwidth}}
    \toprule[1.5pt]
    \head{sujet} & \head{terminaison} \\
    \midrule[1.5pt]    
    nous & -geons \\
    \bottomrule[1.5pt]    
  \end{tabular}
  \caption{-ger}
\end{table}

\subsubsection{-é\_{}er}

Tous les verbes qui se terminent par –é\_{}er changent le é en è dans les conjugaisons à radical modifié.

\subsubsection{-e\_{}er}

La plupart des verbes qui se terminent par –e\_{}er changent le e muet (l'avant-dernier e) en è.

\subsubsection{-eler, -eter}

les verbes suivants qui se terminent par –eler et –eter doublent le l ou le t dans les conjugaisons à radical modifié.

\begin{itemize}
\item appeler
\item chanceler
\item épeler
\item rappeler
\item renouveler
\item ruisseler
\item feuilleter
\item hoqueter
\item jeter
\item projeter
\item rejeter
\end{itemize}

\subsubsection{-ayer, -oyer, -uyer}

Les verbes qui se terminent par –ayer, –oyer, ou –uyer stem-changent le Y en I.

\subsubsection{les Irréguliers}

\begin{table}[H]
  \centering
  \begin{tabular}{p{0.2\columnwidth}p{0.8\columnwidth}}
    \toprule[1.5pt]
    \head{sujet} & \head{conjugaison} \\
    \midrule[1.5pt]
    j' & vais \\
    tu & vas \\
    il/elle/on & va \\
    nous & allons \\
    vous & allez \\
    ils/elles & vont \\
    \bottomrule[1.5pt]
  \end{tabular}
  \caption{aller}
\end{table}

\subsection{-ir}

Il y a 597 verbes se terminant par -ir.

\begin{table}[H]
  \centering
  \begin{tabular}{p{0.2\columnwidth}p{0.8\columnwidth}}
    \toprule[1.5pt]
    \head{sujet} & \head{terminaison} \\
    \midrule[1.5pt]
    je & -is \\
    tu & -is \\
    il/elle/on & -it \\
    nous & -issons \\
    vous & -issez \\
    ils/elles & -issent \\
    \bottomrule[1.5pt]
  \end{tabular}
  \caption{-ir}
\end{table}



\begin{table}[H]
  \centering
  \begin{tabular}{p{0.2\columnwidth}p{0.8\columnwidth}}
    \toprule[1.5pt]
    \head{sujet} & \head{conjugaison} \\
    \midrule[1.5pt]
    je & finis \\
    tu & finis \\
    il/elle/on & finit \\
    nous & finissons \\
    vous & finissez \\
    ils/elles & finissent \\
    \bottomrule[1.5pt]
  \end{tabular}
  \caption{finir}
\end{table}

\subsubsection{partir}


\begin{table}[H]
  \centering
  \begin{tabular}{p{0.2\columnwidth}p{0.8\columnwidth}}
    \toprule[1.5pt]
    \head{sujet} & \head{conjugaisons} \\
    \midrule[1.5pt]
    je & par\underline{\textbf{s}} \\
    tu & par\textbf{\underline{s}} \\
    il/elle/on & par\underline{\textbf{t}}\\
    nous & part\underline{\textbf{ons}} \\
    vous & part\underline{\textbf{ez}} \\
    ils/elles & part\underline{\textbf{ent}} \\
    \bottomrule[1.5pt]
  \end{tabular}
  \caption{partir}
\end{table}


Verbes communs:
\begin{itemize}
\item dormir
\item mentir
\item sentir
\item servir
\item sortir
\item ressentir
\item endormir
\end{itemize}


\subsubsection{-llir, -frir, -vrir}

La plupart des verbes qui se terminent par -llir et tous ceux qui se terminent par -frir ou -vrir sont conjugués comme les verbes réguliers -er


\begin{table}[H]
  \centering
  \begin{tabular}{p{0.2\columnwidth}p{0.8\columnwidth}}
    \toprule[1.5pt]
    \head{sujet} & \head{conjugaison} \\
    \midrule[1.5pt]
    je & couv\underline{\textbf{e}} \\
    tu & couv\textbf{\underline{es}} \\
    il/elle/on & couv\underline{\textbf{e}}\\
    nous & couv\underline{\textbf{ons}} \\
    vous & couv\underline{\textbf{ez}} \\
    ils/elles & couv\underline{\textbf{ent}} \\
    \bottomrule[1.5pt]
  \end{tabular}
  \caption{couvrir}
\end{table}

Verbes communs:
\begin{itemize}
\item assillir
\item couvrir
\item cueillir
\item defaillir
\item offrir	
\item ouvrir
\item saillir	
\item souffrir
\item tressaillir
\end{itemize}


\subsubsection{-enir}

Tous les verbes avec cette terminaison (tenir, venir et tous leurs dérivés) suivent ce modèle.

\begin{table}[H]
  \centering
  \begin{tabular}{p{0.2\columnwidth}p{0.8\columnwidth}}
    \toprule[1.5pt]
    \head{sujet} & \head{conjugaison} \\
    \midrule[1.5pt]
    je & viens \\
    tu & viens \\
    il/elle/on & vient \\
    nous & venons \\
    vous & venez \\
    ils/elles & viennent \\
    \bottomrule[1.5pt]
  \end{tabular}
  \caption{venir}
\end{table}

\subsubsection{les Irréguliers}


\begin{table}[H]
  \centering
  \begin{tabular}{p{0.2\columnwidth}p{0.8\columnwidth}}
    \toprule[1.5pt]
    \head{sujet} & \head{conjugaison} \\
    \midrule[1.5pt]
    j' & ai \\
    tu & as \\
    il/elle/on & a \\
    nous & avons \\
    vous & avez \\
    ils/elles & ont \\
    \bottomrule[1.5pt]
  \end{tabular}
  \caption{avoir}
\end{table}



\begin{table}[H]
  \centering
  \begin{tabular}{p{0.2\columnwidth}p{0.8\columnwidth}}
    \toprule[1.5pt]
    \head{sujet} & \head{conjugaison} \\
    \midrule[1.5pt]
    j' & acquiers \\
    tu & acquiers \\
    il/elle/on & acquiert \\
    nous & acquérons \\
    vous & acquérez \\
    ils/elles & acquièrent \\
    \bottomrule[1.5pt]
  \end{tabular}
  \caption{acquérir}
\end{table}


\begin{table}[H]
  \centering
  \begin{tabular}{p{0.2\columnwidth}p{0.8\columnwidth}}
    \toprule[1.5pt]
    \head{sujet} & \head{conjugaison} \\
    \midrule[1.5pt]
    je & conquiers \\
    tu & conquiers \\
    il/elle/on & conquiert \\
    nous & conquérons \\
    vous & conquérez \\
    ils/elles & conquièrent \\
    \bottomrule[1.5pt]
  \end{tabular}
  \caption{conquérir}
\end{table}


\begin{table}[H]
  \centering
  \begin{tabular}{p{0.2\columnwidth}p{0.8\columnwidth}}
    \toprule[1.5pt]
    \head{sujet} & \head{conjugaison} \\
    \midrule[1.5pt]
    je & bous \\
    tu & bous\\
    il/elle/on & bout \\
    nous & bouillons \\
    vous & bouillez \\
    ils/elles & bouillent \\
    \bottomrule[1.5pt]
  \end{tabular}
  \caption{bouillir}
\end{table}


\begin{table}[H]
  \centering
  \begin{tabular}{p{0.2\columnwidth}p{0.8\columnwidth}}
    \toprule[1.5pt]
    \head{sujet} & \head{conjugaison} \\
    \midrule[1.5pt]
    je & cours \\
    tu & cours\\
    il/elle/on & court \\
    nous & courons \\
    vous & courez \\
    ils/elles & courent \\
    \bottomrule[1.5pt]
  \end{tabular}
  \caption{courir}
\end{table}

\begin{itemize}
\item parcourir
\item secourir
\end{itemize}

\begin{table}[H]
  \centering
  \begin{tabular}{p{0.2\columnwidth}p{0.8\columnwidth}}
    \toprule[1.5pt]
    \head{sujet} & \head{conjugaison} \\
    \midrule[1.5pt]
    je & déçois \\
    tu & déçois\\
    il/elle/on & déçoit \\
    nous & décevons \\
    vous & décevez \\
    ils/elles & déçoivent \\
    \bottomrule[1.5pt]
  \end{tabular}
  \caption{décevoir}
\end{table}

\begin{itemize}
\item recevoir
\end{itemize}

\begin{table}[H]
  \centering
  \begin{tabular}{p{0.2\columnwidth}p{0.8\columnwidth}}
    \toprule[1.5pt]
    \head{sujet} & \head{conjugaison} \\
    \midrule[1.5pt]
    je & dois \\
    tu & dois\\
    il/elle/on & doit \\
    nous & devons \\
    vous & devez \\
    ils/elles & doivent \\
    \bottomrule[1.5pt]
  \end{tabular}
  \caption{devoir}
\end{table}


\begin{table}[H]
  \centering
  \begin{tabular}{p{0.2\columnwidth}p{0.8\columnwidth}}
    \toprule[1.5pt]
    \head{sujet} & \head{conjugaison} \\
    \midrule[1.5pt]
    il & faut \\
    \bottomrule[1.5pt]
  \end{tabular}
  \caption{falloir}
\end{table}

\begin{table}[H]
  \centering
  \begin{tabular}{p{0.2\columnwidth}p{0.8\columnwidth}}
    \toprule[1.5pt]
    \head{sujet} & \head{conjugaison} \\
    \midrule[1.5pt]
    il & pleut \\
    \bottomrule[1.5pt]
  \end{tabular}
  \caption{pleuvoir}
\end{table}


\begin{table}[H]
  \centering
  \begin{tabular}{p{0.2\columnwidth}p{0.8\columnwidth}}
    \toprule[1.5pt]
    \head{sujet} & \head{conjugaison} \\
    \midrule[1.5pt]
    je & meurs \\
    tu & meurs\\
    il/elle/on & mourt \\
    nous & mouront \\
    vous & meurez\\
    ils/elles & meurent \\
    \bottomrule[1.5pt]
  \end{tabular}
  \caption{mourir}
\end{table}

\begin{table}[H]
  \centering
  \begin{tabular}{p{0.2\columnwidth}p{0.8\columnwidth}}
    \toprule[1.5pt]
    \head{sujet} & \head{conjugaison} \\
    \midrule[1.5pt]
    je & meus \\
    tu & meus\\
    il/elle/on & meut \\
    nous & mouvons \\
    vous & mouvez\\
    ils/elles & meuvent \\
    \bottomrule[1.5pt]
  \end{tabular}
  \caption{mouvoir}
\end{table}

\begin{itemize}
\item émouvoir
\item promouvoir
\end{itemize}


\begin{table}[H]
  \centering
  \begin{tabular}{p{0.2\columnwidth}p{0.8\columnwidth}}
    \toprule[1.5pt]
    \head{sujet} & \head{conjugaison} \\
    \midrule[1.5pt]
    je & peux \\
    tu & peux\\
    il/elle/on & peut \\
    nous & pouvons \\
    vous & pouvez\\
    ils/elles & peuvent \\
    \bottomrule[1.5pt]
  \end{tabular}
  \caption{pouvoir}
\end{table}


\begin{table}[H]
  \centering
  \begin{tabular}{p{0.2\columnwidth}p{0.8\columnwidth}}
    \toprule[1.5pt]
    \head{sujet} & \head{conjugaison} \\
    \midrule[1.5pt]
    je & veux \\
    tu & veux\\
    il/elle/on & veut \\
    nous & voulons \\
    vous & voulez\\
    ils/elles & veulent \\
    \bottomrule[1.5pt]
  \end{tabular}
  \caption{vouloir}
\end{table}


\begin{table}[H]
  \centering
  \begin{tabular}{p{0.2\columnwidth}p{0.8\columnwidth}}
    \toprule[1.5pt]
    \head{sujet} & \head{conjugaison} \\
    \midrule[1.5pt]
    je & sais \\
    tu & sais\\
    il/elle/on & sait \\
    nous & savons \\
    vous & savez\\
    ils/elles & savent \\
    \bottomrule[1.5pt]
  \end{tabular}
  \caption{savoir}
\end{table}

\begin{table}[H]
  \centering
  \begin{tabular}{p{0.2\columnwidth}p{0.8\columnwidth}}
    \toprule[1.5pt]
    \head{sujet} & \head{conjugaison} \\
    \midrule[1.5pt]
    je & sers\\
    tu & sers\\
    il/elle/on & sert \\
    nous & servons \\
    vous & servez\\
    ils/elles & servent \\
    \bottomrule[1.5pt]
  \end{tabular}
  \caption{servir}
\end{table}

\begin{itemize}
\item desservir
\end{itemize}



\begin{table}[H]
  \centering
  \begin{tabular}{p{0.2\columnwidth}p{0.8\columnwidth}}
    \toprule[1.5pt]
    \head{sujet} & \head{conjugaison} \\
    \midrule[1.5pt]
    je & vaux\\
    tu & vaux\\
    il/elle/on & vaut \\
    nous & valons \\
    vous & valez\\
    ils/elles & valent \\
    \bottomrule[1.5pt]
  \end{tabular}
  \caption{valoir}
\end{table}




\begin{table}[H]
  \centering
  \begin{tabular}{p{0.2\columnwidth}p{0.8\columnwidth}}
    \toprule[1.5pt]
    \head{sujet} & \head{conjugaison} \\
    \midrule[1.5pt]
    je & vêts\\
    tu & vêts\\
    il/elle/on & vêt \\
    nous & vêtons \\
    vous & veutez\\
    ils/elles & vêtent \\
    \bottomrule[1.5pt]
  \end{tabular}
  \caption{vêtir}
\end{table}

\begin{itemize}
\item revêtir
\end{itemize}



\begin{table}[H]
  \centering
  \begin{tabular}{p{0.2\columnwidth}p{0.8\columnwidth}}
    \toprule[1.5pt]
    \head{sujet} & \head{conjugaison} \\
    \midrule[1.5pt]
    je & vois\\
    tu & vois\\
    il/elle/on & voit \\
    nous & voyons \\
    vous & voyez\\
    ils/elles & voient \\
    \bottomrule[1.5pt]
  \end{tabular}
  \caption{voir}
\end{table}


\subsection{-re}

Il y a 316 verbes se terminant par -re.

\begin{table}[H]
  \centering
  \begin{tabular}{p{0.2\columnwidth}p{0.8\columnwidth}}
    \toprule[1.5pt]
    \head{sujet} & \head{terminaison} \\
    \midrule[1.5pt]
    je & -s \\
    tu & -s \\
    il/elle/on & -t \\
    nous & -ons \\
    vous & -ez \\
    ils/elles & -ent \\
    \bottomrule[1.5pt]
  \end{tabular}
  \caption{-re}
\end{table}

\begin{table}[H]
  \centering
  \begin{tabular}{p{0.2\columnwidth}p{0.8\columnwidth}}
    \toprule[1.5pt]
    \head{sujet} & \head{conjugaison} \\
    \midrule[1.5pt]
    je & romps\\
    tu & romps\\
    il/elle/on & rompt \\
    nous & rompons \\
    vous & rompez\\
    ils/elles & rompent \\
    \bottomrule[1.5pt]
  \end{tabular}
  \caption{rompre}
\end{table}



\subsubsection{-dre}

\begin{table}[H]
  \centering
  \begin{tabular}{p{0.2\columnwidth}p{0.8\columnwidth}}
    \toprule[1.5pt]
    \head{sujet} & \head{conjugaison} \\
    \midrule[1.5pt]
    je & perds\\
    tu & perds\\
    il/elle/on & perd \\
    nous & perdons \\
    vous & perdez\\
    ils/elles & perdent \\
    \bottomrule[1.5pt]
  \end{tabular}
  \caption{perdre}
\end{table}



\subsubsection{-uire, -dire, -fire, -lire}

Tous les verbes qui se terminent par -uire, -dire, -fire et -lire sont conjugués en supprimant -re et en ajoutant s aux formes plurielles.

\begin{table}[H]
  \centering
  \begin{tabular}{p{0.2\columnwidth}p{0.8\columnwidth}}
    \toprule[1.5pt]
    \head{sujet} & \head{conjugaison} \\
    \midrule[1.5pt]
    je & cuis\\
    tu & cuis\\
    il/elle/on & cuit \\
    nous & cuisons \\
    vous & cuisez\\
    ils/elles & cuisent \\
    \bottomrule[1.5pt]
  \end{tabular}
  \caption{cuire}
\end{table}

\subsubsection{-crire}

Tous les verbes qui se terminent par -crire sont conjugués en supprimant -re et en ajoutant v aux formes plurielles.


\begin{table}[H]
  \centering
  \begin{tabular}{p{0.2\columnwidth}p{0.8\columnwidth}}
    \toprule[1.5pt]
    \head{sujet} & \head{conjugaison} \\
    \midrule[1.5pt]
    j' & écris\\
    tu & écris\\
    il/elle/on & écrit \\
    nous & écrivons \\
    vous & écrivez\\
    ils/elles & écrivent \\
    \bottomrule[1.5pt]
  \end{tabular}
  \caption{écrire}
\end{table}

\subsubsection{-aindre, -eindre, -oindre}


All verbs that end in -aindre, -eindre, and -oindre are conjugated by dropping the d in all forms and adding g in front of n in the plural forms.


\begin{table}[H]
  \centering
  \begin{tabular}{p{0.2\columnwidth}p{0.8\columnwidth}}
    \toprule[1.5pt]
    \head{sujet} & \head{conjugaison} \\
    \midrule[1.5pt]
    je & crains\\
    tu & crains\\
    il/elle/on & craint \\
    nous & craignons \\
    vous & craignez\\
    ils/elles & craignent \\
    \bottomrule[1.5pt]
  \end{tabular}
  \caption{craindre}
\end{table}

\subsubsection{-ttre}

Tous les verbes qui se terminent en –ttre se conjuguent au présent en supprimant le second t à toutes les formes singulières.

\begin{table}[H]
  \centering
  \begin{tabular}{p{0.2\columnwidth}p{0.8\columnwidth}}
    \toprule[1.5pt]
    \head{sujet} & \head{conjugaison} \\
    \midrule[1.5pt]
    je & mets\\
    tu & mets\\
    il/elle/on & met \\
    nous & mettons \\
    vous & mettez\\
    ils/elles & mettent \\
    \bottomrule[1.5pt]
  \end{tabular}
  \caption{mettre}
\end{table}

\subsubsection{prendre}

\begin{table}[H]
  \centering
  \begin{tabular}{p{0.2\columnwidth}p{0.8\columnwidth}}
    \toprule[1.5pt]
    \head{sujet} & \head{conjugaison} \\
    \midrule[1.5pt]
    je & prends\\
    tu & prends\\
    il/elle/on & prend \\
    nous & prenons \\
    vous & prenez\\
    ils/elles & prennent \\
    \bottomrule[1.5pt]
  \end{tabular}
  \caption{prendre}
\end{table}

\subsubsection{-aître}

\begin{table}[H]
  \centering
  \begin{tabular}{p{0.2\columnwidth}p{0.8\columnwidth}}
    \toprule[1.5pt]
    \head{sujet} & \head{conjugaison} \\
    \midrule[1.5pt]
    je & connaîs\\
    tu & connaîs\\
    il/elle/on & connaît \\
    nous & connaîssons \\
    vous & connaîssez\\
    ils/elles & connaîssent \\
    \bottomrule[1.5pt]
  \end{tabular}
  \caption{connaître}
\end{table}

\subsubsection{ les irréguliers}

\begin{table}[H]
  \centering
  \begin{tabular}{p{0.2\columnwidth}p{0.8\columnwidth}}
    \toprule[1.5pt]
    \head{sujet} & \head{conjugaison} \\
    \midrule[1.5pt]
    je & bois\\
    tu & bois\\
    il/elle/on & boit \\
    nous & buvons \\
    vous & buvez\\
    ils/elles & boivent \\
    \bottomrule[1.5pt]
  \end{tabular}
  \caption{boire}
\end{table}
\begin{table}[H]
  \centering
  \begin{tabular}{p{0.2\columnwidth}p{0.8\columnwidth}}
    \toprule[1.5pt]
    \head{sujet} & \head{conjugaison} \\
    \midrule[1.5pt]
    je & clos\\
    tu & clos\\
    il/elle/on & clôt \\
    nous &  \\
    vous & \\
    ils/elles & closent \\
    \bottomrule[1.5pt]
  \end{tabular}
  \caption{clore}
\end{table}
\begin{table}[H]
  \centering
  \begin{tabular}{p{0.2\columnwidth}p{0.8\columnwidth}}
    \toprule[1.5pt]
    \head{sujet} & \head{conjugaison} \\
    \midrule[1.5pt]
    je & conclus\\
    tu & conslus\\
    il/elle/on & conclut \\
    nous & concluons \\
    vous & concluez\\
    ils/elles & concluent \\
    \bottomrule[1.5pt]
  \end{tabular}
  \caption{conclure}
\end{table}

\begin{itemize}
\item exclure
\item inclure
\item occlure
\end{itemize}

\begin{table}[H]
  \centering
  \begin{tabular}{p{0.2\columnwidth}p{0.8\columnwidth}}
    \toprule[1.5pt]
    \head{sujet} & \head{conjugaison} \\
    \midrule[1.5pt]
    je & couds\\
    tu & couds\\
    il/elle/on & coud \\
    nous & cousons \\
    vous & cousez\\
    ils/elles & cousent \\
    \bottomrule[1.5pt]
  \end{tabular}
  \caption{condre}
\end{table}
\begin{table}[H]
  \centering
  \begin{tabular}{p{0.2\columnwidth}p{0.8\columnwidth}}
    \toprule[1.5pt]
    \head{sujet} & \head{conjugaison} \\
    \midrule[1.5pt]
    je & crois\\
    tu & crois\\
    il/elle/on & croit \\
    nous & croyons \\
    vous & croyez\\
    ils/elles & croient \\
    \bottomrule[1.5pt]
  \end{tabular}
  \caption{croire}
\end{table}
\begin{table}[H]
  \centering
  \begin{tabular}{p{0.2\columnwidth}p{0.8\columnwidth}}
    \toprule[1.5pt]
    \head{sujet} & \head{conjugaison} \\
    \midrule[1.5pt]
    je & absous\\
    tu & absous\\
    il/elle/on & absout \\
    nous & absolvons\\
    vous & absolvez\\
    ils/elles & absolvent \\
    \bottomrule[1.5pt]
  \end{tabular}
  \caption{absoudre}
\end{table}

\begin{itemize}
\item dissoudre
\item résoudre
\end{itemize}


\begin{table}[H]
  \centering
  \begin{tabular}{p{0.2\columnwidth}p{0.8\columnwidth}}
    \toprule[1.5pt]
    \head{sujet} & \head{conjugaison} \\
    \midrule[1.5pt]
    je & distrais\\
    tu & distrais\\
    il/elle/on & distrait \\
    nous & distrayons \\
    vous & distrayez\\
    ils/elles & distraient \\
    \bottomrule[1.5pt]
  \end{tabular}
  \caption{distraire}
\end{table}
\begin{table}[H]
  \centering
  \begin{tabular}{p{0.2\columnwidth}p{0.8\columnwidth}}
    \toprule[1.5pt]
    \head{sujet} & \head{conjugaison} \\
    \midrule[1.5pt]
    je & suis\\
    tu & es\\
    il/elle/on & est \\
    nous & sommes \\
    vous & êtes\\
    ils/elles & sont \\
    \bottomrule[1.5pt]
  \end{tabular}
  \caption{être}
\end{table}
\begin{table}[H]
  \centering
  \begin{tabular}{p{0.2\columnwidth}p{0.8\columnwidth}}
    \toprule[1.5pt]
    \head{sujet} & \head{conjugaison} \\
    \midrule[1.5pt]
    je & fais\\
    tu & fais\\
    il/elle/on & fait \\
    nous & faisons \\
    vous & faites\\
    ils/elles & font \\
    \bottomrule[1.5pt]
  \end{tabular}
  \caption{faire}
\end{table}

\begin{itemize}
\item défaire
\item parfaire
\item refaire
\end{itemize}

\begin{table}[H]
  \centering
  \begin{tabular}{p{0.2\columnwidth}p{0.8\columnwidth}}
    \toprule[1.5pt]
    \head{sujet} & \head{conjugaison} \\
    \midrule[1.5pt]
    je & mouds\\
    tu & mouds\\
    il/elle/on & moud \\
    nous & moulons \\
    vous & moulez\\
    ils/elles & moulent \\
    \bottomrule[1.5pt]
  \end{tabular}
  \caption{moudre}
\end{table}
\begin{table}[H]
  \centering
  \begin{tabular}{p{0.2\columnwidth}p{0.8\columnwidth}}
    \toprule[1.5pt]
    \head{sujet} & \head{conjugaison} \\
    \midrule[1.5pt]
    je & nais\\
    tu & nais\\
    il/elle/on & naît \\
    nous & naissons \\
    vous & naissez\\
    ils/elles & naissent \\
    \bottomrule[1.5pt]
  \end{tabular}
  \caption{naître}
\end{table}
\begin{table}[H]
  \centering
  \begin{tabular}{p{0.2\columnwidth}p{0.8\columnwidth}}
    \toprule[1.5pt]
    \head{sujet} & \head{conjugaison} \\
    \midrule[1.5pt]
    je & plais\\
    tu & plais\\
    il/elle/on & plait \\
    nous & plaisons \\
    vous & plaisez\\
    ils/elles & plaisent \\
    \bottomrule[1.5pt]
  \end{tabular}
  \caption{plaire}
\end{table}

\begin{itemize}
\item déplaire
\end{itemize}

\begin{table}[H]
  \centering
  \begin{tabular}{p{0.2\columnwidth}p{0.8\columnwidth}}
    \toprule[1.5pt]
    \head{sujet} & \head{conjugaison} \\
    \midrule[1.5pt]
    je & ris\\
    tu & ris\\
    il/elle/on & rit \\
    nous & rions \\
    vous & riez\\
    ils/elles & rient \\
    \bottomrule[1.5pt]
  \end{tabular}
  \caption{rire}
\end{table}

\begin{itemize}
\item sourire
\end{itemize}

\begin{table}[H]
  \centering
  \begin{tabular}{p{0.2\columnwidth}p{0.8\columnwidth}}
    \toprule[1.5pt]
    \head{sujet} & \head{conjugaison} \\
    \midrule[1.5pt]
    je & suis\\
    tu & suis\\
    il/elle/on & suit \\
    nous & suivons \\
    vous & suivez\\
    ils/elles & suivent \\
    \bottomrule[1.5pt]
  \end{tabular}
  \caption{suivre}
\end{table}

\begin{itemize}
\item poursuivre
\end{itemize}

\begin{table}[H]
  \centering
  \begin{tabular}{p{0.2\columnwidth}p{0.8\columnwidth}}
    \toprule[1.5pt]
    \head{sujet} & \head{conjugaison} \\
    \midrule[1.5pt]
    je & tais\\
    tu & tais\\
    il/elle/on & tait \\
    nous & taisons \\
    vous & taisez\\
    ils/elles & taisent \\
    \bottomrule[1.5pt]
  \end{tabular}
  \caption{se taire}
\end{table}
\begin{table}[H]
  \centering
  \begin{tabular}{p{0.2\columnwidth}p{0.8\columnwidth}}
    \toprule[1.5pt]
    \head{sujet} & \head{conjugaison} \\
    \midrule[1.5pt]
    je & vis\\
    tu & vis\\
    il/elle/on & vit \\
    nous & vivons \\
    vous & vivez\\
    ils/elles & vivent \\
    \bottomrule[1.5pt]
  \end{tabular}
  \caption{vivre}
\end{table}
\begin{itemize}
\item suivivre
\end{itemize}




\section{Indicatif Passé composé}
La majorité des verbes français sont réguliers et la formation de leur participe passé est facile.

Utilisez simplement la recette ci-dessous:

\begin{table}[H]
  \centering
  \begin{tabular}{p{0.3\columnwidth}p{0.3\columnwidth}}
    \toprule[0.5pt]
    -er & é \\
    -ir & i \\
    -re & u \\
    \bottomrule[0.5pt]
  \end{tabular}
  \caption{regular}
\end{table}


\begin{table}[H]
  \centering
  \begin{tabular}{p{0.3\columnwidth}p{0.3\columnwidth}}
    \toprule[0.5pt]
    -ire & it \\
    -aitre & u \\
    -enir & enu \\
    -endre & ris \\
    \bottomrule[0.5pt]
  \end{tabular}
  \caption{irregular}
\end{table}

Certains verbes irréguliers ne correspondent à aucun de ces modèles, si tel est le cas, vous devez rechercher la conjugaison du participe passé individuel.

\section{Indicatif Participe présent}

\begin{table}[H]
  \centering
  \begin{tabular}{p{0.3\columnwidth}p{0.3\columnwidth}p{0.3\columnwidth}}
    \toprule[1.5pt]
    \keyword{l'infinitif} & \keyword{présent} & \keyword{participe présent} \\
    \midrule[1.5pt]
    parler & nous parl\textbf{\underline{ons}} & parl\textbf{\underline{ant}} \\
    dormir & nous dorm\textbf{\underline{ons}} & dorm\textbf{\underline{ant}} \\
    prendre & nous pren\textbf{\underline{ons}} & pron\textbf{\underline{ant}} \\
    \bottomrule[1.5pt]
  \end{tabular}
  \caption{Participe présent}
\end{table}



\section{Indicatif Imparfait}

Pour la plupart des verbes, le redical de l'imparfait est la première personne du pluriel (nous) du présent de l'indicadif.

Les terminisons sont:
\begin{table}[H]
  \centering
  \begin{tabular}{p{0.3\columnwidth}p{0.3\columnwidth}}
    \toprule[1.5pt]
    \keyword{sujet} & \keyword{terminaison} \\
    \midrule[1.5pt]{}
    je & -ais \\
    tu & -ais \\
    il/elle/on & -ait \\
    nous & -ions \\
    vous & -iez \\
    ils/elles & -aient \\
    \bottomrule[1.5pt]{}
  \end{tabular}
  \caption{Les terminaisons de l'imparfait}
\end{table}


\begin{table}[H]
  \centering
  \begin{tabular}{p{0.3\columnwidth}p{0.3\columnwidth}p{0.3\columnwidth}}
    \toprule[1.5pt]
    \keyword{Verbe} & \keyword{Présent} & \keyword{Imparfait} \\
    \midrule[1.5pt]
    aimer & Nous \textbf{aim}ons & J'aim\textbf{ais} \\
    choisir & Nous \textbf{choissi}ons & Tu choissi\textbf{ais} \\
    partir & Nous \textbf{part}ons & Il part\textbf{ait} \\
    pouvoir & Nous \textbf{pouv}ons & Nous pouv\textbf{ions} \\
    fair & Nous \textbf{fai}sons & Vous fais\textbf{iez} \\
    venir & Nous \textbf{ven}ons & Ils ven\textbf{aient} \\
    \bottomrule[1.5pt]
  \end{tabular}
  \caption{Example}
\end{table}

Attention! Il y a une seule exception, le verbr être

\begin{table}[H]
  \centering
  \begin{tabular}{p{0.3\columnwidth}p{0.3\columnwidth}}
    \toprule[1.5pt]
    j' & étais \\
    tu & étais \\
    il/elle/on & était \\
    nous & étions \\
    vous & étiez \\
    ils/elles & étaient \\
    \bottomrule[1.5pt]
  \end{tabular}
  \caption{être}
\end{table}




\section{Subjonctif Présent}

\subsection{Les terminaisons}

\begin{table}[H]
  \centering
  \begin{tabular}{cc}
    \toprule[1.5pt]
    Je & -e \\
    Tu & -es \\
    Il & -e \\
    Nous & -ions \\
    Vous & -iez \\
    Ils & -ent \\
    \bottomrule[1.5pt]
  \end{tabular}
  \caption{Les terminaisons}
\end{table}

\subsection{Les radicaux}

a. Pour former la subjonctif, on utilise le radical de la $3^e$ personne du pluriel du présent de l'indicatif. Par example:

\begin{table}[H]
  \centering
  \begin{tabular}{cc}
    \toprule[1.5pt]
    \keyword{Présent de l'indicatif} & \keyword{Subjonctif} \\
    \midrule[1.5pt]
    je finis & que je \keyword{finiss}e \\
    tu finis & que tu \keyword{finiss}es \\
    il finit & qu'il \keyword{finiss}e \\
    nous finissons & que nous \keyword{finiss}ions \\
    vous finissez & que vous \keyword{finiss}iez \\
    ils \keyword{finiss}ent & qu'ils \keyword{finiss}ent \\
    \bottomrule[1.5pt]
  \end{tabular}
  \caption{Les radicaux}
\end{table}

b. Si le radical du présent de l'indicatif avec $\ll$ nous $\gg$ et $\ll$ vous $\gg$ est différent de celui avec $\ll$ ils $\gg$, on garde ce radical avec $\ll$ nous $\gg$ et $\ll$ vous $\gg$ au subjonctif.

\begin{table}[H]
  \centering
  \begin{tabular}{cc}
    \toprule[1.5pt]
    \keyword{Présent de l'indicatif} & \keyword{Subjonctif} \\
    \midrule[1.5pt]
    je bois & que je \keyword{boiv}e \\
    tu bois & que tu \keyword{boiv}es \\
    il boit & qu'il \keyword{boiv}e \\
    nous \keyword{buv}ons & que nous \keyword{buv}ions \\
    vous \keyword{buv}ez & que vous \keyword{buv}iez \\
    ils \keyword{boiv}ent & qu'ils \keyword{boiv}ent \\
    \bottomrule[1.5pt]
  \end{tabular}
  \caption{Les radicaux}
\end{table}

\subsection{Les verbes irrégullers}

\section{Indicatif futur simple}

\begin{itemize}
\item On ajoute les terminaisons $\ll$ -ai, -as, -a, -ons, -ez, -ont $\gg$ à l'infinitif du verbe.
\item Si l'infinitif finit par un $\ll$ E $\gg$, on enlève le $\ll$ E $\gg$ avant d'ajouter les terminaisons.
\item Il y a plusieurs exceptions.
\end{itemize}



\begin{table}[H]
  \centering
  \begin{tabular}{ccc}
    \toprule[1.5pt]
    aimer & boire & faire \\
    \midrule[1.5pt] 
    j'aimer\keyword{ai} & je boir\keyword{ai} & je fer\keyword{ai} \\ 
    tu aimer\keyword{as}      & tu boir\keyword{as} & tu fer\keyword{as} \\
    il aimer\keyword{a}      & il boir\keyword{a} & il fer\keyword{a} \\
    nous aimer\keyword{ons}     & nous boir\keyword{ons} & nous fer\keyword{ons} \\
    vous aimer\keyword{ez}&vous boir\keyword{ez}& vous fer\keyword{ez} \\
    ils aimer\keyword{ont}&ils boir\keyword{ont}& ils fer\keyword{ont}\\
    \bottomrule[1.5pt]{}
  \end{tabular}
  \caption{Futur simple}
\end{table}
\end{document}
%%% Local Variables:
%%% mode: latex
%%% TeX-master: t
%%% End:
