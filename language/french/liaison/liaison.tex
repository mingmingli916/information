\documentclass{article}
\input{~/latex/article-preamble.tex}
\begin{document}

\title{Liaison}
\date{}
\begin{titlepage}
  \maketitle{}
  \tableofcontents{}
\end{titlepage}

\section{Qu'est-ce qu'une Liaison ?}

La liaison en français consiste à prendre la dernière lettre d'un mot et à la relier au début du mot suivant.
Cela ne se produit que lorsque le deuxième mot commence par une voyelle.

En français, la liaison est une consonne muet à la fin d'un mot prononce avec une voyelle au début d'un mot.


Pourquoi diable feriez-vous cela?

Puisque les consonnes finales des mots français ne sont (généralement) pas prononcées, la plupart d'entre elles se terminent par un son de voyelle.
Cela signifie que pour passer de la fin d'un mot au début d'un mot avec une voyelle, il faut faire un coup de glotte.
Pour éviter d'avoir à faire ce son, vous prononcez la consonne finale du mot, permettant ainsi aux deux mots d'être "liés" ensemble.

\section{Comment prononcer les lettres dans les liaisons?}

Dans certains cas, la lettre à la fin du mot est simplement prononcée telle qu'elle est écrite.

Avec quelques autres lettres, cependant, la prononciation est légèrement modifiée.
\begin{itemize}
\item s [z]
\item d [t]
\item f [v]
\item x [z]
\end{itemize}


\section{Quand faites-vous la liaison ?}

La liaison française est liée à la grammaire : vous ne lierez que deux mots qui sont déjà liés grammaticalement.
Selon la façon dont les mots sont liés dans la phrase, la liaison peut être obligatoire, interdite ou facultative.


\subsection{Liaisons obligatoires}

\subsubsection{Groupe nominal}

\begin{itemize}
\item Article + nom ou adjectif (les amis \textipa{[lE za mi]}, les anciens élèves \textipa{[lE z\~a si j\~E ze lEv]})
\item Nombre + nom ou adjectif (deux enfants \textipa{[d\o \ z\~a f\~a]})
\item Adjectif + nom ou adjectif (les anciens élèves \textipa{[lE z\~a si j\~E ze lEv]}, petit ami \textipa{[p@ ti ta mi]})
\end{itemize}


\subsubsection{Groupe verbal}

\begin{itemize}
\item Pronom + pronom (Nous en avons \textipa{[nu z\~a na v\~O]})
\item Pronom + verbe ou adjectif (Nous en avons \textipa{[nu z\~a na v\~O]}, Vous avez \textipa{[nu za ve]})
\end{itemize}

\subsubsection{Adverbes, conjonctions et prépositions à une syllabe}

bien étrange \textipa{[bj\~E ne tr\~a Z]}

chez elle \textipa{[Se zEl]}

quand on décidera \textipa{[k\~a t\~O de si d@ ra]}

tout entier \textipa{[tu t\~a ti tje]}

très utile \textipa{[trE zy til]}	


\subsubsection{Quand + est-ce que }

Quand + est-ce que \textipa{[k\~a tEsk]}


\subsubsection{De nombreuses expressions fixes}

avant hier \textipa{[a v\~a tjEr]}

c’est-à-dire	\textipa{[sE ta dir]}

comment allez-vous ? \textipa{[cO m\~a ta le vu]}

plus ou moins	\textipa{[ply zu mw\~E]}


\subsection{Liaisons interdites}


\subsubsection{Avant h aspiré}

en haut \textipa{[\~a o]}

les héros \textipa{[lE e ro]}


\subsubsection{Avant onze et oui}

les onze enfants \textipa{[lE O z\~a f\~a]}

deux oui et un non \textipa{[d\o wi e \~a n\~O]}

\subsubsection{Après les noms}

Robert a 15 ans. \textipa{[rO bEr a k\~E z\~a]}

\subsubsection{Après les noms singuliers}

mon chat aime jouer \textipa{[m\~O Sa Em jue]}

un garçon intelligent \textipa{[\~a gar s\~O \~E tE li Z\~a]}


\subsubsection{Après et}

un homme et une femme \textipa{[\~a nOm e yn fam]}

\subsubsection{Après les adverbes interrogatifs et toujours}

Comment est-elle ?	\textipa{[kO m\~a e  tEl]}

Combien en vois-tu ?	\textipa{[k\~O bj\~E \~a vwa ty]}

Quand aimes-tu étudier ? \textipa{[k\~a Em ty e ty dje]}

toujours aimable  \textipa{[tu Zu rE mabl]}

\subsubsection{Après renversement}

A-t-on osé ?	\textipa{[a t\~O o ze]}

Parlez-vous espagnol ?	\textipa{[par le vu E spa \textltailn Ol]}

Ont-elles étudié ?	\textipa{[\~O tE le ty die]}

Vont-ils habiter bien ?	\textipa{[v\~O ti la bi te bj\~E]}

\subsection{Liaisons facultatives}

De manière générale, lorsque la liaison est facultative, la plupart des gens choisissent de ne pas le faire.

\end{document}