\documentclass{article}
\input{~/latex/article-preamble}

\begin{document}

\section{Indicatif Présent}

Il sert souvent à situer les faits au moment de l'énonciation.

\subsection{L'emploi du présent de l'indicatif}

Le présent peut exprimer:
\begin{itemize}
\item un fait qui s'accomplit au moment où l'on parle. (Il fait soleil ce matin.)
\item une vérité. (Deux et deux font quatre.)
\item un fait habituel. (Je quitte la maison tous les matins à 7 heures.)
\item Le présent peut annoncer un évènement qui aura lieu dans un futur rapproché. (Demain, je prends l'avion à 12 h 30.)
\item Le présent est utilisé après un ``si'' de condition dans une phrase dont le verbe principal peut être au présent, au futur simple, au conditionnel présent ou à l'impératif présent. (Si je veux partir en voyage, je dois économiser dès maintenant (présent).)
\end{itemize}


\section{Indicatif Future Proche}


Il exprime une action ou un état qui se produira dans un futur très rapproché. C’est un temps utilisé essentiellement dans le langage parlé.

\begin{tcolorbox}
aller + l'indicatif  
\end{tcolorbox}

Je vais rentre chez moi.

\section{Indicatif Passé Composé}

Il fait partie du mode indicatif. Il sert souvent à exprimer un fait accompli qui a eu lieu dans le passé.

Le passé composé est formé de l'auxiliaire avoir ou être au présent de l'indicatif et du partici​pe passé du verbe à conjuguer.

Utiliser:
\begin{itemize}
\item On utilise le passé composé pour exprimer un fait s'étant produit dans le passé et qui est terminé dans le présent. (J'ai mangé des fruits ce matin.)
\item Le passé composé peut servir à formuler une affirmation qui a toujours été vraie par le passé et qui le sera encore probablement dans le futur. (J'ai toujours respecté mes parents.)
\end{itemize}

\section{Indicatif Imparfait}

Il sert souvent à situer dans le passé un fait de longue durée, qui n'est pas terminé. 

Utiliser:
\begin{itemize}
\item L'imparfait traduit un fait non achevé et d'une durée non définie se déroulant au même moment qu'un autre. (Il pleuvait quand nous sommes arrivés.)
\item L'imparfait exprime une action passée qui s'est répétée dans le temps. (Chaque semaine, elle allait nager à la piscine municipale.)
\item L'imparfait est utilisé pour formuler une description associée à une réalité issue du passé. (Ce soir-là, elle portait une longue robe rouge.)
\item L'imparfait exprime une hypothèse irréalisable dans le présent, mais qui pourrait être réalisée dans l'avenir. (Si j'avais ce livre, je vous le donnerais.)
\end{itemize}
​

\section{Indicatif Passé Recent}

Le passé récent exprime une nuance par rapport aux autres temps du passé en situant les actions dans un temps antérieur, mais très proche.

\begin{tcolorbox}
  venir de + l'indicatif
\end{tcolorbox}

je viens de recevoir votre mail.



\end{document}




