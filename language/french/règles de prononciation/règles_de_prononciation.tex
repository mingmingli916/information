\documentclass{article}
\input{~/latex/article-preamble.tex}
\begin{document}

\tableofcontents{}
\newpage{}

\section{Les lettres individuelles}


\begin{table}[H]
  \centering
  \begin{tabular}{lll}
    \toprule[1.5pt]
    a & \textipa{[a]} & lama\textipa{[lama]}\\
    i & \textipa{[i]} & ni\textipa{[ni]}\\
    y & \textipa{[i]} & cyclone\textipa{[siklon]}\\
    u & \textipa{[y]} & unique\textipa{[ynik]}\\
    b & \textipa{[b]} & habit\textipa{[abi]}\\
    d & \textipa{[d]} & hardi\textipa{[ardi]}\\
    f & \textipa{[f]} & film\textipa{[film]}\\
    l & \textipa{[l]} & film\textipa{[film]}\\
    m & \textipa{[m]} & film\textipa{[film]}\\
    n & \textipa{[n]} & ni\textipa{[ni]}\\
    j & \textipa{[Z]} & jour\textipa{[Zur]}\\
    r & \textipa{[r]} & France\textipa{[fr\~as]}\\
    h & \textipa{[]} muet & habit\textipa{[abi]} \\
    v & \textipa{[v]} & vie\textipa{[vi]}\\
    z & \textipa{[z]} & bizarre\textipa{[bizar]}\\
    \bottomrule[1.5pt]
  \end{tabular}
  \caption{Le prononciation des lettres individuelles dans les mots}
\end{table}

\section{Quatre règles de prononciation}

\begin{itemize}
\item Le ``e'' est muet à la fin d'un mot, mais elle prononce \textipa{[@]} à la fin d'un mot très court, par exemple, me, le, te, se.
\item Les consonnes sont muettes à la fin d'un mot sauf pour ``c'', ``f'', ``l'', ``r'' et ``q''.
\item Le ``h'' est muet.
\item Deux lettres identiques font un son.
\end{itemize}

\section{Les lettres avec plusieurs prononciations}


\subsection{c}


\begin{table}[H]
  \centering
  \begin{tabular}{p{0.1\columnwidth}p{0.5\columnwidth}p{0.3\columnwidth}}
    \toprule[1.5pt]
    \textipa{[s]} & avant ``e'', ``i'' ou ``y'' & ceci\textipa{[s@si]}, cycle\textipa{[sikl]} \\
    \textipa{[k]} & pas avant ``e'', ``i'' et ``y'' & car\textipa{[kar]}, lac\textipa{[lak]} \\
    \bottomrule[1.5pt]
  \end{tabular}
  \caption{Prononciation de ``c''}
\end{table}

\subsection{g}

\begin{table}[H]
  \centering
  \begin{tabular}{p{0.1\columnwidth}p{0.5\columnwidth}p{0.3\columnwidth}}  
    \toprule[1.5pt]
    \textipa{[Z]} & avant ``e'', ``i'' ou ``y'' & girafe\textipa{[Ziraf]}, gel\textipa{[ZEl]} \\
    \textipa{[g]} & pas avant ``e'', ``i'' et ``y'' & gare\textipa{[gar]}, glace\textipa{[glas]} \\
    \textipa{[g]} & gu & digue\textipa{[dig]} \\
    \bottomrule[1.5pt]
  \end{tabular}
  \caption{Prononciation de ``g''}
\end{table}

\subsection{o}

\begin{table}[H]
  \centering
  \begin{tabular}{p{0.1\columnwidth}p{0.5\columnwidth}p{0.3\columnwidth}}  
    \toprule[1.5pt]
    \textipa{[O]} & en général & porte\textipa{[pOrt]}, sport\textipa{[spOr]} \\
    \textipa{[o]} & syllabe ouverte à la fin d'un mot & mot\textipa{[mo]} \\
    \textipa{[o]} & avant \textipa{[z]} & rose\textipa{[roz]} \\
    \bottomrule[1.5pt]
  \end{tabular}
  \caption{Prononciation de ``o''}
\end{table}

\subsection{q}

\begin{table}[H]
  \centering
  \begin{tabular}{p{0.1\columnwidth}p{0.5\columnwidth}p{0.3\columnwidth}}  
    \toprule[1.5pt]
    \textipa{[k]} & à la fin d'un mot & coq\textipa{[kOk]} \\
    \textipa{[k]} & qu & qui\textipa{[ki]}, que\textipa{[k@]} \\
    \bottomrule[1.5pt]
  \end{tabular}
  \caption{Prononciation de ``q''}
\end{table}

\subsection{s}

\begin{table}[H]
  \centering
  \begin{tabular}{p{0.1\columnwidth}p{0.5\columnwidth}p{0.3\columnwidth}}  
    \toprule[1.5pt]
    \textipa{[s]} & en général & si\textipa{[si]}, sel\textipa{[sEl]} \\
    \textipa{[z]} & entre deux voyelles & rose\textipa{[roz]}, pose\textipa{[poz]} \\
    \bottomrule[1.5pt]
  \end{tabular}
  \caption{Prononciation de ``s''}
\end{table}

\subsection{w}

\begin{table}[H]
  \centering
  \begin{tabular}{p{0.1\columnwidth}p{0.5\columnwidth}p{0.3\columnwidth}}    
    \toprule[1.5pt]
    \textipa{[w]} & mots étrangers & wifi\textipa{[wifi]}, walt\textipa{[walt]} \\
    \textipa{[v]} & en général & wagon\textipa{[vag\~O]} \\
    \bottomrule[1.5pt]
  \end{tabular}
  \caption{Prononciation de ``w''}
\end{table}

\subsection{x}

\begin{table}[H]
  \centering
  \begin{tabular}{p{0.1\columnwidth}p{0.5\columnwidth}p{0.3\columnwidth}}        
    \toprule[1.5pt]
    \textipa{[s]} & à la fin de mots (peu) & six\textipa{[sis]}, dix\textipa{[dis]} \\
    \textipa{[ks]} & en général & taxi\textipa{[taksi]} \\
    \textipa{[Egz]} & ex + voyelle &  exemple\textipa{[Egz\~apl]} \\
    \textipa{[Eks]} & ex + consonne & excuse\textipa{[Ekskyz]} \\
    \bottomrule[1.5pt]
  \end{tabular}
  \caption{Prononciation de ``x''}
\end{table}

\subsection{e}

\begin{table}[H]
  \centering
  \begin{tabular}{p{0.1\columnwidth}p{0.5\columnwidth}p{0.3\columnwidth}}        
    \toprule[1.5pt]
    \textipa{[E]} & dans syllabe fermée & sel\textipa{[sEl]}, mer\textipa{[mEr]} \\
    \textipa{[@]} & dans syllabe ouverte & ceci\textipa{[s@si]}, sela\textipa{[s@la]} \\
    \textipa{[E]} & avant deux consonnes identiques & pelle\textipa{[pEl]} \\
    \textipa{[a]} & avant ``mm'' & femme\textipa{[fam]} \\
    \textipa{[]}muet & voyelle + consonne + ``e'' + consonne + voyelle & samedi\textipa{[samdi]} \\
    \bottomrule[1.5pt]
  \end{tabular}
  \caption{Prononciation de ``e''}
\end{table}


\section{Les lettres avec l'accent}

\begin{table}[H]
  \centering
  \begin{tabular}{p{0.1\columnwidth}p{0.5\columnwidth}p{0.3\columnwidth}}        
    \toprule[1.5pt]
    é & \textipa{[e]} &  répéter\textipa{[repete]} \\
    è, ê, ë & \textipa{[E]} &  lève\textipa{[lEv]} \\
    î, ï & \textipa{[i]} &  maïs\textipa{[mais]} \\
    ô & \textipa{[o]} &  hôpital\textipa{[opital]} \\
    ç & \textipa{[s]} & français\textipa{[fr\~asE]} \\
    \bottomrule[1.5pt]
  \end{tabular}
  \caption{Les lettres avec l'accent}
\end{table}

\section{Les combinaisons de lettres}

\subsection{ai, ei, ay, ey}

\begin{table}[H]
  \centering
  \begin{tabular}{p{0.2\columnwidth}p{0.8\columnwidth}}
    \toprule[1.5pt]
    \textipa{[E]} & faire\textipa{[fEr]}, clair\textipa{[klEr]}, Seine\textipa{[sEn]}, seize\textipa{[sEz]} \\
    \bottomrule[1.5pt]
  \end{tabular}
  \caption{ai, ei, ay, ey}
\end{table}

\subsection{oi}

\begin{table}[H]
  \centering
  \begin{tabular}{p{0.2\columnwidth}p{0.8\columnwidth}}    
    \toprule[1.5pt]
    \textipa{[wa]} & soir\textipa{[swar]}, voir\textipa{[vwar]} \\
    \bottomrule[1.5pt]
  \end{tabular}
  \caption{oi}
\end{table}


\subsection{au, eau}


\begin{table}[H]
  \centering
  \begin{tabular}{p{0.2\columnwidth}p{0.8\columnwidth}}    
    \toprule[1.5pt]
    \textipa{[o]} & auto\textipa{[oto]}, aussi\textipa{[osi]}, au\textipa{[o]}, beau\textipa{[bo]} \\
    \bottomrule[1.5pt]
  \end{tabular}
  \caption{au, eau}
\end{table}


\subsection{eu}


\begin{table}[H]
  \centering
  \begin{tabular}{p{0.2\columnwidth}p{0.4\columnwidth}p{0.4\columnwidth}}
    \toprule[1.5pt]
    \textipa{[\o]} & en général & bleu\textipa{[bl\o]}, feu\textipa{[f\o]} \\
    \textipa{[\oe]} & avant ``r'' & leur\textipa{[l\oe r]} \\ 
    \bottomrule[1.5pt]
  \end{tabular}
  \caption{eu}
\end{table}




\subsection{\textipa{\oe u}}

\begin{table}[H]
  \centering
  \begin{tabular}{p{0.2\columnwidth}p{0.8\columnwidth}}
    \toprule[1.5pt]
    \textipa{[\oe]} & sœur\textipa{[s\oe r]}, bœuf\textipa{[b\oe f]}\\ 
    \bottomrule[1.5pt]
  \end{tabular}
  \caption{\textipa{\oe u}}
\end{table}

\subsection{gn}


\begin{table}[H]
  \centering
  \begin{tabular}{p{0.2\columnwidth}p{0.8\columnwidth}}
    \toprule[1.5pt]
    [\textltailn] & signal\textipa{[si\textltailn al]}, ligne\textipa{[li\textltailn]} \\
    \bottomrule[1.5pt]
  \end{tabular}
  \caption{gn}
\end{table}

\subsection{ou}

\begin{table}[H]
  \centering
  \begin{tabular}{p{0.2\columnwidth}p{0.8\columnwidth}}
    \toprule[1.5pt]
    [u] & amour\textipa{[amur]}, où[u], bisou[bizu], cou[ku], soupe[sup], coût[ku] \\
    \bottomrule[1.5pt]
  \end{tabular}
  \caption{ou}
\end{table}

\subsection{an, am, en, em}

\begin{table}[H]
  \centering
  \begin{tabular}{p{0.2\columnwidth}p{0.8\columnwidth}}
    \toprule[1.5pt]
    [\~a] & France[fr\~as], enfant[\~af\~a], dans[d\~a], enchanté\textipa{[\~aS\~ate]} \\
    \bottomrule[1.5pt]
  \end{tabular}
  \caption{an, am, en, em}
\end{table}

\begin{tcolorbox}
  ATTENTION:    
  \begin{itemize}
  \item Y a pas voyelle après: banane[banan]
  \item Y a pas ``m'' et ``n'' après: anneé[ane]
  \end{itemize}
\end{tcolorbox}

\subsection{om, on}


\begin{table}[H]
  \centering
  \begin{tabular}{p{0.2\columnwidth}p{0.8\columnwidth}}
    \toprule[1.5pt]
    \textipa{[\~O]} & mon\textipa{[m\~O]}, ton\textipa{[t\~O]}, pompe\textipa{[p\~Op]}, nombre\textipa{[n\~Obr]} \\
    \bottomrule[1.5pt]
  \end{tabular}
  \caption{om, on}
\end{table}

\begin{tcolorbox}
  ATTENTION:  
  \begin{itemize}
  \item Y a pas voyelle après: sonar[sonar]
  \item Y a pas ``m'' et ``n'' après: tonne[ton]
  \end{itemize}
\end{tcolorbox}

\subsection{im,  in, aim, ain, ein, yn, ym}

\begin{table}[H]
  \centering
  \begin{tabular}{p{0.2\columnwidth}p{0.8\columnwidth}}
    \toprule[1.5pt]
    \textipa{[\~E]} & important\textipa{[\~EpOrt\~a]}, prince\textipa{[pr\~Es]}, plein\textipa{[pl\~E]}, syndicat\textipa{[s\~Edika]} \\
    \bottomrule[1.5pt]
  \end{tabular}
  \caption{im, in, aim, ain, ein, yn, ym}
\end{table}

\begin{tcolorbox}
  ATTENTION:
  \begin{itemize}
  \item Y a pas voyelle après: timide[timid]
  \item Y a pas ``m'' et ``n'': immeuble\textipa[im\oe bl]
  \end{itemize}
\end{tcolorbox}


\subsection{um, un}

\begin{table}[H]
  \centering
  \begin{tabular}{p{0.2\columnwidth}p{0.8\columnwidth}}
    \toprule[1.5pt]
    \textipa{[\~\oe]} & un[\~\oe], lundi[l\~\oe di], chacun\textipa{[Sak\~\oe]}, parfum[parf\~\oe] \\
    \bottomrule[1.5pt]
  \end{tabular}
  \caption{um, un}
\end{table}

\begin{tcolorbox}
  ATTENTION:  
  \begin{itemize}
  \item Y a pas voyelle après: une[yn]
  \item Y a pas ``m'' et ``n'' après: tunnel\textipa{[tynEl]}
  \end{itemize}
\end{tcolorbox}


\subsection{tion}


\begin{table}[H]
  \centering
  \begin{tabular}{p{0.2\columnwidth}p{0.4\columnwidth}p{0.4\columnwidth}}
    \toprule[1.5pt]
    \textipa{[sj\~O]} & à la fin d'un mot & nation\textipa{[nasj\~O]} \\
    \textipa{[tj\~O]} & s + tion & question\textipa{[kEstj\~O]} \\
    \bottomrule[1.5pt]
  \end{tabular}
  \caption{tion}
\end{table}

\subsection{sion}

\begin{table}[H]
  \centering
  \begin{tabular}{p{0.2\columnwidth}p{0.8\columnwidth}}
    \toprule[1.5pt]
    \textipa{[sj\~O]} & passion\textipa{[pasj\~O]} \\
    \bottomrule[1.5pt]
  \end{tabular}
  \caption{sion}
\end{table}

\subsection{ien}


\begin{table}[H]
  \centering
  \begin{tabular}{p{0.2\columnwidth}p{0.4\columnwidth}p{0.4\columnwidth}}
    \toprule[1.5pt]
    \textipa{[j\~E]} & à la fin d'un mot & mien\textipa{[mj\~E]}, canadien\textipa{[kanadj\~E]} \\
    \textipa{[j\~a]} & en général & orient\textipa{[Orj\~a]}, patient\textipa{[pasj\~a]} \\    
    \bottomrule[1.5pt]
  \end{tabular}
  \caption{ien}
\end{table}

\begin{tcolorbox}
  ATTENTION:
  \begin{itemize}
  \item Y a pas voyelle après
  \item Parfois elle prononce \textipa{[j\~E]}: bienvenue\textipa{[bj\~Ev@ny]}
  \end{itemize}
\end{tcolorbox}


\subsection{tien}
\begin{table}[H]
  \centering
  \begin{tabular}{p{0.2\columnwidth}p{0.4\columnwidth}p{0.4\columnwidth}}
    \toprule[1.5pt]
    \textipa{[sj\~a]} & en général & patient\textipa{[pasj\~a]} \\    
    \bottomrule[1.5pt]
  \end{tabular}
  \caption{ien}
\end{table}

\begin{tcolorbox}
  ATTENTION:  
  \begin{itemize}
  \item Y a pas voyelle après.
  \end{itemize}

\end{tcolorbox}


\subsection{oin}


\begin{table}[H]
  ATTENTION:
  \centering
  \begin{tabular}{p{0.2\columnwidth}p{0.8\columnwidth}}
    \toprule[1.5pt]
    \textipa{[w\~E]} & loin\textipa{[lw\~E]}, point\textipa{[pw\~E]} \\
    \bottomrule[1.5pt]
  \end{tabular}
  \caption{oin}
\end{table}


\subsection{il}

\begin{table}[H]
  \centering
  \begin{tabular}{p{0.2\columnwidth}p{0.4\columnwidth}p{0.4\columnwidth}}
    \toprule[1.5pt]
    [j] & à la fin d'un mot & travail[travaj], fauteuil[fot\oe j] \\
    \bottomrule[1.5pt]
  \end{tabular}
  \caption{il}
\end{table}

\subsection{ill, ille}


\begin{table}[H]
  \centering
  \begin{tabular}{p{0.2\columnwidth}p{0.4\columnwidth}p{0.4\columnwidth}}
    \toprule[1.5pt]
    \textipa{[j]} & après voyelle & illtravailler[travaje], volaille\textipa{[vOlaj]} \\
    \textipa{[ij]} & après consonne &  famille[famij]\\
    \bottomrule[1.5pt]
  \end{tabular}
  \caption{ill, ille}
\end{table}

\subsection{ch, sch}


\begin{table}[H]
  \centering
  \begin{tabular}{p{0.2\columnwidth}p{0.8\columnwidth}}
    \toprule[1.5pt]
    \textipa{[S]} & Chine\textipa{[Sin]}, chinois\textipa{[Sinwa]}, schéma\textipa{[Sema]} \\
    \bottomrule[1.5pt]
  \end{tabular}
  \caption{ch, sch}
\end{table}


\subsection{ph}


\begin{table}[H]
  \centering
  \begin{tabular}{p{0.2\columnwidth}p{0.8\columnwidth}}
    \toprule[1.5pt]
    [f] & photo\textipa{[fOto]}, phare\textipa{[far]} \\
    \bottomrule[1.5pt]
  \end{tabular}
  \caption{ph}
\end{table}

\subsection{ent}

\begin{table}[H]
  \centering
  \begin{tabular}{p{0.2\columnwidth}p{0.4\columnwidth}p{0.4\columnwidth}}
    \toprule[1.5pt]
    [] muet & à la fin du mot (conjugaison des verbes) & parlent\textipa{[parl]} \\
    \bottomrule[1.5pt]
  \end{tabular}
  \caption{ent}
\end{table}

\subsection{er}

\begin{table}[H]
  \centering
  \begin{tabular}{p{0.2\columnwidth}p{0.4\columnwidth}p{0.4\columnwidth}}
    \toprule[1.5pt]
    [e] muet & à la fin du mot (l'infinitif) & parler\textipa{[parle]} \\
    \bottomrule[1.5pt]
  \end{tabular}
  \caption{er}
\end{table}

\subsection{tiel}



\begin{table}[H]
  \centering
  \begin{tabular}{p{0.2\columnwidth}p{0.4\columnwidth}p{0.4\columnwidth}}
    \toprule[1.5pt]
    \textipa{[sjEl]} & à la fin du mot & essentiel\textipa{[es\~asjEl]} \\
    \bottomrule[1.5pt]
  \end{tabular}
  \caption{tiel}
\end{table}


\end{document}