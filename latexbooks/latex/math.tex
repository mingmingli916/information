
\chapter{Math Formulas}
\label{cha:math-formulas}

\section{Math Mode}
\label{sec:math-mode}

Using math environments to enter math mode.

\subsection{Embedding Math Expressions Within Text}
\label{sec:embedd-math-expr}

LaTeX provides the math environment in-text formulas:
\begin{lstlisting}
\begin{math}
  expression
\end{math}
\end{lstlisting}


Since it's very laborious to write this environment for each small expression or symbol, LaTeX offers an alias that's doing the same:
\begin{lstlisting}
\(
expression
\)
% or
\(expression\)
\end{lstlisting}

A third way is by using a shortcut, coming from TeX:
\begin{lstlisting}
$expression$
\end{lstlisting}

\subsection{Displaying Formulas}
\label{sec:displaying-formulas}

For displayed formulas, which have to be centered, LaTeX offers the displaymath environment:
\begin{lstlisting}
\begin{displaymath}
  expression
\end{displaymath}
\end{lstlisting}

The effect of this environment is that the paragraph will be ended, some vertical space follows, then the centered formula plus the following vertical space.
As this math environment takes care of the spacing, don't leave empty lines before and after it!
This would cause additional vertical space because of the superfluous paragraph breaks.

A shortcut is:
\begin{lstlisting}
\[
expression
\]
\end{lstlisting}

\subsection{Numbering Equations}
\label{sec:numbering-equations}

Equations and formulas in general may be numbered.
However, this applies only to displayed formulas.
The equation environment is responsible for this:
\begin{lstlisting}
\begin{equation}
  \label{key}
  expression
\end{equation}
\end{lstlisting}

\section{Common Formulas}
\label{sec:common}

\begin{table}[H]
  \centering
  \begin{tabular}{ll}
    \toprule
    \head{Source code} & \head{Output}\\
    \midrule
    \lstinline|x^2| & $x^{2}$ \\
    \lstinline|x_2| & $x_{2}$ \\
    \lstinline|\sqrt[3]{x}| & $\sqrt[3]{x}$\\
    \lstinline|\frac{x}{y}| & $\frac{x}{y}$\\
    \bottomrule
  \end{tabular}
  \caption{Common}
  \label{tab:common}
\end{table}


\section{Dots}
\label{sec:dots}

\begin{table}[H]
  \centering
  \begin{tabular}{>{\textbackslash\ttfamily}ll}
    \toprule
    \normal{\head{Source code}} & \head{Output}\\
    \midrule
    ddots & $\ddots$\\
    cdot & $\cdot$\\
    ldots & $\ldots$\\
    vdots & $\vdots$\\
    dot\{\} & $\dot{}$\\
    cdots & $\cdots$\\
    \bottomrule
  \end{tabular}
  \caption{Dots}
  \label{tab:dots}
\end{table}


\section{Greek Letters}
\label{sec:greek-letters}

To get a lowercase Greek letter, just write the name with a backslash for the command.

\begin{table}[H]
  \centering
  \begin{tabular}{>{\textbackslash\ttfamily}ll>{\textbackslash\ttfamily}ll}
    \toprule
    \normal{\head{Source code}} & \head{Output} & \normal{\head{Source code}} & \head{Output}\\
    \midrule
    alpha & $\alpha$ & beta & $\beta$\\
    gamma & $\gamma$ &  delta & $\delta$\\
    epsilon & $\epsilon$ & zeta & $\zeta$\\
    eta & $\eta$ & theta & $\theta$\\
    iota & $\iota$ & kappa & $\kappa$\\
    lambda & $\lambda$ & mu & $\mu$\\
    nu & $\nu$ & xi & $\xi$\\
    \normal{o} & o & pi & $\pi$\\
    rho & $\rho$ & sigma & $\sigma$\\
    tau & $\tau$ & upsilon & $\upsilon$\\
    phi & $\phi$ & chi & $\chi$\\
    psi & $\psi$ & omega & $\omega$\\
    \midrule
    varepsilon & $\varepsilon$ & vartheta & $\vartheta$\\
    varpi & $\varpi$ & varrho & $\varrho$\\
    varsigma & $\varsigma$ & varphi & $\varphi$\\
    \midrule
    Gamma & $\Gamma$ & Delta & $\Delta$\\
    Theta & $\Theta$ & Lambda & $\Lambda$\\
    Xi & $\Xi$ & Pi & $\Pi$\\
    Sigma & $\Sigma$ & Upsilon & $\Upsilon$\\
    Phi & $\Phi$ & Psi & $\Psi$\\
    Omega & $\Omega$\\
    \bottomrule
  \end{tabular}
  \caption{Greek Letters}
  \label{tab:greek-letters}
\end{table}




\section{Fonts}
\label{sec:fonts}

\begin{table}[H]
  \centering
  \begin{tabular}{>{\textbackslash{}\ttfamily{}}l<{\{...\}}ll}
    \toprule
    \normal{\head{Source code}} & \head{Package} & \head{Output}\\
    \midrule
    mathrm &  & $\mathrm{abc\ 123}$\\
    mathit & & $\mathit{abc\ 123}$\\
    mathsf & & $\mathsf{abc\ 123}$\\
    mathbb & amsfonts & $\mathbb{ABC}$\\
    mathbbm & bbm & $\mathbbm{ABC}$\\
    mathds & dsfont & $\mathds{ABC}$\\
    mathfrak & eufrak & $\mathfrak{ABC\ 123}$\\
    mathnormal & & $\mathnormal{ABC\ 123}$\\
    \bottomrule
  \end{tabular}
  \caption{Fonts}
  \label{tab:fonts}
\end{table}

\section{Multi-line Formulas}
\label{sec:multi-line-formulas}

\begin{lstlisting}
% package amsmath needed
\begin{multline}
\sum = a + b + c + d + e \\
           + f + g + h + i + j \\
           + k + l + m + n 
\end{multline}


\begin{gather}
x + y + z = 0 \\ 
y-z= 1
\end{gather}


\begin{align}
  x + y + z &= 0 \\
  y - z &= 1
\end{align}
\end{lstlisting}


\begin{multline}
\sum = a + b + c + d + e \\
           + f + g + h + i + j \\
           + k + l + m + n 
\end{multline}


\begin{gather}
x + y + z = 0 \\ 
y-z= 1
\end{gather}


\begin{align}
  x + y + z &= 0 \\
  y - z &= 1
\end{align}

\section{Operators}
\label{sec:operators}

Trigonometric functions, logarithm functions, and other analytic and algebraic functions are commonly written with upright Roman letters.
Simply typing log would otherwise look like a product of the three variables, namely, l, o, and g.
To ease the input, there are commands for many common functions or so called \keyword{operators}.
Here's an alphabetical list of the predefined ones:
\begin{lstlisting}
\arccos, \arcsin, \arctan, \arg, \cos, \cosh, \cot, \coth, \scs, \deg, \det, \dim, \exp, \gcd, \hom, \inf, \ker, \lg, \lim, \liminf, \limsup, \ln, \log, \max, \min, \Pr, \sec, \sin, \sinh, \sup, \tan, \tanh
\end{lstlisting}

\newpage{}
\section{Standard LaTeX Symbols}
\begin{table}[H]
  \centering
  \begin{tabular}{>{\textbackslash\ttfamily}ll>{\textbackslash{}\ttfamily{}}ll}
    \toprule
    \normal{\head{Source code}} & \head{Output} & \normal{\head{Source code}} & \head{Output}\\
    \midrule
    circ & $\circ$ & bigcirc & $\bigcirc$\\
    star & $\star$ & ast & $\ast$\\
    cup & $\cup$ & cap & $\cap$\\
    ominus & $\ominus$ & oplus & $\oplus$\\
    oslash & $\oslash$ & otimes & $\otimes$\\
    times & $\times$ & div & $\div$\\
    pm & $\pm$ & mp & $\mp$\\
    odot & $\odot$ & bullet & $\bullet$\\
    \midrule
    approx & $\approx$ & equiv & $\equiv$\\
    propto & $\propto$ & sim & $\sim$\\
    simeq & $\simeq$\\
    parallel & $\parallel$ & perp & $\perp$\\
    subset & $\subset$ & supset & $\supset$\\
    subseteq & $\subseteq$ & supseteq & $\supseteq$\\
    \midrule
    geq & $\geq$ & gg & $\gg$\\
    leq & $\leq$ & ll & $\ll$\\
    neq & $\neq$\\
    \midrule
    prod & $\prod$ & sum & $\sum$ \\
    coprod & $\coprod$ & int & $\int$\\
    oint & $\oint$\\
    \midrule
    rightarrow & $\rightarrow$ & Rightarrow & $\Rightarrow$\\
    longrightarrow & $\longrightarrow$ & Longrightarrow & $\Longrightarrow$\\
    hookleftarrow & $\hookleftarrow$ & hookrightarrow & $\hookrightarrow$\\
    leftrightarrow & $\leftrightarrow$ & Leftrightarrow $\Longleftrightarrow$\\
    \midrule
    bot & $\bot$ & forall & $\forall$\\
    ni & $\ni$ & top & $\top$\\
    hbar & $\hbar$ & in & $\in$\\
    exists & $\exists$ \\
    \midrule
    langle & $\langle$ & lceil & $\lceil$\\
    lfloor & $\lfloor$ & | & $\|$\\
    \midrule
    sharp & $\sharp$ & nabla & $\nabla$\\
    emptyset & $\emptyset$ & angle & $\angle$\\
    flat & $\flat$ & neg & $\neg$\\
    surd & $\surd$ & infty & $\infty$\\
    prime & $\prime$ & triangle & $\triangle$\\
    \bottomrule
  \end{tabular}
  \caption{Standard latex sysmbols}
  \label{tab:standard-latex-symbols}
\end{table}

\section{Math Structures}
\label{sec:math-structures}

\begin{lstlisting}
\[
\binom{n}{k} = \frac{n!}{k!(n-k)!}
\]
\end{lstlisting}

\[
\binom{n}{k} = \frac{n!}{k!(n-k)!}
\]


\begin{lstlisting}
\[
A = 
\begin{pmatrix}
  a_{11} & a_{12} \\
  a_{21} & a_{22}
\end{pmatrix}
\]
\end{lstlisting}

\[
A =
\begin{pmatrix}
  a_{11} & a_{12} \\
  a_{21} & a_{22}
\end{pmatrix}
\]

These are amsmath's matrix environments:

\begin{table}[H]
  \centering
  \begin{tabular}{>{\ttfamily}ll}
    \toprule
    \normal{\head{Name}} & \head{Delimiters of the matrix}\\
    \midrule
    matrix & no delimiters\\
    pmatrix & parentheses()\\
    bmatrix & square brackets[]\\
    Bmatrix & braces\{\}\\
    vamtrix & $|$\\
    Vmatrix & $\|$$\|$\\
    smallmatrix & without delimiters, add them if needed, more compact\\
    \bottomrule
  \end{tabular}
  \caption{Matrix}
  \label{tab:matrix}
\end{table}

\section{Stakcing Expressions}
\label{sec:stakcing-expressions}

\subsection{Underlining and Overlining}
\label{sec:underl-overl}

\begin{lstlisting}
\[s = \overline{AB}\]
\[s = \underline{AB}\]
\[N = \underbrace{1 + 1 + \cdots + 1}_n\]
\[N = \overbrace{1 + 1 + \cdots + 1}_n\]
\end{lstlisting}
\[s = \overline{AB}\]
\[s = \underline{AB}\]
\[N = \underbrace{1 + 1 + \cdots + 1}_n\]
\[N = \overbrace{1 + 1 + \cdots + 1}_n\]

\subsection{Setting Accents}
\label{sec:setting-accents}

\begin{table}[H]
  \centering
  \begin{tabular}{>{\textbackslash\ttfamily}l<{\{a\}}l>{\textbackslash{}\ttfamily{}}l<{\{a\}}l}
    \toprule
    \normal{\head{Source code}} & \head{Output} & \normal{\head{Source code}} & \head{Output}\\
    \midrule
    bar & $\bar{a}$ & acute & $\acute{a}$\\
    check & $\check{a}$ & grave & $\grave{a}$\\
    tilde & $\tilde{a}$ & ddot & $\ddot{a}$\\
    hat & $\hat{a}$ & vec & $\vec{a}$\\
    breve & $\breve{a}$ & dot & $\dot{a}$\\
    mathring & $\mathring{a}$\\
    widehat & $\widehat{abc}$ & widetilde & $\widetilde{a}$\\
    \bottomrule
  \end{tabular}
  \caption{Accents}
  \label{tab:accents}
\end{table}


\subsection{Puting a Symbol Above Another}
\label{sec:puting-symbol-above}

\begin{lstlisting}
% package amsmath

\[\underset{A}{\equiv}\]
\[\overset{\equiv}{A}\]
\end{lstlisting}

\[\underset{A}{\equiv}\]
\[\overset{\equiv}{A}\]

%%% Local Variables:
%%% mode: latex
%%% TeX-master: "latex"
%%% End:
