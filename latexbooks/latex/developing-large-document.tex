
\chapter{Developing Large Documents}
\label{cha:devel-large-docum}


\section{Spliting the Input}
\label{sec:spliting-input}


Using ``divide and conquer'' thought to develop large documents, i.e. break down a document into several sub-documents.

There are two common commands to combine the sub-documents into a large document:
\begin{lstlisting}
\input{filename}
\include{filename}
\end{lstlisting}

When LaTeX encounters \lstinline|\input| command, it reads in the file with the name \argument{filename} exactly as if its contents have been typed at that point.
Accordingly, all commands in this file would be processed by the LaTeX compiler.
You can even nest \lstinline|\input| — this command may be used inside an included file.

The argument is treated the same way as \lstinline|\input|.
However, there are some important differences:
\begin{enumerate}
\item \lstinline|\include| implicitly starts new pages. \lstinline|\include{filename}| behaves like:
\begin{lstlisting}
\clearpage
\include{filename}
\clearpage
\end{lstlisting}
  
\item \lstinline|\include| cannot be nested.

\item \lstinline|\include| supports a mechanism of choosing which parts of the document you wish to compile (\lstinline|\includeonly|).
\end{enumerate}


\begin{lstlisting}
\includeonly{file list}
\end{lstlisting}

The argument may be a comma-separated list of filenames.
If a file, \argument{name.tex}, is not specified within this argument, \lstinline|\include{name}| would not insert this file but just behave like \clearpage instead.
This allows excluding chunks or whole chapters from compiling.
If you work on a huge document, this speeds up compilation if you choose to include just your current chapter while keeping the labels and references of the excluded chapter this way.

\section{Creating Front and Back Matter}
\label{sec:creating-front-back}

Books often begin with introductory material such as copyright information, a foreword, acknowledgements, or a dedication. This part, including the title page and the table of contents, is called the \keyword{front matter}.
At the end, a book might include an afterword and supporting material like a bibliography, and an index. This part is called the \keyword{back matter}.


\begin{lstlisting}
\documentclass{article}
\begin{document}

% Pages are numbered with lowercase Roman numbers.
% Chapters generate a table of contents entry but don't get a number.
\frontmatter 
\chapter*{Dedication}

我学习项目管理的目的有三个:
\begin{itemize}
\item 考取PMP证书。
\item 开阔自己的视野,从更广的角度去思考问题。
\item 指导自己的生活,将项目管理的知识应用于自己的生活中。
\end{itemize} 
\tableofcontents 
\listoftables
\listoffigures

% Pages are numbered with Arabic numbers.
% Chapters are numbered and produce a table of contents entry.
\mainmatter 
\include{chapter1} 
\include{chapter2} 

% Pages are numbered with Arabic numbers.
% Chapters generate a table of contents entry but don't get a number.
\backmatter
\include{proofs}                
\nocite{*} 
\bibliographystyle{alpha} 
\bibliography{tex}              % use tex.bib

\end{document}

\end{lstlisting}

\section{Creating a Title Page}
\label{sec:creating-title-page}


\begin{lstlisting}

\begin{titlepage}

\newcommand{\HRule}{\rule{\linewidth}{0.5mm}} % Defines a new command for the horizontal lines, change thickness here

\center % Center everything on the page
 
%----------------------------------------------------------------------------------------
%	HEADING SECTIONS
%----------------------------------------------------------------------------------------

\includegraphics[width=0.5\textwidth]{images/logo.png}\\[1cm] % Include a department/university logo - this will require the graphicx package

%----------------------------------------------------------------------------------------
%	TITLE SECTION
%----------------------------------------------------------------------------------------

\HRule \\[0.4cm]
{ \huge \bfseries \LaTeX}\\[0.4cm] % Title of your document
\HRule \\[1.5cm]
 
%----------------------------------------------------------------------------------------
%	AUTHOR SECTION
%----------------------------------------------------------------------------------------

\begin{minipage}{0.4\textwidth}
\begin{center} \large
Mingming \textsc{Li}\\ % Your name
\end{center}

\end{minipage}\\[2cm]


%----------------------------------------------------------------------------------------
%	DATE SECTION
%----------------------------------------------------------------------------------------

{\large \today}\\[2cm] % Date, change the \today to a set date if you want to be precise

\vfill % Fill the rest of the page with whitespace

\end{titlepage}

%%% Local Variables:
%%% mode: latex
%%% TeX-master: "latex"
%%% End:

\end{lstlisting}

%%% Local Variables:
%%% mode: latex
%%% TeX-master: "latex"
%%% End:
