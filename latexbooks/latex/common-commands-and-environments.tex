
\chapter[Commands and Environments]{Common Commands and Environments}

\section{Commands}
\label{sec:commands}


\lstset{language=TeX}
\begin{lstlisting}
% produces some space.
\quad                           

% ended a line.
\\ or \newline                  

% prevents a line break at the current position.
\nolinebreak                    

%  `` and '' is the quotation in latex. 
``hello''                       

% ragged left  
{\raggedleft Example text}      
% ragged right
{\raggedright Example text}     
% centering 
{\centering Example text}        


% tells LaTeX to produce a file with the extension .toc. 
% This file will be used to generate a table of contents. 
% We had to typeset twice: in the first run, the .toc file
% was written and in the second run, LaTeX read it and processed it.
\tableofcontents{}              


% causes a page break. Furthermore, the text has been 
% stretched to fill the page down to the bottom.
\pagebreak{}
% breaks the page as well, but it doesn't stretch the
% text: the remaining space of the page will stay empty.
\newpage{}
% forbids page breaking
\nopagebreak{}


% to squeeze more text onto a page. 
\enlargethispage{2\baselineskip}



% placed a superscripted number at the current position.
% Further, it prints its argument text into the bottom of 
% the page, marked by the same number
\footnote{text}
% modify the line that separates footnotes from the text 
% is produced by the command \footnoterule.
\renewcommand{\footnoterule}{\noindent\smash{\rule[3pt]{\textwidth}{0.4pt}}}

\rule[raising]{width}{height}
% draw a line 1pt at thick, as wide as text, raised a bit by 3 pt
\rule[3pt]{\textwidth}{1pt}

% \smash, we let our line pretend to have a height and a depth of 
zero, so it's occupying no vertical space at all


% when we use \footnote inside an argument, there will be an error. 
% \protect simply prevents this processing error. 
\protect{}
\section{Section \protect{\footnote{text}}}
% to avoid footnote appearing in heading and table of content
\section[title without footnote]{Section \protect{\footnote{text}}}



% ends the current page and causes all already defined figures and
tables to be printed out.
\clearpage{}

\cleardoublepage{}


% To be able to refer to a certain point, we have to mark it by a label.
% We can reference to the name of that label afterwards.
% notice , typeset twice to produce the cross reference
\label % marks the position
\ref % prints the number of the element we refer to 
\pageref % prints the page number of that element


\end{lstlisting}


\section{Environments}
\label{sec:environments}


\begin{lstlisting}
% quote long text
\begin{quotation}
\end{quotation}
\end{lstlisting}

\begin{lstlisting}
% center environment
\begin{center}
\end{center}
\end{lstlisting}


%%% Local Variables:
%%% mode: latex
%%% TeX-master: "latex"
%%% End:
