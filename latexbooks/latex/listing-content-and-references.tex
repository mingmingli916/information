
\chapter{Listing Content and References}
\label{cha:list-cont-refer}



\section{Table of Content}
\label{sec:table-content}

\begin{table}[H]
  \centering
  \begin{tabular}{>{\textbackslash\ttfamily}ll}
    \toprule
    \normal{\head{Command}} & \head{Level}\\
    \midrule
    part & -1 (\argument{book} and \argument{report} class)\\
    chapter & 0 (not available in \argument{article})\\
    section & 1\\
    subsection & 2\\
    subsubsection & 3\\
    paragraph & 4\\
    subparagraph & 5\\
    \bottomrule
  \end{tabular}
  \caption{Depth of the TOC}
  \label{tab:depth-of-toc}
\end{table}


There's a variable representing the level, namely, \lstinline|\tocdepth|.
It's an integer variable which we call a \keyword{counter}.

There are two basic ways to adjust a counter value:
\begin{lstlisting}
% specifies an integer value of 'n' for the counter 'name'.
\setcounter{name}{n}            
% adds the integer value of 'n' to value of the counter 'name'. 'n' may be negative.
\addtocounter{name}{n}          


\setcounter{tocdepth}{3}
% you can raise or lower the level without knowing its number.
\addtocounter{tocdepth}{1}
\end{lstlisting}


\subsection{Adding entries manually}
\label{sec:adding-entr-manu}

\begin{lstlisting}
% file extension:
% toc: table of contents file
% lof: list of figures file
% lot: list of tables file

% sectional unit: part, chapter, section, subsection, paragraph, subparagraph
\addcontentsline{file extension}{sectional unit}{text}

% In contrary to \addcontentsline, the argument entry is written directly to the file
% without any additional formatting. You may choose any formatting you like.
\addtocontents{file extension}{entry}


% Examples
\addcontentsline{toc}{chapter}{Preface}
\addcontentsline{toc}{part}{Appendix}

\addtocontents{toc}{\bigskip}
% extends the text height such that one additional line fits to the contents page.
\addtocontents{toc}{\protect\enlargethispage{\baselineskip}}
% causes a page break in the TOC.
\addtocontents{toc}{\protect\newpage} 
% changes the page style of the current TOC page to fancy.
\addtocontents{toc}{\protect\thispagestyle{fancy}} 

\end{lstlisting}



\section{Creating and Customizing Lists of Figures}
\label{sec:creat-cust-lists}

\begin{lstlisting}
% renamed the figures and the list heading 
\renewcommand{\figurename}{Diagram}
\renewcommand{\listfigurename}{List of Diagrams}
\listoffigures
\end{lstlisting}


\section{Creating and Customizing Lists of Tables}
\label{sec:creat-cust-lists-1}

\begin{lstlisting}
\renewcommand{\tablename}{Diagram}
\renewcommand{\listtablename}{List of Diagrams}
\listoffigures
\end{lstlisting}


\section{Generating an Index}
\label{sec:generating-an-index}

Steps to generating index list:
\begin{enumerate}
\item load the index package and add the command to create the index
\begin{lstlisting}
\usepackage{index}
\makeindex{}
\end{lstlisting}

\item mark index point
\begin{lstlisting}
% simple entry
\index{entry}
% example \index{enterprise}                   

% subentry
\index{entry!subentry}          
% example \index{enterprise!organization}

% subsubentry
\index{entry!subentry!subsubentry} 
% example \index{enterprise!organization!operation}

This will be written to a file with the extension .idx.
\end{lstlisting}

\item create an entry for the index for the table of contents
\begin{lstlisting}
\clearpage
\addcontentsline{toc}{chapter}{Index}
\end{lstlisting}

\item in the next line, typeset the index
\begin{lstlisting}
\printindex{}
\end{lstlisting}

\item use shell command to typeset the tex document \label{item:1}
\begin{lstlisting}[language=sh]
xelatex latex.tex               # .tex is optional
# or
pdflatex latex.tex              # .tex is optional
\end{lstlisting}
  
\item use shell command to produce \argument{.idx} file.
\begin{lstlisting}[language=sh]
makeindex latex.idx             # .idx is optional
\end{lstlisting}
  
\item typeset the tex document again, refer to \ref{item:1}
  
\end{enumerate}


\subsection{Specifying Page Ranges}
\label{sec:spec-page-rang}

\begin{lstlisting}
% Example
\index{network|(}
...
\index{network|)}
\end{lstlisting}


\subsection[Symbols in the Index]{Using Symbols in the Index}
\label{sec:using-symbols-macros}

\argument{makeindex} sorts the entries alphabetically.
If you would like to include symbols in the index, for example, Greek letters, chemical formulas, or math symbols, you could face the problem of integrating them into the sorting.
For this purpose, \lstinline|\index| understands a sort key.
Use this key as prefix for the entry, separated by an @ symbol, for instance:
\begin{lstlisting}
\index{Gamma@$\Gamma$}
\end{lstlisting}

\subsection{Referring to Other Index Entries}
\label{sec:referr-other-index}


Different words may stand for the same concept.
For such cases, it's possible to add a cross-reference to the main phrase without a page number.
Adding the code \lstinline.see{entry list}. achieves that, for example:
\begin{lstlisting}
\index{network|see{WLAN}}
\index{WLAN}
\end{lstlisting}

As such references don't print a page number, their position in the text doesn't matter. You could collect them in one place of your document.


\subsection{Fine-tuning Page Numbers}
\label{sec:fine-tuning-page}

If an index entry refers to several pages, you might want to emphasize one page number to indicate it as the primary reference.
You could define a command for emphasizing as follows:
\begin{lstlisting}
\newcommand{\main}[1]{\emph{#1}}
\index{WLAN|main}
\end{lstlisting}


\subsection{Designing the Index Layout}
\label{sec:design-index-layo}

LaTeX provides some index styles called \keyword{latex} (the default), \keyword{gind, din}, and \keyword{iso}.
To use another style, specify it using the \argument{–s} option of the makeindex program, for example:
\begin{lstlisting}[language=sh]
makeindex -s iso latex
\end{lstlisting}


\section{Creating a Bibliography}
\label{sec:creat-bibl}

\begin{lstlisting}
\begin{thebibliography}{widest label}
   \bibitem[label]{key} author, title, year etc.
   \bibitem...
   ...
\end{thebibliography}
\end{lstlisting}


\begin{lstlisting}
% Example
To study \TeX\ in depth, see \cite{DK86}. 
For writing math texts, see \cite{DK89}.


\begin{thebibliography}{8}
\bibitem{DK86} D.E. Knuth, \emph{The {\TeX}book}, 1986
\bibitem{DK89} D.E. Knuth, \emph{Typesetting Concrete Mathematics}, 1989
\end{thebibliography}
\end{lstlisting}

Each item is specified using the command \lstinline|\bibitem|.
This command requires a mandatory argument determining the \argument{key}.
We may simply refer to this key by \lstinline|\cite{key}| or \lstinline|\cite{key1,key2}|.
\lstinline|\cite| accepts an optional argument stating a page range, for example, \lstinline|\cite[p.\,18--20]{key}|.
You may choose a label by the optional argument of \lstinline|\bibitem|.
If no label has been given, LaTeX will number the items consecutively in square brackets.


\subsection[Bibtex]{Using Bibliography Databases With Bibtex}
\label{sec:using-bibl-datab}


\begin{enumerate}
\item Create a new document. For example \argument{latex.bib}.
\begin{lstlisting}
@book{DK86,
author = "D.E. Knuth",
title = "The {\TeX}book",
publisher = "Addison Wesley",
year = 1986
}

@article{DK89,
author = "D.E. Knuth",
title = "Typesetting Concrete Mathematics",
journal = "TUGboat",
volume = 10,
number = 1,
pages = "31--36",
month = apr,
year = 1989
}
\end{lstlisting}

\item Include the database in to your tex document. For example \argument{latex.tex}.
\begin{lstlisting}
\bibliographystyle{alpha}       % plain, unsrt, alpha, abbrv
\bibliography{latex}            % latex stands for latex.bib here
\end{lstlisting}
  
\item Typeset one time with \keyword{pdfLaTeX} or \keyword{xelatex}.
\begin{lstlisting}
xelatex latex
\end{lstlisting}
  
\item \argument{bibtex} document.
\begin{lstlisting}[language=sh]
bibtex latex                    # here latex is the documentname
\end{lstlisting}
  
\item Typeset again the tex file.
\end{enumerate}

\section{Changing the Headings}
\label{sec:changing-headings}

You can use \lstinline|\renewcommand| to change the headings.


\begin{table}[H]
  \centering
  \begin{tabular}{l>{\textbackslash\ttfamily}ll}
    \toprule
    \head{List} & \normal{\head{Heading Command}} & \head{Default heading}\\
    \midrule
    Table of contents & contentsname & \keyword{Contents}\\
    List of figures & listfigurename & \keyword{List of figures}\\
    List of tables & listtablename & \keyword{List of tables}\\
    \multirow{2}{*}{Bibliography} & bibname in book and report & \keyword{Bibliography} in book and report\\
                & refname in article & \keyword{References} in article\\
    Index & indexname & \keyword{index}\\
    \bottomrule
  \end{tabular}
  \caption{Headings name}
  \label{tab:headings}
\end{table}

\begin{table}[H]
  \centering
  \begin{tabular}{l>{\textbackslash\ttfamily}l>{\bfseries}l}
    \toprule
    \head{Name} & \normal{\head{Heading Command}} & \head{Default heading}\\
    \midrule
    figure & figurename & Figure\\
    table & tablename & Table\\
    part & partname & Part\\
    chapter & chaptername & Chapter\\
    abstract & abstractname & Abstract\\
    appendix & appendixname & Appendix\\
    \bottomrule

  \end{tabular}
  \caption{Macros name}
  \label{tab:macros-name}
\end{table}


%%% Local Variables:
%%% mode: latex
%%% TeX-master: "latex"
%%% End:
