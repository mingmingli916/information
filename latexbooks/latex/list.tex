
\chapter{Lists}
\label{cha:lists}


\section{Bulleted Lists}

\begin{lstlisting}
  \begin{itemize}
  \item geometry
  \item amsmath
  \end{itemize}
\end{lstlisting}


\begin{itemize}
\item geometry
\item amsmath
\end{itemize}


\section{Numbered Lists}
\begin{lstlisting}
  \begin{enumerate}
  \item geometry
  \item amsmath
  \end{enumerate}
\end{lstlisting}


\begin{enumerate}
\item geometry
\item amsmath
\end{enumerate}




\section{Definition Lists}
\begin{lstlisting}
  \begin{description}
  \item[paralist] provides compact lists and list versions that
    can be used within paragraphs, helps to customize labels and
    layout
  \item[enumitem] gives control over labels and lengths
    in all kind of lists
  \item[mdwlist] is useful to customize description lists, it
    even allows multi-line labels. It features compact lists and
    the capability to suspend and resume.
  \item[desclist] offers more flexibility in definition list
  \item[multenum] produces vertical enumeration in multiple
    columns
  \end{description}
\end{lstlisting}


\begin{description}
\item[paralist] provides compact lists and list versions that
  can be used within paragraphs, helps to customize labels and
  layout
\item[enumitem] gives control over labels and lengths
  in all kind of lists
\item[mdwlist] is useful to customize description lists, it
  even allows multi-line labels. It features compact lists and
  the capability to suspend and resume.
\item[desclist] offers more flexibility in definition list
\item[multenum] produces vertical enumeration in multiple
  columns
\end{description}


%%% Local Variables:
%%% mode: latex
%%% TeX-master: "latex"
%%% End:
