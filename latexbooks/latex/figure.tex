\chapter{Figure}

\argument{graphicx} is need to insert figure into you document.


LaTeX supports the following file types:
\begin{itemize}
\item \argument{PNG, JPG}, and \argument{PDF} if you directly compile
  to PDF (\keyword{pdfLaTeX})
\item \argument{EPS} if you compile to DVI and convert to PS and PDF (traditional LaTeX)
\end{itemize}


\begin{tcolorbox}
  \begin{itemize}
  \item PS: PostScript
  \item EPS: Encapsulated PostScript
  \item DVI: Device Independent Format
  \end{itemize}
\end{tcolorbox}


You don't need to specify a filename extension, it will be
automatically added.
Don't use blanks in the filename or path!
Blanks and special characters may cause problems with \lstinline|\includegraphics|.
If such symbols in filenames are required, load the package
\argument{grffile} to try to fix it.


\begin{lstlisting}
  \usepackage[demo]{graphicx}
  \begin{figure}
    \centering
    \includegraphics{test}
    \caption{Test figure}
  \end{figure}
\end{lstlisting}

Because we specified the \argument{demo} option, \argument{graphicx}
doesn't require a file \argument{test.png} or any other file; instead
it's just printing a black filled rectangle. This is useful for
testing or if you would like to discuss a LaTeX problem in an online
forum, but don't wish to publish your pictures.


\begin{lstlisting}
  % syntax
  \includegraphics[k=v]{file name}
\end{lstlisting}

Here are the most popular ones:
\begin{itemize}
\item \argument{width}: \lstinline|width=0.8\textwidth|
\item \argument{height}
\item \argument{scale}: \lstinline|scale=0.5|
\item \argument{angle}: \lstinline{angle=90}
\end{itemize}





