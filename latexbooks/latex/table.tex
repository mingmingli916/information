\chapter{Tables}

\section[Table]{tabular}
\label{sec:tabular}


LaTeX provides the \argument{tabular} environment for typesetting simple and complex tables which can be nested.


\begin{lstlisting}
  \newcommand{\head}[1]{\textnormal{\textbf{#1}}}
  \begin{tabular}{ccc}
    \hline                      % Draw a horizontal line over whole width of the table
    \head{Command} & \head{Declaration} & \head{Output} \\
    \hline
    \verb|\textrm| & \verb|\rmfamily| & \rmfamily Example text \\
    \verb|\textsf| & \verb|\sffamily| & \sffamily Example text \\
    \verb|\texttt| & \verb|\ttfamily| & \ttfamily Example text \\
    \hline
  \end{tabular}
\end{lstlisting}


\begin{tabular}{ccc}
  \hline
  \head{Command} & \head{Declaration} & \head{Output} \\
  \hline
  \verb|\textrm| & \verb|\rmfamily| & \rmfamily Example text \\
  \verb|\textsf| & \verb|\sffamily| & \sffamily Example text \\
  \verb|\texttt| & \verb|\ttfamily| & \ttfamily Example text \\
  \hline
\end{tabular}

Within tabular, three types of lines may be used:
\begin{itemize}
\item \lstinline|\hline|: draws a horizontal line over the whole width of the table
\item \lstinline|\cline{m-n}|: draws a horizontal line starting at the beginning of column \argument{m} and ending at the end of column \argument{n}
\item \lstinline|vline|: draws a vertical line over the full height and depth of the current row
\end{itemize}


The options understood by the \argument{tabular} environment are as follows:
\begin{itemize}
\item \argument{l}: for left alignment.
\item \argument{c}: for centered alignment.
\item \argument{r}: for right alignment.
\item \lstinline|p{width}|: for a "paragraph" cell of a certain width. . If you place several \argument{p} cells next to each other, they will be aligned at their top line. It's equivalent to using \lstinline|\parbox[t]{width}| within a cell.
\item \lstinline|@{code}| inserts \argument{code} instead of empty space before or after a column. This might also be some text or it could be left empty to avoid this space.
\item \argument{|}: stands for a vertical line.
\item \lstinline|*{n}{options}| is equivalent to \argument{n} copies of options, where n is a positive integer and options may consist of one or more column specifiers including * as well.
\end{itemize}





\section[Multiple Columns]{Spanning Entries Over Multiple Columns}
\label{sec:spann-entr-over}


\begin{lstlisting}
\begin{tabular}{@{}l*2{>{\textbackslash\ttfamily}l}l<{Example text}@{}}
  \toprule
  & \multicolumn{2}{c}{\head{Input}} & \multicolumn{1}{c}{\head{Output}}\\
  & \normal{\head{Command}} & \normal{\head{Declaration}} & \normal{}\\
  \cmidrule(lr){2-3}\cmidrule(l){4-4}
  Family & textrm&rmfamily & \rmfamily\\
  & textsf & sffamily & \sffamily\\
  & texttt & ttfamily & \ttfamily\\
  \bottomrule
\end{tabular}
  
\end{lstlisting}

\begin{tabular}{@{}l*2{>{\textbackslash\ttfamily}l}l<{Example text}@{}}
  \toprule
  & \multicolumn{2}{c}{\head{Input}} & \multicolumn{1}{c}{\head{Output}}\\
  & \normal{\head{Command}} & \normal{\head{Declaration}} & \normal{}\\
  \cmidrule(lr){2-3}\cmidrule(l){4-4}
  Family & textrm&rmfamily & \rmfamily\\
  & textsf & sffamily & \sffamily\\
  & texttt & ttfamily & \ttfamily\\
  \bottomrule
\end{tabular}


\section{Adding Captions to Tables}
\label{sec:adding-capt-tabl}


\begin{lstlisting}
\begin{table}
  \centering
  \begin{tabular}{}
    
  \end{tabular}
  \caption{Caption}
  \label{tab:caption}
\end{table}  
\end{lstlisting}

