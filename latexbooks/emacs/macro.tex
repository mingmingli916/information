
\chapter{Macro}
\label{cha:macro}

In Emacs, a macro is simply a group of recorded keystrokes you can play back over and over again.
Macros are a great way to save yourself \keyword{repetitive} work.


\section{Macro Commands}
\label{sec:macro-commands}

\begin{table}[H]
  \centering
  \begin{tabular}{l>{\bfseries}ll}
    \toprule
    \head{Group} & \head{Binding} & \head{Meaning}\\
    \midrule
    \multirow{2}{*}{define a macro} & C-x ( & start macro definition\\
                 & C-x ) & end macro definition\\
    \midrule
    \multirow{2}{*}{execute macro} & C-x e & call the last defined macro\\
                 & C-x C-k r & apply last macro to all lines in region\\
    \midrule
    edit macro & C-x C-k e & edit macro\\
    \midrule
    \multirow{4}{*}{macro ring} & C-x C-k C-d & delete current macro from macro ring\\
                 & C-x C-k C-t & swap first two elements in macro ring\\
                 & C-x C-k C-p & move to previous macro in macro ring\\
                 & C-x C-k C-n & move to next macro in macro ring\\
    \midrule
    bind macro & C-x C-k b & bind macro\\
    \midrule
    name macro & C-x C-k n & name macro\\
    \bottomrule
  \end{tabular}
  \caption{Macro Commands}
  \label{tab:macro-commands}
\end{table}



\section[Saving Macros]{Naming, Saving, and Executing Your Macros}

\begin{enumerate}
\item Define a macro.
\item name it with \keyword{C-x C-k n}.
\item Open a file.
\item \keyword{M-x insert-kbd-macro RET <macroname> RET}.
\item add \lstinline[language=elisp]|(load-file "<your macro file>")| to \argument{.emacs}.
\item add \lstinline[language=elisp]|(global-set-key "\C-x\C-k<your key>" '<your macro name>)| to \argument{.emacs}.
\end{enumerate}




\section[Macro with Input]{Pausing a Macro for Keyboard Input}
\label{sec:paus-macro-keyb}

When you’re defining a macro, type \keyword{C-u C-x q} at the point where you want the recursive edit to occur.
Emacs enters a recursive edit.
You can tell you’re in a recursive edit because square brackets appear on the mode line.
Nothing you type during the recursive edit becomes a part of the macro.
You can type whatever you want to and then press \keyword{C-M-c} to exit the recursive edit. 



\section{Adding a Query to a Macro}
\label{sec:adding-query-macro}

When you’re defining a macro, type \keyword{C-x q} at the point where you want to add a query.
Nothing happens immediately; go on defining the macro as you normally would.
When you execute the macro and it gets to the point in the macro where you typed \keyword{C-x q}, Emacs prints a query in the minibuffer:
\begin{verbatim}
Proceed with macro? (y, n, RET, C-l, C-r)
\end{verbatim}

Here's the meaning of the options:
\begin{itemize}
\item \keyword{y} means to continue and go on to the next repetition, if any.
\item \keyword{n} means to stop executing the macro but go on to the next repetition, if any.
\item \keyword{Enter} means to stop executing the macro and cancel any repetitions.
\item \keyword{C-r} C-r starts a recursive edit. To exit a recursive edit, press \keyword{C-M-c}.
\item \keyword{C-l} puts the line the cursor in on in the middle of the screen.
\end{itemize}





%%% Local Variables:
%%% mode: latex
%%% TeX-master: "emacs"
%%% End:
