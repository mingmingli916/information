
\chapter[Working Environment]{Emacs as Working Environment}
\label{cha:emacs-as-working}

\section{Spell Checking}

Emacs provide Ispell interface.
We say ``interfaces'' because Emacs does not include the executable.


On MacOSX, There are two popular software manager \keyword{MacPorts} and \keyword{Homebrew}. You can use any of it to install Ispell.
After installation, use command \lstinline[language=sh]|which ispell| in terminal to locate the Ispell executable file.
In Emacs, type \keyword{C-h a ispell} to get all functions containing the keyword ispell.
You can find the function \argument{ispell-check-version}.
Type \keyword{M-x ispell-check-version} to check whether Ispell is working correctly.
It will show you where should the Ispell executable should be located.
Link the Ispell executable to where Emacs expects Ispell to be:
\begin{lstlisting}[language=sh]
ln -s /opt/local/bin/ispell /usr/local/bin/ispell
\end{lstlisting}



\begin{table}[H]
  \centering
  \begin{tabular}{>{\bfseries}l>{\ttfamily}ll}
    \toprule
    \head{Binding} & \head{Command} & \head{Meaning}\\
    \midrule
                   & ispell-buffer & \\
                   & ispell-region & \\
    M-\$ & ispell-word & \\
    \midrule
                   & ispell-complete-word & \\
    \bottomrule
  \end{tabular}
  \caption{Ispell Commands}
  \label{tab:ispell-commands}
\end{table}

You can use \keyword{C-h f ispell-help} for the options available when a misspelling is encountered.
The options are the following:
\newpage{}

\begin{table}[H]
  \centering
  \begin{tabular}{>{\bfseries}lp{0.6\textwidth{}}}
    \toprule
    \head{Option} & \head{Meaning}\\
    \midrule
    DIGIT & replace the word with a digit offered in the *Choices* buffer\\
    SPC &   accept word this time\\
    i & accept word and insert into private dictionary\\
    a & accept word for this session\\
    A & accept word and place in ‘buffer-local dictionary’\\
    r & replace word with typed-in value.  Rechecked\\
    R & replace word with typed-in value.  Query-replaced in buffer.  Rechecked\\
    x & exit spelling buffer.  Move cursor to original point\\
    X & exit spelling buffer.  Leaves cursor at the current point, and permits the aborted check to be completed later\\
    q & quit spelling session\\
    l & look up typed-in replacement in alternate dictionary\\
    u & like ‘i’, but the word is lower-cased first\\
    m & place typed-in value in personal dictionary, then recheck current word\\
    C-r &recursive edit\\
    \bottomrule
  \end{tabular}
  \caption{Ispell options}
  \label{tab:ispell-options}
\end{table}

\section{Flyspell}
\label{sec:flyspell-1}

Flyspell highlights misspelled words as you type.
There are two mode related with Flyspell: \keyword{flyspell-mode} and \keyword{flyspell-prog-mode}.
The latter mode is designed for programmers.
In this mode Emacs highlights misspellings only in comments or strings.
To check existing text, you run \keyword{M-x flyspell-buffer Enter}.


Flyspell highlights misspelled words in red.
Words that are repeatedly misspelled are highlighted in yellow.


Here are some useful commands in \keyword{flyspell-mode}:
\begin{table}[H]
  \centering
  \begin{tabular}{>{\bfseries}ll}
    \toprule
    \head{Binding} & \head{Meaning}\\
    \midrule
    M-\$ & correct words (using Ispell)\\
    C-M-i & automatically correct word\\
    C-; & automatically correct the last misspelled word\\
    \bottomrule
  \end{tabular}
  \caption{Flyspell commands}
  \label{tab:flyspell-commands}
\end{table}

You can type \keyword{C-M-i} or \keyword{C-;} again to change to another  correct word if the corrected word is not what you want.

\section{Bookmarks}
\label{sec:bookmarks}


It saves any new bookmarks in this file automatically when you exit Emacs.
Bookmark points to a position in a file and not to a piece of text.


\subsection{Bookmark Commands}
\label{sec:bookmark-commands}

\begin{table}[H]
  \centering
  \begin{tabular}{>{\bfseries}ll}
    \toprule
    \head{Binding} & \head{Meaning}\\
    \midrule
    C-x r m & set a bookmark\\
    C-x r b & move to a bookmark\\
    \midrule
    C-x r l & \argument{*Bookmark List*} buffer appears\\
    \bottomrule
  \end{tabular}
  \caption{Bookmark Command}
  \label{tab:bookmark-commands}
\end{table}


\subsection{Bookmark List}
\label{sec:bookmark-list}

\begin{table}[H]
  \centering
  \begin{tabular}{>{\bfseries}ll}
    \toprule
    \head{Binding} & \head{Meaning}\\
    \midrule
    RET, or j & open bookmark\\
    o & open the bookmark in another window and move the cursor to that window\\
    C-o & open the bookmark in another window and keep the cursor in current window\\
    \midrule
    r & rename bookmark\\
    s & save all bookmark\\
    \midrule
    m & mark bookmarks to be displayed in multiple windows\\
    d & mark bookmark for deletion\\
    u & remove mark\\
    \midrule
    x & delete bookmarks marked for deletion\\
    v & display marked bookmarks or the one the cursor is on if none are marked\\
    \midrule
    e & edit (or create) annotation for the current bookmark\\
    a & display annotation for current bookmark\\
    A & display all annotations\\
    \midrule
    q & exit bookmark list\\
    \bottomrule
  \end{tabular}
  \caption{Bookmark list commands}
  \label{tab:bookmark-list-commands}
\end{table}

\section{Shell}
\label{sec:shell}


\subsection{One Command at a Time}
\label{sec:one-command-at}

\begin{table}[H]
  \centering
  \begin{tabular}{>{\bfseries}ll}
    \toprule
    \head{Binding} & \head{Meaning}\\
    \midrule
    M-! & execute command \\
    M-| & execute command with region as input\\
    M-\& & execute command asynchronously in background\\
    \bottomrule
  \end{tabular}
  \caption{One command at a time}
  \label{tab:one-command-at-a-time}
\end{table}

With \keyword{C-u} prefix, the command output is insert to current buffer.

\subsection{Shell Mode}
\label{sec:shell-mode}

To start a shell buffer, type \keyword{M-x shell Enter}.
This creates a buffer named \argument{*shell*}.



How does Emacs know which shell to start?
First, it looks at the variable \argument{shell-file-name}.
Then it looks for a Unix environment variable named \argument{ESHELL}.
Finally it looks for an environment variable named \argument{SHELL}.
If you want to run another particular shell (for example, the zed shell) when you’re in Emacs, you can add the following command to your \argument{.emacs} file:
\begin{lstlisting}
(setq shell-file-name "/bin/zsh")
\end{lstlisting}


When Emacs starts an interactive shell, it runs an additional initialization file after your shell’s normal startup files.
The name of this file is \argument{.emacs\_shell-name}, where \argument{shell-name} is the name of the shell you want to use in Emacs.
It must be located in your home directory.
For example, if you use zsh, you can add Emacs-only startup commands by placing them in the file \argument{.emacs\_zsh}

\begin{table}[H]
  \centering
  \begin{tabular}{l>{\bfseries}ll}
    \toprule
    \head{Group} & \head{Binding} & \head{Meaning}\\
    \midrule
    \multirow{6}{*}{terminal} & C-c C-c & like \keyword{C-c} in terminal\\
                 & C-c C-u & line \keyword{C-u} in terminal\\
                 & M-r & like \keyword{C-r} in terminal\\
                 & M-p & like \keyword{C-p} in terminal\\
                 & M-n & like \keyword{C-n} in terminal\\
                 & C-c SPC & like \keyword{\textbackslash{}} in terminal\\
    \midrule
    \multirow{2}{*}{delete} & C-c C-o & delete output from last command\\
                 & C-c M-o & clear the shell buffer\\
    \midrule
    \multirow{2}{*}{window} & C-c C-r & move first line of output to top of window\\
                 & C-c C-e & move last line of output to bottom of window\\
    \midrule
    \multirow{3}{*}{history} & C-c C-p & move to previous command\\
                 & C-c C-n & move to next command\\
                 & C-c C-l & list the commands you have typed\\
    \midrule
    argument & C-c . & insert previous command argument\\
    \midrule
    \multirow{2}{*}{move} & C-c C-f & \argument{shell-forward-command}\\
                 & C-c C-b & \argument{shell-backward-command}\\
    \midrule
    save & C-c C-s & write the privous output into a file (overwrite)\\
    \bottomrule
  \end{tabular}
  \caption{Shell mode commands}
  \label{tab:shell-mode-commands}
\end{table}


To create multiple shell buffer, you can use \keyword{C-u M-x shell}.


\subsection{Eshell, Shell and Term}
\label{sec:eshell}

Here's the differences:
\begin{itemize}
\item \keyword{M-x shell} starts the standard emacs interface to Operating System's command line interface.

\item \keyword{M-x term} starts the terminal emulator in emacs.
It behaves like a dedicated terminal app, such as xterm, gnome-terminal, puTTY.
It is compatible to more shell apps than emacs shell interface, but standard emacs keys such as moving cursor doesn't work here.

\item \keyword{M-x eshell} starts eshell which is a shell written entirely in emacs lisp.
Note: it is not a bash emulator.
Eshell is a shell by itself, but similar to bash or other shells.

\end{itemize}

Which should you use?
It depends on your preference and needs.
The following is some general guide.

\begin{itemize}
\item shell is the most popular. It is good for general use of classic/standard unix shell commands.
\item term are good if you want to run stuff like \argument{ssh}, or other command line interactive interface (such as \argument{python}), or text based GUI app such as \argument{vim}.
\item Eshell is super fast on startup. If you are a emacs lisp programer, you might prefer eshell because direct access to emacs lisp and better integration with emacs.
\end{itemize}




\section{Directory Editor}
\label{sec:directory-editor}

There are several ways to start directory editing.
If you're not in Emacs, invoke Emacs with a directory name as an argument.
For example, if you want to edit the directory \argument{notebook}, type the following
\begin{lstlisting}[language=sh]
emacs notebook
\end{lstlisting}
If you are in Emacs, you can use \keyword{C-x C-f DIRECTORY} or \keyword{C-x d DIRECTORY} to edit the directory.

\begin{center}
  \begin{longtable}[H]{l>{\bfseries}lp{0.6\textwidth}}
    \toprule
    \head{Group} & \head{Binding} & \head{Meaning}\\
    \midrule
    \endfirsthead

    \toprule
    \head{Group} & \head{Binding} & \head{Meaning}\\
    \midrule
    \endhead

    \midrule
    \multicolumn{3}{c}{{Continued on next page}}\\
    \bottomrule
    \endfoot

    \endlastfoot

    
    \multirow{12}{*}{mark} & m & mark current file\\
                 & * * & mark executable files\\
                 & * @ & mark symlinks\\
                 & * / & mark directories\\
                 & \% m & mark files matching \argument{REGEXP}\\
                 & \% g & mark files containing \argument{REGEXP}\\
                 & \% d & flag for deletion files that match regular expression\\
                 & d & mark file for deletion\\
                 & \# & mark autosave files for deletion\\
                 & \textasciitilde{} & mark backup files for deletion, \keyword{C-u} to prefix remove the flag\\
                 & \& & mark garbage files for deletion\\
                 & u & unmark current file\\
                 & U & unmark all files\\
                 & t & toggle mark\\
                 & * c & change marks on specified files\\
    \midrule
    \multirow{4}{*}{visit} & v & view file read only, \keyword{q} to quit\\
                 & o & find file in another window; move there\\
                 & C-o & find file in another window; don't move there\\
                 & RET & visit file\\
    \midrule
    \multirow{17}{*}{operation} & O & change ownership of file\\
                 & D & delete a file immediately\\
                 & R & rename\\
                 & C & copy marked files\\
                 & G & change group permissions\\
                 & L & load the marked Emacs Lisp files\\
                 & Z & compress or uncompress\\
                 & M & use \keyword{chmod} on current file\\
                 & S & create a symbolic link to this file\\
                 & s & sort the Dired display by date or by filename (toggle)\\
                 & w & copy filename into the kill ring\\
                 & y & display information on the type of the file using the \keyword{file} command\\
                 & ! & run shell command\\
                 & + & create a directory\\
                 & Q & replace matches in all marked files\\
                 & A & do a regular expression search on marked files\\
                 & x & delete the files flagged for deletion\\
    \midrule
                 & n & next line\\
                 & p & previous line\\       
                 & \textasciicircum{} & up directory\\
                 & > & next directory line\\
    \multirow{8}{*}{navigation} & < & previous directory line\\
                 & i & insert this subdirectory into the same dired buffer\\
                 & \$ & hide or show the current directory or subdirectory\\
                 & M-\$ & hide all subdirectories, leaving only their names; repeat command to show\\
                 & C-M-n & move to next subdirectory (if you've inserted subdirectories using \keyword{i})\\
                 & C-M-p & move to previous subdirectory (if you’ve inserted subdirectories using \keyword{i})\\
                 & M-\} & next marked file\\
                 & M-\{ & previous marked file\\
    \bottomrule
    \caption{Dired commands}
    \label{tab:dired-commands}
  \end{longtable}
\end{center}



Here's one trick.
In Dired mode, you can use \keyword{C-x C-q} to change the readonly status.
After is becomes editable, you can edit the dired buffer directly.
When you finish the edit, type \keyword{C-c C-c} to make the change work.

\section{Calendar}
\label{sec:calendar}

To display the calendar, type \keyword{M-x calendar}.
Emacs displays a calendar window with three months: last month, this month, and next month.
Here's my simple configuration in \argument{.emacs} file:
\begin{lstlisting}
;; set the first day of a week to Monday
(setq calendar-week-start-day 1)
;; start calendar at start emacs
(calendar)
\end{lstlisting}

\newpage{}

\begin{table}[H]
  \centering
  \begin{tabular}{l>{\bfseries}ll}
    \toprule
    \head{Group} & \head{Binding} & \head{Meaning}\\
    \midrule
    \multirow{9}{*}{relative move}  & . & today\\
                 & C-f & next day\\
                 & C-b & previous day\\
                 & C-n & forward by week\\
                 & C-p & backward by week\\
                 & M-\} & forward by month\\
                 & M-\{ & backward by month\\
                 & C-x [ & forward by year\\
                 & C-x ] & backward by year\\
    \midrule
    \multirow{7}{*}{absolute move} & C-a & beginning of the week\\
                 & C-e & end of the week\\
                 & M-a & beginning of the month\\
                 & M-e & end of the month\\
                 & M-< & beginning of the year\\
                 & M-> & end of the year\\
                 & g d & go to the specified day\\
    \midrule
    \multirow{4}{*}{scroll} & C-x > & scroll forward by 1 month\\
                 & C-x < & scroll backward by 1 month\\
                 & C-v & scroll forward by 3 months\\
                 & M-v & scroll backward by 3 months\\
    \midrule
    \multirow{4}{*}{holiday}  & a & show holidays for the current calendar window\\
                 & h & show whether today or the specified day is a holiday\\
                 & x & highlight holidays\\
                 & u & remove the highlights\\
    \midrule
    \multirow{7}{*}{add entry} & i d & add day entry\\
                 & i w & add weekly entry\\
                 & i m & add month entry\\
                 & i y & add annual entry\\
                 & i a & add annual entry (the year is included for reference)\\
                 & i c & add cyclic entry\\
                 & i b & add block entry\\
    \midrule
    \multirow{3}{*}{display} & m & highlight entries\\
                 & d & display entry for the current date\\
                 & s & display all entries\\
    \bottomrule
  \end{tabular}
  \caption{Calendar Commands}
  \label{tab:calendar-commands}
\end{table}








%%% Local Variables:
%%% mode: latex
%%% TeX-master: "emacs"
%%% End:
