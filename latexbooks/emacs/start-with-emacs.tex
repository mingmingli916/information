
\chapter{Start With Emacs}
\label{cha:start-with-emacs}

\section{Start Emacs}
\label{sec:start-emacs}

There are two ways to start Emacs:
\begin{itemize}
\item Click the Emacs icon \parbox{2em}{\includegraphics[width=2em]{logo}}.
\item Use the command \argument{emacs} in your terminal.
\end{itemize}


\begin{lstlisting}[language=sh]
# To get emacs usage.
emacs --help

# To simply start the Emacs
emacs
\end{lstlisting}


\section{Exit Emacs}
\label{sec:exit-emacs}

In Emacs, use \keyword{C-x C-c} (x: execute; c: clear) to exit.


\section{Working With Files}
\label{sec:working-with-files}

\begin{table}[H]
  \centering
  \begin{tabular}{>{\bfseries}lp{0.5\textwidth}}
    \toprule
    Binding & \head{Meaning}\\
    \midrule
    C-x C-f (f: file)& Open or create a file.\\
    C-x C-v & Read an alternate file, replacing the one read with \keyword{C-x C-f}.\\
    C-x i (i: insert) & Insert file at cursor position.\\
    C-x C-s (s: save) & Save buffer.\\
    C-x C-w (w: write) & Save buffer as another file.\\
    \bottomrule
  \end{tabular}
  \caption{Working with files}
  \label{tab:working-with-files}
\end{table}
%%% Local Variables:
%%% mode: latex
%%% TeX-master: "emacs"
%%% End:
