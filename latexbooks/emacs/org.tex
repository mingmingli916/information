
\chapter{Org}
\label{cha:org}


Org is a mode for keeping notes, maintaining TODO lists, and project planning with a fast and effective plain-text markup language.
It also is an authoring system with unique support for literate programming and reproducible research.
Org is implemented on top of Outline mode.

Files with the '.org' extension use Org mode by default.





\section{Document Structure}
\label{sec:document-structure}


\subsection{Headlines}
\label{sec:headlines-1}


Headlines define the structure of an outline tree.
Org headlines start on the left margin with one or more stars followed by a space.

\begin{tcolorbox}
\begin{verbatim}
* Top level headline
** Second level
*** Third level
    some text
*** Third level
    more text
* Another top level headline
\end{verbatim}
\end{tcolorbox}

\subsection{Motion}
\label{sec:motion}


\begin{table}[H]
  \centering
  \begin{tabular}{>{\bfseries}ll}
    \toprule
    \head{Binding} & \head{Meaning}\\
    \midrule
    C-c C-n & Next heading\\
    C-c C-p & Previous heading\\
    C-c C-f & Next heading same level\\
    C-c C-b & Previous heading same level\\
    C-c C-u & Backward to higher level heading.\\
    \bottomrule
  \end{tabular}
  \caption{Motion commands}
  \label{tab:org-motion-cmds}
\end{table}



\subsection{Visibility Cycling}
\label{sec:visibility-cycling}

Outlines make it possible to hide parts of the text in the buffer.
Org uses just two commands, bound to \keyword{TAB} and \keyword{S-TAB} to change the visibility in the buffer.




\begin{table}[H]
  \centering
  \begin{tabular}{>{\bfseries}lp{0.6\textwidth}}
    \toprule
    \head{Binding} & \head{Meaning}\\
    \midrule
    TAB & Rotate current subtree among ``fold - children - subtree''. Point must be on a headline for this to work.\\
    S-TAB & Rotate the entire buffer among ``overview - content - show all''\\
    C-c C-r & Reveal context around point, showing the current entry, the following heading and the hierarchy above.\\
    C-c C-k & Expose all the headings of the subtree, but not their bodies.\\
    C-c TAB & Expose all direct children of the subtree. With a numeric prefix argument N, expose all children down to level N.\\
    C-c C-x b & Show the current subtree in an indirect buffer\\
    \bottomrule
  \end{tabular}
  \caption{Visibility cycling commands}
  \label{tab:visibility-cycling-cmds}
\end{table}

\newpage{}
\subsection{Structure Editing}
\label{sec:structure-editing}

\begin{table}[H]
  \centering
  \begin{tabular}{>{\bfseries}lp{0.6\textwidth}}
    \toprule
    \head{Binding} & \head{Meaning}\\
    \midrule
    M-RET & Insert a new heading, item or row.\\
    C-RET & Insert a new heading at the end of the current subtree.\\
    M-S-RET & Insert new TODO entry with same level as current heading.\\
    TAB & In a new entry with no text yet, the first TAB demotes the entry to become a child of the previous one. The next TAB makes it a parent, and so on, all the way to top level. Yet another TAB, and you are back to the initial level.\\
    M-\(\leftarrow\) & Promote current heading by one level.\\
    M-\(\rightarrow\) & Demote current heading by one level\\
    M-S-\(\leftarrow\) & Promote the current subtree by one level.\\
    M-S-\(\rightarrow\) & Demote the current subtree by one level.\\
    M-\(\uparrow\) & Move subtree up, i.e., swap with previous subtree of same level.\\
    M-\(\downarrow\) & Move subtree down, i.e., swap with next subtree of same level.\\
    C-c @ & Mark the subtree at point. Hitting repeatedly marks subsequent subtrees of the same level as the marked subtree.\\
    C-c C-x C-w & Kill subtree, i.e., remove it from buffer but save in kill ring. With a numeric prefix argument N, kill N sequential subtrees.\\
    C-c C-x M-w & Copy subtree to kill ring. With a numeric prefix argument N, copy the N sequential subtrees.\\
    C-c C-x C-y & Yank subtree from kill ring. This does modify the level of the subtree to make sure the tree fits in nicely at the yank position. \\
    C-c C-w & Move the entry or entries at point to another heading.\\
    C-c \textasciicircum{} & Sort same-level entries. When there is an active region, all entries in the region are sorted. Otherwise the children of the current headline are sorted. \\
    C-x n s & Narrow buffer to current subtree.\\
    C-x n b & Narrow buffer to current block.\\
    C-x n w & Widen buffer to remove narrowing.\\
    C-c * & Turn a normal line or plain list item into a headline—so that it becomes a subheading at its location.\\
    \bottomrule
  \end{tabular}
  \caption{Structure editing commands}
  \label{tab:structure-editing-cmds}
\end{table}



\subsection{Sparse Tree}
\label{sec:sparse-tree}


An important feature of Org mode is the ability to construct \keyword{sparse trees} for selected information in an outline tree, so that the entire document is folded as much as possible, but the selected information is made visible along with the headline structure above it.

\begin{table}[H]
  \centering
  \begin{tabular}{>{\bfseries}ll}
    \toprule
    \head{Binding} & \head{Meaning}\\
    \midrule
    C-c / & This prompts for an extra key to select a sparse-tree creating command.\\
    M-g n or M-g M-n & Jump to the next sparse tree match in this buffer.\\
    M-g p or M-g M-p & Jump to the previous sparse tree match in this buffer.\\
    \bottomrule
  \end{tabular}
  \caption{Sparse tree commands }
  \label{tab:sparse-tree-cmds}
\end{table}

\subsection{Drawers}
\label{sec:drawers}

Sometimes you want to keep information associated with an entry, but you normally do not want to see it.
For this, Org mode has \keyword{drawers}.
They can contain anything but a headline and another drawer.
Drawers look like this:

\begin{tcolorbox}
\begin{verbatim}
** This is a headline
Still outside the drawer
:DRAWERNAME:
This is inside the drawer.
:END:
After the drawer.
\end{verbatim}
\end{tcolorbox}


You can interactively insert a drawer at point by typing \keyword{C-c C-x d}.
With an active region, this command puts the region inside the drawer.
With a prefix argument, this command creates a 'PROPERTIES' drawer right below the current headline.
Org mode uses this special drawer for storing properties.
You cannot use it for anything else.


Visibility cycling on the headline hides and shows the entry, but keep the drawer collapsed to a single line.
In order to look inside the drawer, you need to move point to the drawer line and press \keyword{TAB} there.

\subsection{Block}
\label{sec:block}

Org mode uses \keyword{\#+BEGIN … \#+END} blocks for various purposes from including source code examples to capturing time logging information.
These blocks can be folded and unfolded by pressing TAB in the \keyword{\#+BEGIN} line. 

\section{Plain Lists}
\label{sec:plain-lists}

Org knows ordered lists, unordered lists, and description lists.
\begin{itemize}
\item Unordered list items start with \keyword{-, + } or \keyword{*} as bullets.
\item Ordered list items start with a numeral followed by either a period or a right parenthesis, such as \keyword{1.} or \keyword{1)}. If you want a list to start with a different value — e.g., 10 — start the text of the item with \keyword{[@10]}
\item Description list items are unordered list items, and contain the separator \keyword{::} to distinguish the description term from the description.
\end{itemize}

The following commands act on items when point is in the first line of an item—the line with the bullet or number.
Some of them imply the application of automatic rules to keep list structure intact.
If some of these actions get in your way, configure \argument{org-list-automatic-rules} to disable them individually.



\begin{table}[H]
  \centering
  \begin{tabular}{>{\bfseries}lp{0.6\textwidth}}
    \toprule
    \head{Binding} & \head{Meaning}\\
    \midrule
    TAB & Items can be folded  or unfolded\\
    M-RET & Insert new item at current level.\\
    M-S-RET & Insert a new item with a checkbox\\
    M-\(\uparrow\) or M-\(\downarrow\) & Move the item including subitems up/down.\\
    M-\(\leftarrow\) or M-\(\rightarrow\) & Decrease/increase the indentation of an item, leaving children alone.\\
    M-S-\(\leftarrow\) or M-S-\(\rightarrow\) & Decrease/increase the indentation of the item, including subitems.\\
    C-c C-c & If there is a checkbox in the item line, toggle the state of the checkbox.\\
    C-c \# & Update the statistic cookie in the current outline entry.\\    
    C-c - &  Cycle the entire list level through the different itemize/enumerate bullets .\\
    C-c * & Turn a plain list item into a headline.\\
    C-c C-* & Turn the whole plain list into a subtree of the current heading.\\
    C-c \textasciicircum{} & Sort the plain list.\\
    \bottomrule
  \end{tabular}
  \caption{Plain list commands}
  \label{tab:plain-list-cmds}
\end{table}


\subsection{Checkboxes}
\label{sec:checkboxes}

Every item in a plain list can be made into a checkbox by starting it with the string \keyword{[ ]}.
This feature is similar to TODO items, but is more lightweight.
Checkboxes are not included into the global TODO list, so they are often great to split a task into a number of simple steps.


\begin{verbatim}

* light task [25%]
  - [-] task 1 [33%]
    - [X] task 1-1
    - [ ] task 1-2
    - [ ] task 1-3
  - [X] task 2
  - [ ] task 3
  - [ ] task 4

\end{verbatim}

\section{Hyperlinks}
\label{sec:hyperlinks}

\subsection{Link Format}
\label{sec:link-format}


The general link format looks like this:
\begin{tcolorbox}
\begin{verbatim}
[[LINK][DESCRIPTION]]
or alternatively
[[LINK]]
\end{verbatim}
\end{tcolorbox}

Once a link in the buffer is complete, with all brackets present, Org changes the display so that \keyword{DESCRIPTION} is displayed instead of \keyword{[[LINK][DESCRIPTION]]} and \keyword{LINK} is displayed instead of \keyword{[[LINK]]}.
You can directly edit the visible part of a link.
This can be either the LINK part, if there is no description, or the DESCRIPTION part otherwise.
To also edit the invisible LINK part, use \keyword{C-c C-l} with point on the link.


\subsection{Internal Links}
\label{sec:internal-links}

A link that does not look like a URL—i.e., does not start with a known scheme or a file name—refers to the current document.
You can follow it with \keyword{C-c C-o} when point is on the link.

Org provides several refinements to internal navigation within a document.
Most notably, a construct like \keyword{\lstinline|[[\#my-custom-id]]|} specifically targets the entry with the \keyword{CUSTOM\_ID} property set to \keyword{my-custom-id}.
Also, an internal link looking like \keyword{\lstinline|[[*Some section]]|} points to a headline with the name \keyword{Some section}.


When the link does not belong to any of the cases above, Org looks for a dedicated target: the same string in double angular brackets, like \keyword{\textless{}\textless{}My Target\textgreater{}\textgreater{}}.

If no dedicated target exists, the link tries to match the exact name of an element within the buffer.


Following a link pushes a mark onto Org’s own mark ring.
You can return to the previous position with \textgreater{C-c \&}.
Using this command several times in direct succession goes back to positions recorded earlier.


\subsection{External Links}
\label{sec:external-links}

External links are URL-like locators.
They start with a short identifying string followed by a colon.
There can be no space after the colon.


Here is the part set of built-in link types:
\begin{itemize}[itemsep=10pt]
\item file\\
  File links. File name may be remote, absolute, or relative.
  Additionally, you can specify a line number, or a text search.
  In Org files, you may link to a headline name, a custom ID, or a code reference instead.
  \verb|file:/home/li/notebook/emacs/emacs.tex|.
\item attachment\\
  Same as file links but for files and folders attached to the current node.
  Attachment links are intended to behave exactly as file links but for files relative to the attachment directory.
  \verb|attachment:projects.org|.
\item docview\\
  Link to a document opened with DocView mode.
  You may specify a page number.\\
  \verb|docview:papers/last.pdf|.
\item doi\\
  Link to an electronic resource, through its handle.
  \verb|doi:10.1000/182|.
\item elisp\\
  Execute an Elisp command upon activation.
  \verb|elisp:(find-file "~/notebook")|.
\item http\\
  \verb|http://www.google.com|.
\item https\\
  \verb|https://www.google.com|.
\item mailto\\
  Link to message composition.
  \verb|mailto:mingmingli916@gmail.com|.
\item shell\\
  Execute a shell command upon activation.
  \verb|shell:date|.
\end{itemize}




\subsection{Handling Links}
\label{sec:handling-links}

\begin{lrbox}{\lstbox}
\begin{lstlisting}[language=elisp, basicstyle=\footnotesize]
(global-set-key (kbd "C-c l") #'org-store-link)
\end{lstlisting}
\end{lrbox}

Org provides methods to create a link in the correct syntax, to insert it into an Org file, and to follow the link.
The main function is \argument{org-store-link} (\keyword{C-c l})\footnote{\usebox{\lstbox}}.
It stores a link to the current location. The link is stored for later insertion into an Org buffer.
The kind of link that is created depends on the current buffer:
\begin{itemize}
\item Org mode buffers\\
  For Org files, if there is a \keyword{\textless{}\textless{}target\textgreater{}\textgreater{}} at point, the link points to the target.
  Otherwise it points to the current headline.
  If the headline has a \keyword{CUSTOM\_ID} property, store a link to this custom ID.
\item Other files\\
  For any other file, the link points to the file, with a search string pointing to the contents of the current line.
  If there is an active region, the selected words form the basis of the search string.
\item Agenda view\\
  The created link points to the entry referenced by the current line.
\end{itemize}

\subsection{Link Abbreviations}
\label{sec:link-abbreviations}

Long URL can be cumbersome to type, and often many similar links are needed in a document.
For this you can use link abbreviations.
An abbreviated link looks like this:
\begin{verbatim}
[[linkword:tag][description]]
\end{verbatim}
where the tag is optional.
Abbreviations are resolved according to the information in the variable \argument{org-link-abbrev-alist} that relates the linkwords to replacement text.
\begin{lstlisting}[language=elisp]
(setq org-link-abbrev-alist
      '(("google" . "https://www.google.com")))
\end{lstlisting}
If the replacement text contains the string \keyword{\%s}, it is replaced with the tag.
Using \keyword{\%(my-function)} passes the tag to a custom Lisp function, and replace it by the resulting string.


\subsection{Search Options in File Links}
\label{sec:search-options-file}

File links can contain additional information to make Emacs jump to a particular location in the file when following a link. This can be a line number or a search option after a double colon.

\begin{verbatim}
[[file:~/code/main.c::255]]
[[file:~/xx.org::My Target]]
[[file:~/xx.org::*My Target]]
[[file:~/xx.org::#my-custom-id]]
[[file:~/xx.org::/regexp/]]
[[attachment:main.c::255]]
\end{verbatim}

\subsection{Summary}
\label{sec:summary-1}


\begin{lrbox}{\lstbox}
\begin{lstlisting}[language=elisp, basicstyle=\footnotesize]
(with-eval-after-load 'org
  (define-key org-mode-map (kbd "M-n") #'org-next-link)
  (define-key org-mode-map (kbd "M-p") #'org-previous-link))
\end{lstlisting}
\end{lrbox}

\begin{table}[H]
  \centering
  \begin{tabular}{>{\bfseries}lp{0.6\textwidth}}
    \toprule
    \head{Binding} & \head{Meaning}\\
    \midrule
    C-c C-l & Insert a link. With a \keyword{C-u} prefix, prompts for a file to link to. When point is on an existing link, edit the link and description parts of the link.\\
    C-c C-o & Open the link when point is on the link.\\
    C-c \% & Push the current position onto the Org mark ring, to be able to return easily.\\
    C-c \& & Following a link pushes a mark onto Org’s own mark ring. You can return to the previous position with this command.\\
    M-n & Move forward to the next link in the buffer.\tablefootnote{This is achieved by:\par\usebox{\lstbox}}\\
    M-p & Move backward to the previous link in the buffer.\\
    \bottomrule
  \end{tabular}
  \caption{Hyperlinks command summary}
  \label{tab:hyperlinks-cmds}
\end{table}

\section{TODO items}
\label{sec:todo-items}

Org mode does not maintain \keyword{TODO} lists as separate documents1.
Instead, TODO items are an integral part of the notes file, because TODO items usually come up while taking notes!
With Org mode, simply mark any entry in a tree as being a TODO item.

\subsection{Basic TODO Functionality}
\label{sec:basic-todo-funct}

Any headline becomes a TODO item when it starts with the word \keyword{TODO}.


\begin{table}[H]
  \centering
  \begin{tabular}{>{\bfseries}ll}
    \toprule
    \head{Binding} & \head{Meaning}\\
    \midrule
    C-c C-t & Roate the TODO state of the current item.\\
    S-\(\rightarrow\) S-\(\leftarrow\) & Select the following/preceding TODO state.\\
    S-M-RET & Insert a new TODO entry below the current one.\\
    \bottomrule
  \end{tabular}
  \caption{Basic TODO commands}
  \label{tab:basic-todo-cmds}
\end{table}

\subsection{Extended Use of TODO Keywords}
\label{sec:extended-use-todo}

By default, marked TODO entries have one of only two states: TODO and DONE.
Org mode allows you to classify TODO items in more complex ways with TODO keywords (stored in \keyword{org-todo-keywords}).


The structure of Org files makes it easy to define TODO dependencies.
Usually, a parent TODO task should not be marked as done until all TODO subtasks, or children tasks, are marked as done.
Sometimes there is a logical sequence to (sub)tasks, so that one subtask cannot be acted upon before all siblings above it have been marked as done.
If you customize the variable \argument{org-enforce-todo-dependencies}, Org blocks entries from changing state to DONE while they have TODO children that are not DONE. Furthermore, if an entry has a property \keyword{ORDERED}, each of its TODO children is blocked until all earlier siblings are marked as done.

\begin{table}[H]
  \centering
  \begin{tabular}{>{\bfseries}lp{0.6\textwidth}}
    \toprule
    \head{Binding} & \head{Meaning}\\
    \midrule
    C-c C-x o & Toggle the 'ORDERED' property of the current entry.\\
    \bottomrule
  \end{tabular}
  \caption{Extended TODO commands}
  \label{tab:extended-todo-cmds}
\end{table}


\subsection{Progress Logging}
\label{sec:progress-logging}

\begin{table}[H]
  \centering
  \begin{tabular}{>{\bfseries}lp{0.6\textwidth}}
    \toprule
    \head{Binding} & \head{Meaning}\\
    \midrule
    C-u C-c C-t & Prompt for a note and record the time of the TODO state change\\
    \bottomrule
  \end{tabular}
  \caption{Progress logging commands}
  \label{tab:progress-logging-commands}
\end{table}


\subsection{Priorities}
\label{sec:priorities}

Prioritizing can be done by placing a \keyword{priority cookie} into the headline of a TODO item right after the TODO keyword, like this:
\begin{verbatim}
* TODO [#A] Summarize org mode
\end{verbatim}

By default, Org mode supports three priorities: ‘A’, ‘B’, and ‘C’.
‘A’ is the highest priority.
An entry without a cookie is treated as equivalent if it had priority ‘B’.
Priorities make a difference only for sorting in the agenda.
Outside the agenda, they have no inherent meaning to Org mode.


Priorities can be attached to any outline node; they do not need to be TODO items.

\begin{table}[H]
  \centering
  \begin{tabular}{>{\bfseries}lp{0.6\textwidth}}
    \toprule
    \head{Binding} & \head{Meaning}\\
    \midrule
    C-c , & Set the priority of the current headline. The command prompts for a priority character ‘A’, ‘B’ or ‘C’. When you press SPC instead, the priority cookie, if one is set, is removed from the headline.\\
    S-\(\uparrow\) & Increase the priority of the current headline.\\
    S-\(\downarrow\) & Decrease the priority of the current headline.\\
    \bottomrule
  \end{tabular}
  \caption{}
  \label{tab:}
\end{table}

\subsection{Breaking Down Tasks into Subtasks}
\label{sec:breaking-down-tasks}

It is often advisable to break down large tasks into smaller, manageable subtasks.
You can do this by creating an outline tree below a TODO item, with detailed subtasks on the tree.
To keep an overview of the fraction of subtasks that have already been marked as done, insert either \keyword{[/]} or \keyword{[\%]} anywhere in the headline.
These cookies are updated each time the TODO status of a child changes, or when pressing C-c C-c on the cookie.



\section{Dates and Times}
\label{sec:dates-times}


To assist project planning, TODO items can be labeled with a date and/or a time.
The specially formatted string carrying the date and time information is called a \keyword{timestamp} in Org mode.

\subsection{Timestamps}
\label{sec:timestamps}

A timestamp is a specification of a date in a special format.
A timestamp can appear anywhere in the headline or body of an Org tree entry.
Its presence causes entries to be shown on specific dates in the agenda.


There are the following timestamps:
\begin{itemize}
\item Plain timestamp\\
  A simple timestamp just assigns a date/time to an item.
\begin{verbatim}
<2022-11-07 Mon 14:00>
\end{verbatim}
\item Timestamp with repeater interval\\
  A timestamp may contain a \keyword{repeater interval}, indicating that it applies not only on the given date, but again and again after a certain interval of N days (d), weeks (w), months (m), or years (y).
\begin{verbatim}
<2022-11-07 Mon 7:00 +1d>
\end{verbatim}
\item Diary-style expression entries\\
  For more complex date specifications, Org mode supports using the special expression diary entries implemented in the Emacs Calendar package.
\begin{verbatim}
<%%(diary-float t 4 2)>
\end{verbatim}
\item Time/Data range\\
  Two timestamps connected by \keyword{-{}-} denote a range.
\begin{verbatim}
<2022-11-07 Mon>--<2022-11-08 Tue>
\end{verbatim}
\item Inactive timestamp\\
  Just like a plain timestamp, but with square brackets instead of angular ones.
  These timestamps are inactive in the sense that they do not trigger an entry to show up in the agenda.
\begin{verbatim}
[2022-11-07 Mon]
\end{verbatim}
\end{itemize}


\subsection{Creating Timestamps}
\label{sec:creating-timestamps}

For Org mode to recognize timestamps, they need to be in the specific format. All commands listed in Table \ref{tab:creating-timestamp-cmds} produce timestamps in the correct format.

\begin{table}[H]
  \centering
  \begin{tabular}{>{\bfseries}lp{0.6\textwidth{}}}
    \toprule
    \head{Binding} & \head{Meaning}\\
    \midrule
    C-c . & Prompt for a date and insert a corresponding timestamp. When point is at an existing timestamp in the buffer, the command is used to modify this timestamp. When this command is used twice in succession, a time range is inserted.\\
    C-u C-c . & Use the alternative format which contains date and time.\\
    C-u C-u C-c . & Insert an active timestamp with the current time without prompting.\\
    C-c ! & Like C-c ., but insert an inactive timestamp that does not cause an agenda entry.\\
    C-c C-c & Normalize timestamp, insert or fix day name if missing or wrong.\\
    C-c < & Insert a timestamp corresponding to point date in the calendar.\\
    C-c > & Access the Emacs calendar for the current date. If there is a timestamp in the current line, go to the corresponding date instead.\\
    C-c C-o & Access the agenda for the date given by the timestamp or -range at point.\\
    S-\(\leftarrow\) or S-\(\rightarrow\) & Change date at point by one day\\
    S-\(\uparrow\) or S-\(\downarrow\) & On the beginning or enclosing bracket of a timestamp, change its type. Within a timestamp, change the item under point. Point can be on a year, month, day, hour or minute. When the timestamp contains a time range like ‘15:30-16:30’, modifying the first time also shifts the second, shifting the time block with constant length. To change the length, modify the second time. \\
    C-c C-y & Evaluate a time range by computing the difference between start and end. With a prefix argument, insert result after the time range.\\ 

    \bottomrule
  \end{tabular}
  \caption{Creating timestamp commands}
  \label{tab:creating-timestamp-cmds}
\end{table}

When Org mode prompts for a date/time, the default is shown in default date/time format, and the prompt therefore seems to ask for a specific format.
But it in fact accepts date/time information in a variety of formats.
Generally, the information should start at the beginning of the string.
Org mode finds whatever information is in there and derives anything you have not specified from the \keyword{default date and time}.
The default is usually the current date and time, but when modifying an existing timestamp, or when entering the second stamp of a range, it is taken from the stamp in the buffer.
When filling in information, Org mode assumes that most of the time you want to enter a date in the future: if you omit the month/year and the given day/month is before today, it assumes that you mean a future date.


For example, let’s assume that today is June 13, 2006.
\begin{verbatim}
‘3-2-5’                =>  2003-02-05                            
‘2/5/3’                =>  2003-02-05                            
‘14’                   =>  2006-06-14                            
‘12’                   =>  2006-07-12                            
‘2/5’                  =>  2007-02-05                            
‘Fri’                  =>  nearest Friday (default date or later)
‘sep 15’               =>  2006-09-15                            
‘feb 15’               =>  2007-02-15                            
‘sep 12 9’             =>  2009-09-12                             
‘12:45’                =>  2006-06-13 12:45                       
‘22 sept 0:34’         =>  2006-09-22 0:34                        
‘w4’                   =>  ISO week for of the current year 2006  
‘2012 w4 fri’          =>  Friday of ISO week 4 in 2012           
‘2012-w04-5’           =>  Same as above                          
\end{verbatim}



Furthermore you can specify a relative date by giving, as the first thing in the input: a plus/minus sign, a number and a letter—‘h’, ‘d’, ‘w’, ‘m’ or ‘y’—to indicate a change in hours, days, weeks, months, or years.
With ‘h’ the date is relative to the current time, with the other letters and a single plus or minus, the date is relative to today at 00:00.
With a double plus or minus, it is relative to the default date.
If instead of a single letter, you use the abbreviation of day name, the date is the Nth such day.

\begin{verbatim}
‘+0’         =>  today
‘.’          =>  today
‘+2h’        =>  two hours from now
‘+4d’        =>  four days from today
‘+4’         =>  same as +4d
‘+2w’        =>  two weeks from today
‘++5’        =>  five days from default date
‘+2tue’      =>  second Tuesday from now
\end{verbatim}

Parallel to the minibuffer prompt, a calendar is popped up.
You can control the calendar fully from the minibuffer:
\begin{table}[H]
  \centering
  \begin{tabular}{>{\bfseries}lp{0.6\textwidth}}
    \toprule
    \head{Binding} & \head{Meaning}\\
    \midrule
    RET & Choose date at point in calendar.\\
    S-\(\rightarrow\) & One day forward.\\
    S-\(\leftarrow\) & One day backward.\\
    S-\(\downarrow\) & One week forward.\\
    S-\(\uparrow\) & One week backward.\\
    M-S-\(\rightarrow\) & One month forward.\\
    M-S-\(\leftarrow\) & One month backward.\\
    > & Scroll calendar forward by one month.\\
    < & Scroll calendar backward by one month.\\
    C-v & Scroll calendar forward by 3 months.\\
    M-v & Scroll calendar backward by 3 months.\\
    C-. & Select today's date.\\
    \bottomrule
  \end{tabular}
  \caption{Minibuffer commands}
  \label{tab:minibuffer-cmds}
\end{table}

\subsection{Deadlines and Scheduling}
\label{sec:deadlines-scheduling}

A timestamp may be preceded by special keywords to facilitate planning.
Both the timestamp and the keyword have to be positioned immediately after the task they refer to.


\begin{itemize}[itemsep=10pt]
\item \keyword{DEADLINE}\\
\begin{verbatim}
* TODO Task
  DEADLINE: <2022-11-07 Mon>
\end{verbatim}
\item \keyword{SCHEDULED}\\
\begin{verbatim}
* TODO Task
  SCHEDULED: <2022-11-07 Mon>
\end{verbatim}
\end{itemize}



\begin{table}[H]
  \centering
  \begin{tabular}{>{\bfseries}ll}
    \toprule
    \head{Binding} & \head{Meaning}\\
    \midrule
    C-c C-d & Insert \keyword{DEADLINE} keyword along with a stamp.\\
    C-c C-s & Insert \keyword{SCHEDULED} keyword along with a stamp.\\
    \bottomrule
  \end{tabular}
  \caption{}
  \label{tab:}
\end{table}

\subsubsection{Repeated tasks}
\label{sec:repeated-tasks}

Some tasks need to be repeated again and again.
Org mode helps to organize such tasks using a so-called \keyword{repeater} in a ‘DEADLINE’, ‘SCHEDULED’, or plain timestamps.
\begin{verbatim}
** TODO Pay the rent
   DEADLINE: <2022-11-07 Mon +1m>
\end{verbatim}

the \keyword{+1m} is a repeater; the intended interpretation is that the task has a deadline on ‘<2022-11-07>’ and repeats itself every (one) month starting from that time.
You can use yearly, monthly, weekly, daily and hourly repeat cookies by using the ‘y’, ‘m’, ‘w’, ‘d’ and ‘h’ letters.
If you need both a repeater and a special warning period in a deadline entry, the repeater should come first and the warning period last.
\begin{verbatim}
   DEADLINE: <2022-11-07 Mon +1m -3d>
\end{verbatim}



Deadlines and scheduled items produce entries in the agenda when they are over-due, so it is important to be able to mark such an entry as done once you have done so.
When you mark a ‘DEADLINE’ or a ‘SCHEDULED’ with the TODO keyword ‘DONE’, it no longer produces entries in the agenda.
The problem with this is that then also the next instance of the repeated entry will not be active.
Org mode deals with this in the following way: when you try to mark such an entry as done, using C-c C-t, it shifts the base date of the repeating timestamp by the repeater interval, and immediately sets the entry state back to TODO.
In the example above, setting the state to ‘DONE’ would actually switch the date like this:
\begin{verbatim}
** TODO Pay the rent
   DEADLINE: <2022-12-07 Wed +1m>
\end{verbatim}

To mark a task with a repeater as DONE, use \keyword{C-{}- 1 C-c C-t}, i.e., with a numeric prefix argument of \keyword{-1}.


With the \keyword{+1m} cookie, the date shift is always exactly one month.
Org mode has special repeaters \keyword{++} and \keyword{.+}.

\begin{verbatim}
** TODO Call Father
   DEADLINE: <2008-02-10 Sun ++1w>
   Marking this DONE shifts the date by at least one week, but also
   by as many weeks as it takes to get this date into the future.
   However, it stays on a Sunday, even if you called and marked it
   done on Saturday.

** TODO Check the batteries in the car
   DEADLINE: <2005-11-01 Tue .+1m>
   Marking this DONE shifts the date to one month after today.
\end{verbatim}


\newpage{}
\subsection{Clocking Work Time}
\label{sec:clocking-work-time}

\begin{table}[H]
  \centering
  \begin{tabular}{>{\bfseries}lp{0.6\textwidth}}
    \toprule
    \head{Binding} & \head{Meaning}\\
    \midrule
    C-c C-x C-i & Start the clock on the current item (clock-in).
                  When called with a C-u prefix argument, select the task from a list of recently clocked tasks.
                  With two C-u C-u prefixes, clock into the task at point and mark it as the default task.
                  With three C-u C-u C-u prefixes, force continuous clocking by starting the clock when the last clock stopped.\\
    C-c C-x C-o & Stop the clock (clock-out).\\
    C-c C-x C-x & Re-clock the last clocked task. With one C-u prefix argument, select the task from the clock history.
                  With two C-u prefixes, force continuous clocking by starting the clock when the last clock stopped.\\
    C-c C-x C-e & Update the effort estimate for the current clock task.\\
    C-c C-x C-q & Cancel the current clock. \\
    C-c C-x C-j & Jump to the headline of the currently clocked in task.
                  With a C-u prefix argument, select the target task from a list of recently clocked tasks.\\
    C-c C-x C-d & Display time summaries for each subtree in the current buffer.\\
    \midrule
    C-c C-x x & Insert or update a clock table.\\
    \bottomrule
  \end{tabular}
  \caption{Clocking commands}
  \label{tab:}
\end{table}

\subsubsection{Effort Estimates}
\label{sec:effort-estimates}

\begin{table}[H]
  \centering
  \begin{tabular}{>{\bfseries}lp{0.6\textwidth}}
    \toprule
    \head{Binding} & \head{Meaning}\\
    \midrule
    C-c C-x e & Set the effort estimate for the current entry. \\
    C-c C-x C-e & Modify the effort estimate of the item currently being clocked.\\
    \bottomrule
  \end{tabular}
  \caption{Effort commands}
  \label{tab:}
\end{table}

\subsubsection{Relative Timer}
\label{sec:relative-timer}

Org provides two types of timers.
There is a relative timer that counts up, which can be useful when taking notes during, for example, a meeting or a video viewing.
There is also a countdown timer.
The relative and countdown are started with separate commands.

\begin{table}[H]
  \centering
  \begin{tabular}{>{\bfseries}lp{0.6\textwidth}}
    \toprule
    \head{Binding} & \head{Meaning}\\
    \midrule
    C-c C-x 0 & Start or reset the relative timer.
                By default, the timer is set to 0.
                When called with a C-u prefix, prompt the user for a starting offset.\\
    C-c C-x ; & Start a countdown timer.\\
    \midrule
    \multicolumn{2}{l}{Once started, relative and countdown timers are controlled with the same commands.}\\
    \midrule
    C-c C-x . & Insert a relative time into the buffer.\\
    C-c C-x - & Insert a description list item with the current relative time.
                With a prefix argument, first reset the timer to 0.\\
    C-c C-x , & Pause the timer, or continue it if it is already paused.\\
    C-c C-x \_ & Stop the timer.\\
    \bottomrule
  \end{tabular}
  \caption{Timer}
  \label{tab:}
\end{table}



\section{Tags}
\label{sec:tags}

An excellent way to implement labels and contexts for cross-correlating information is to assign \keyword{tags} to headlines.
Every headline can contain a list of tags; they occur at the end of the headline.
Tags are normal words containing letters, numbers, ‘\_’, and ‘@’.
Tags must be preceded and followed by a single colon, e.g., \keyword{:work:}.
Several tags can be specified, as in \keyword{:work:urgent:}.


Tags make use of the hierarchical structure of outline trees.
If a heading has a certain tag, all subheadings inherit the tag as well.
To limit tag inheritance to specific tags, or to turn it off entirely, use the variables \argument{org-use-tag-inheritance} and \argument{org-tags-exclude-from-inheritance}.



\begin{table}[H]
  \centering
  \begin{tabular}{>{\bfseries}lp{0.6\textwidth}}
    \toprule
    \head{Binding} & \head{Meaning}\\
    \midrule
    C-c C-q & Enter new tags for the current headline.\\
    C-u C-c C-q & Align all tags to make things look nice.\\
    \bottomrule
  \end{tabular}
  \caption{Tag commands}
  \label{tab:tag-cmds}
\end{table}



Org supports tag insertion based on a list of tags.
By default this list is constructed dynamically, containing all tags currently used in the buffer.
You may also globally specify a hard list of tags with the variable \argument{org-tag-alist}.


\section{Agenda Views}
\label{sec:agenda-views}

Due to the way Org works, TODO items, time-stamped items, and tagged headlines can be scattered throughout a file or even a number of files.
To get an overview of open action items, or of events that are important for a particular date, this information must be collected, sorted and displayed in an organized way.

Org can select items based on various criteria and display them in a separate buffer.
Six different view types are provided:
\begin{itemize}
\item \keyword{agenda} that is like a calendar and shows information for specific dates,
\item \keyword{TODO list} that covers all unfinished action items,
\item \keyword{match view}, showings headlines based on the tags, properties, and TODO state associated with them,
\item \keyword{text search view} that shows all entries from multiple files that contain specified key- words,
\item \keyword{stuck projects view} showing projects that currently do not move along, and
\item \keyword{custom views} that are special searches and combinations of different views.
\end{itemize}

The extracted information is displayed in a special agenda buffer.
This buffer is read-only, but provides commands to visit the corresponding locations in the original Org files, and even to edit these files remotely.


\subsection{Agenda Files}
\label{sec:agenda-files}

The information to be shown is normally collected from all agenda files, the files listed in the variable \argument{org-agenda-files}.
If a directory is part of this list, all files with the extension ‘.org’ in this directory are part of the list.


\begin{table}[H]
  \centering
  \begin{tabular}{>{\bfseries}ll}
    \toprule
    \head{Binding} & \head{Meaning}\\
    \midrule
    C-c [ & Add current file to the list of agenda files.\\
    C-c ] & Remove current file from the list of agenda files.\\
    C-, & Cycle through agenda file list, visiting one file after the other.\\
    C-c C-x < & Restrict the agenda to the current subtree.\\
    C-c C-x > & Remove the restriction created by C-c C-x <.\\
    \bottomrule
  \end{tabular}
  \caption{Agenda files commands}
  \label{tab:agenda-files-cmds}
\end{table}


\subsection{The Agenda Dispatcher}
\label{sec:agenda-dispatcher}

The views are created through a dispatcher, accessible with \keyword{C-c a}.
It displays a menu from which an additional letter is required to execute a command.




%%% Local Variables:
%%% mode: latex
%%% TeX-master: "emacs"
%%% End:

\section{org-roam}
\label{sec:org-roam}

Org-roam is a tool for networked thought.
It reproduces some of Roam Research’s\footnote{\url{https://roamresearch.com/}}  key features within Org-mode.


Org-roam implement the ``Zettelkasten'' method or ``slip-box'' digitally.
The Zettelkasten is a personal tool for thinking and writing.
It places heavy emphasis on connecting ideas, building up a web of thought.

This method is attributed to German sociologist Niklas Luhmann, who using the method had produced volumes of written works.
Luhmann’s slip-box was simply a box of cards.
These cards are small – often only large enough to fit a single concept.
The size limitation encourages ideas to be broken down into individual concepts.
These ideas are explicitly linked together.
The breakdown of ideas encourages tangential exploration of ideas, increasing the surface for thought.
Making linking explicit between notes also encourages one to think about the connections between concepts.

At the corner of each note, Luhmann ascribed each note with an ordered ID, allowing him to link and jump between notes.
In Org-roam, we simply use hyperlinks.

Org-roam is the slip-box, digitalized in Org-mode.
Every zettel (card) is a plain-text, Org-mode file.
In the same way one would maintain a paper slip-box.

\subsection{Installation}
\label{sec:installation-2}

Add the following into your emacs configuration file.
\begin{lstlisting}[language=elisp]
(require 'package)
(add-to-list 'package-archives
             '("melpa" . "http://melpa.org/packages/") t)

(add-to-list 'package-archives '("org" . "https://orgmode.org/elpa/") t)
\end{lstlisting}

Then execute the following command in emacs:
\begin{lstlisting}
M-x package-refresh-contents RET
M-x package-install RET org-roam RET
\end{lstlisting}


\subsection{The Basic}
\label{sec:basic-1}

\subsubsection{The Org-roam Node}
\label{sec:org-roam-node}


A node is any headline or top level file with an ID.
For example, with this example file content:

\begin{verbatim}
:PROPERTIES:
:ID:       foo
:END:
#+title: Foo

* Bar
:PROPERTIES:
:ID:       bar
:END:
\end{verbatim}

We create two nodes:

\begin{itemize}
\item A file node “Foo” with id foo.
\item A headline node “Bar” with id bar.
\end{itemize}

\subsubsection{Links between Nodes}
\label{sec:links-between-nodes}


Headlines without IDs will not be considered Org-roam nodes.
Org IDs can be added to files or headlines via the interactive command \argument{M-x org-id-get-create}.


We link between nodes using Org’s standard ID link (e.g. id:foo).
While only ID links will be considered during the computation of links between nodes, Org-roam caches all other links in the documents for external use.

\subsubsection{Setting up Org-roam}
\label{sec:setting-up-org}


Org-roam’s capabilities stem from its aggressive caching: it crawls all files within \argument{org-roam-directory}, and maintains a cache of all links and nodes.

To start using Org-roam, pick a location to store the Org-roam files.
The directory that will contain your notes is specified by the variable \argument{org-roam-directory}.
Org-roam searches recursively within \argument{org-roam-directory} for notes.
This variable needs to be set before any calls to Org-roam functions.

\begin{lstlisting}
(use-package org-roam
  :ensure t
  :config
  ;; set org-roam-directory.
  ;; The file-truename function is only necessary when you use symbolic links inside org-roam-directory: Org-roam does not resolve symbolic links.
  ;; One can however instruct Emacs to always resolve symlinks, at a performance cost:
  ;; (setq find-file-visit-truename t)
  (setq org-roam-directory (file-truename "~/note/org-roam")
        org-roam-capture-templates
        '(("d" "default" plain "%?" :target
           (file+head "${slug}-%<%Y%m%d%H%M%S>.org" "#+title: ${title}
")
           :unnarrowed t)
          ("m" "machine learning" plain "%?" :target
           (file+head "machine-learning/${slug}-%<%Y%m%d%H%M%S>.org" "#+title: ${title}
")
           :unnarrowed t)))
  ;; setup Org-roam to run functions on file changes to maintain cache consistency.   
  (org-roam-db-autosync-mode)
  :bind (("\C-ci" . org-roam-node-insert)
         ("\C-cf" . org-roam-node-find)))

(use-package org-roam-ui
  :after org-roam
  ;;         normally we'd recommend hooking orui after org-roam, but since org-roam does not have
  ;;         a hookable mode anymore, you're advised to pick something yourself
  ;;         if you don't care about startup time, use
  ;;  :hook (after-init . org-roam-ui-mode)
  :config
  (setq org-roam-ui-sync-theme t
        org-roam-ui-follow t
        org-roam-ui-update-on-save t
        org-roam-ui-open-on-start t))

;; Use M-x org-roam-ui-mode RET to enable the global mode.
;; It will start a web server on http://127.0.0.1:35901/ and connect to it via a WebSocket for real-time updates.


\end{lstlisting}






\subsubsection{Creating and Linking Nodes}
\label{sec:creat-link-nodes}

\begin{itemize}
\item \funcword{org-roam-node-insert}: creates a node if it does not exist, and inserts a link to the node at point.
\item \funcword{org-roam-node-find}: creates a node if it does not exist, and visits the node.
\item \funcword{org-roam-capture}: creates a node if it does not exist, and restores the current window configuration upon completion.
\end{itemize}

If you want to insert whitespace in node, use \argument{C-q} prefix.


%%% Local Variables:
%%% mode: latex
%%% TeX-master: "emacs"
%%% End:
