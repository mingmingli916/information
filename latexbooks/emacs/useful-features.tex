
\chapter{Useful Features}
\label{cha:useful-features}

\section{Get Help}
\label{sec:get-help}

Emacs has extensive help.
You should always check the help information to learn how to use the command, how to configure the Emacs, how to search for what you want and so on.


The whole help entry is \keyword{C-h C-h}.
You should read the help buffer carefully to find what the next help information you want to get.


For example, if you want to replace some string but forget the key binding and the function, you can use \keyword{C-h a replace Enter}. You will get all the functions containing the keyword \keyword{replace}.


\section{Auto Fill}
\label{sec:auto-fill}

\subsection{Dynamic Expanding}
\label{sec:dynamic-expanding}


\keyword{M-/} runs the command \argument{dabbrev-expand}.
It expands previous word ``dynamically''.
Expands to the most recent, preceding word for which this is a prefix.
If no suitable preceding word is found, words following point are considered.
It also search other buffers.

\subsection{Auto Correct Word}
\label{sec:auto-correct-word}

If flyspell-mode is enabled,
\keyword{C-M-i} runs the command \argument{flyspell-auto-correct-word}.
This command proposes various successive corrections for the current word.
If invoked repeatedly on the same position, it cycles through the possible corrections of the current word.

Otherwise, \keyword{C-M-i} runs corresponding command to do completion.

\subsection{Auto Fill Commands}
\label{sec:auto-fill-commands}
If you are in minibuffer, you can use \keyword{Tab} to auto-fill commands and filenames.

For example, type \keyword{M-x open Tab}. If there are only one command matching the ``open'' prefix, it will be auto-filled.
If there are more commands matching the prefix, type \keyword{Tab} again to get all candidates. In AUCTeX mode, you use \keyword{C-c C-m} to start a macro.
In the minibuffer, you can use \keyword{Tab} to auto-fill \LaTeX{}\xspace command.

If you type \keyword{C-x C-f} to open a file. In the minibuffer, you can use \keyword{Tab} to auto-fill the filename or use \keyword{Tab Tab} to get all candidates.



\section{Canceling Commands}
\label{sec:canceling-commands}

When you want to cancel any command that’s in progress, press \keyword{C-g}.
The word \keyword{Quit} appears in the command area.
This command is helpful when you are stuck in the minibuffer and didn’t really mean to go there.
Depending on what you were doing, you may have to press \keyword{C-g} a few times.


\section{Undoing Changes}
\label{sec:undoing-changes}

What happens if you make a mistake while you’re editing? You can undo your changes by \keyword{undo} function.
Type \keyword{C-h k undo Enter} and choose your favorite key binding.


What if you’d like to redo a command after you type undo?
There is no formal redo command, but you can use undo in the following way.
Just move the cursor in any direction, and use undo again.
Emacs redoes the last command you undid.
You can repeat it to redo previous undos.


\section{History Commands}
\label{sec:history-commands}

Emacs will record the commands you have used before, so you can reuse them to avoiding typing them again to save you time.


\keyword{C-x Esc Esc} will let you edit then re-evaluate the last complex command.
A \keyword{ command} is one that used the minibuffer.
The command is placed in the minibuffer as a Lisp form for editing.
The result is executed, repeating the command as changed.
In the minibuffer, you can use \keyword{M-n} and \keyword{M-p} to navigate fore ward and backward the history.


\keyword{C-x z} repeats the most recently executed command.


In LaTeX mode, when you use \keyword{C-c C-e} to insert an environment, 
In the minibuffer, you can use \keyword{M-n} and \keyword{M-p} to navigate fore ward and backward the history.


\section{C-u}
\label{sec:c-u}

\keyword{C-u} runs the command \argument{universal-argument}.
It has the following situations:
\begin{itemize}
\item following \keyword{digits}, it will repeat the following command \keyword{digits} times
\item following \keyword{- or - digits}, some commands support negative digits and it will repeat the following command \keyword{1 or digits} times in the opposite direction
\item without digits or minus sign, it provides 4 as argument
\item repeating \keyword{C-u} without digits or minus sign many times, it multiplies the argument by 4 each time
\item for some commands, just \keyword{C-u} by itself serves as a flag that change the command function.
\end{itemize}



\section{Saving Positions in Registers}
\label{sec:saving-posit-regist}

Typing \keyword{C-x r SPC (point-to-register)}, followed by a character \keyword{r}, saves both the position of point and the current buffer in register \keyword{r}.
The register retains this information until you store something else in it.

The command \keyword{C-x r j r} switches to the buffer recorded in register \keyword{r}, pushes a mark, and moves point to the recorded position.
(The mark is not pushed if point was already at the recorded position, or in successive calls to the command.)
The contents of the register are not changed, so you can jump to the saved position any number of times.


\section{rgrep}
\label{sec:rgrep}

\keyword{M-x rgrep} can recursively search the content in the directory and files you specified.
%%% Local Variables:
%%% mode: latex
%%% TeX-master: "emacs"
%%% End:
