
\chapter{Editing}
\label{cha:editing}

\section{Moving Cursor}
\label{sec:moving-cursor}

\begin{table}[H]
  \centering
  \begin{tabular}{l>{\bfseries}lp{0.5\textwidth}}
    \toprule
    \head{Group} & \head{Binding} & \head{Meaning}\\
    \midrule
    \multirow{4}{*}{char} & C-f (f: forward) & forward one char\\
                 & C-b (b: backward) & backward one char\\
                 & C-n (n: next line) & downward one char\\
                 & C-p (p: previous line) & upward one char\\
    \midrule
    \multirow{2}{*}{word} & M-f & forward one word\\
                 & M-b & backward one word\\
    \midrule
    \multirow{2}{*}{line} & C-a & beginning of line\\
                 & C-e & end of line\\
    \midrule
    \multirow{2}{*}{sentence} & M-a & start of sentence\\
                 & M-e & end of sentence\\
    \midrule
    \multirow{2}{*}{paragraph} & M-\{ & start of paragraph\\
                 & M-\} & end of paragraph\\
    \midrule
    \multirow{2}{*}{screen} & C-v & downward one screen\\
                 & M-v & upward one screen\\
                 & C-l & scroll the window so that current line is in the middle of the window, type again, cursor on top, again, bottom\\
                 & C-M-v & scroll other window\\
    \midrule
    \multirow{2}{*}{buffer} & M-< & start of buffer\\
                 & M-> & end of buffer\\
    \midrule
                 & M-m & first non-whitespace character on this line\\
    \midrule
                 & F10 & start key navigation of the menu bar\\
    \bottomrule
  \end{tabular}
  \caption{Moving Cursor}
  \label{tab:moving-cursor}
\end{table}

Keep in mind that: \keyword{Meta} bindings move larger than \keyword{Ctrl} bindings.
Table \ref{tab:moving-cursor} show commands for moving cursor.


\section{Marking}
\label{sec:marking}

To define a region using the keyboard, you use a secondary pointer called a \keyword{mark}. You set the mark, move the cursor to somewhere to define a region between the mark and the cursor.

Table \ref{tab:marking} show commands for marking.
\begin{table}[H]
  \centering
  \begin{tabular}{>{\bfseries}ll}
    \toprule
    \head{Binding} & \head{Meaning}\\
    \midrule
    C-Space or C-@ & set the mark\\
    C-x C-x & exchange point and mark\\
    M-h & mark paragraph\\
    C-x h & mark buffer\\
    \bottomrule
  \end{tabular}
  \caption{Marking}
  \label{tab:marking}
\end{table}


\keyword{C-Space} sets mark and highlights the \keyword{region}.
The region is still there even if you can not see it.
Because the region is define between mark and point.
The mark is there even if you can not see it.

\section{Deleting, Copying and Pasting}
\label{sec:delet-copy-past}
Table \ref{tab:del-cop-pas} show commands for deleting, copying and pasting.
\begin{table}[H]
  \centering
  \begin{tabular}{l>{\bfseries}ll}
    \toprule
    \head{Group} & \head{Binding} & \head{Meaning}\\
    \midrule
    \multirow{8}{*}{delete} & Del & delete backward char\\
                 & C-d (d: delete)& delete current char\\
                 & M-Del & delete between start of word and one char before cursor\\
                 & M-d & delete between cursor and end of word\\
                 & C-k (k: kill) & delete between cursor and end of line\\
                 & M-k & delete between cursor and end of sentence\\
                 & M-{}- M-k & delete between start of sentence and one char before cursor\\
                 & C-w & delete marked region\\
    \midrule
    copy & M-w & copy marked region\\
    \midrule
    past & C-y (y: yank)& past the most recently deleted or copied\\
    \bottomrule
  \end{tabular}
  \caption{Deleting, Copying and Pasting}
  \label{tab:del-cop-pas}
\end{table}

After the \keyword{C-y}, you can immediately use \keyword{M-y} several times to navigate the kill ring.


\section{Transposition and Capitalization}
Transposition and capitalization can be a edit trick and save you time.
Here are some commands shown in Table \ref{tab:trans-cap} for achieving that.
\begin{table}[H]
  \centering
  \begin{tabular}{>{\bfseries}ll}
    \toprule
    \head{Binding} & \head{Meaning}\\
    \midrule
    C-t & interchange characters around point, moving forward one character\\
    M-t & Interchange words around point, leaving point at end of them\\
    C-x C-t & exchange current line and previous line, leaving point after both\\
    M-c (c: capital)& capitalize from point to the end of word, moving over\\
    M-u (u: upper) & convert to upper case from point to end of word, moving over\\
    M-l (l: lower) & convert to lower case from point to end of word, moving over\\
    \bottomrule
  \end{tabular}
  \caption{Transposition and Capitalization}
  \label{tab:trans-cap}
\end{table}

\section{Rectangle Editing}
\label{sec:rectangle-editing}

You can select a region and do region editing.
Refer to Table \ref{tab:rect-edit} for this kind of editing.
\begin{table}[H]
  \centering
  \begin{tabular}{>{\bfseries}ll}
    \toprule
    \head{Binding} & \head{Meaning}\\
    \midrule
    C-x r k (r: rectangle; k: kill)& delete a rectangle and store it\\
    C-x r c (c: clear) & using spaces, blank out the rectangle and do not store it\\
    C-x r o & insert a blank rectangle, shift text right\\
    \midrule
    C-x r y & insert the last rectangle killed\\
    \midrule
    C-x r r $\beta$ & copy rectangle to register $\beta$ ($\beta$ is any character)\\
    C-x r i $\beta$ & insert rectangle from register $\beta$\\
    \midrule
    C-x r t \argument{string} Enter & change contents of rectangle to \argument{string}\\
    \bottomrule
  \end{tabular}
  \caption{Rectangle Editing}
  \label{tab:rect-edit}
\end{table}



%%% Local Variables:
%%% mode: latex
%%% TeX-master: "emacs"
%%% End:
