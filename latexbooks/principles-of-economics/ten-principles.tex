
\chapter{Ten principles of economics}


The word economy comes from the Greek word for ``one who manages a household.''
Economics is the study of how society manages its scarce resources.

\section{How people make decisions}

\subsection*{PRINCIPLE 1: PEOPLE FACE TRADEOFFS}

To get one thing that we like, we usually have to give up another thing that we like.
Making decisions requires trading off one goal against another.


\subsection*{PRINCIPLE 2: THE COST OF SOMETHING IS WHAT YOU GIVE UP TO GET IT}

Because people face tradeoffs, making decisions requires comparing the costs and benefits of alternative courses of action.
The opportunity cost of an item is what you give up to get that item.

\subsection*{PRINCIPLE 3: RATIONAL PEOPLE THINK AT THE MARGIN}

Economists use the term marginal changes to describe small incremental adjustments to an existing plan of action.
Keep in mind that ``margin'' means ``edge,'' so marginal changes are adjustments around the edges of what you are doing.

\subsection*{PRINCIPLE 4: PEOPLE RESPOND TO INCENTIVES}

Because people make decisions by comparing costs and benefits, their behavior may change when the costs or benefits change.
That is, people respond to incentives. 

\section{How people interact}

\subsection*{PRINCIPLE 5: TRADE CAN MAKE EVERYONE BETTER OFF}

Trade allows each person to specialize in the activities he or she does best, whether it is farming, sewing, or home building.
By trading with others, people can buy a greater variety of goods and services at lower cost.
Countries as well as families benefit from the ability to trade with one another.
Trade allows countries to specialize in what they do best and to enjoy a greater variety of goods and services.

\begin{tcolorbox}
  Thus, it is import to decide what you want to \emph{specialize} in.
\end{tcolorbox}


\subsection*{PRINCIPLE 6: MARKETS ARE USUALLY A GOOD WAY TO ORGANIZE ECONOMIC ACTIVITY}

In a market economy, the decisions of a central planner are replaced by the decisions of millions of firms and households.
Firms decide whom to hire and what to make.
Households decide which firms to work for and what to buy with their incomes.
These firms and households interact in the marketplace, where prices and self-interest guide their decisions.

In his 1776 book \emph{An Inquiry into the Nature and Causes of the Wealth of Nations}, economist Adam Smith made the most famous observation in all of economics: Households and firms interacting in markets act as if they are guided by an ``invisible hand'' that leads them to desirable market outcomes.

Prices are the instrument with which the invisible hand directs economic activity.
Prices reflect both the value of a good to society and the cost to society of making the good.
Because households and firms look at prices when deciding what to buy and sell, they unknowingly take into account the social benefits and costs of their actions.
As a result, prices guide these individual decisionmakers to reach outcomes that, in many cases, maximize the welfare of society as a whole.

\subsection*{PRINCIPLE 7: GOVERNMENTS CAN SOMETIMES IMPROVE MARKET OUTCOMES}

Although markets are usually a good way to organize economic activity, this rule has some important exceptions.
There are two broad reasons for a government to intervene in the economy: to promote efficiency and to promote equity.
That is, most policies aim either to enlarge the economic pie or to change how the pie is divided.



The invisible hand usually leads markets to allocate resources efficiently.
Nonetheless, for various reasons, the invisible hand sometimes does not work.
Economists use the term \emph{market failure} to refer to a situation in which the market on its own fails to allocate resources efficiently.



One possible cause of market failure is an \emph{externality}.
An externality is the im- pact of one person’s actions on the well-being of a bystander.
Another possible cause of market failure is \emph{market power}.
Market power refers to the ability of a single person (or small group of people) to unduly influence market prices.


\section{How the economy as a whole works}

\subsection*{PRINCIPLE 8: A COUNTRY’S STANDARD OF LIVING DEPENDS ON ITS ABILITY TO PRODUCE GOODS AND SERVICES}

\subsection*{PRINCIPLE 9: PRICES RISE WHEN THE GOVERNMENT PRINTS TOO MUCH MONEY}

What causes inflation? In almost all cases of large or persistent inflation, the culprit turns out to be the same—growth in the quantity of money.
When a government creates large quantities of the nation’s money, the value of the money falls.



\subsection*{PRINCIPLE 10: SOCIETY FACES A SHORT-RUN TRADEOFF BETWEEN INFLATION AND UNEMPLOYMENT}

If inflation is so easy to explain, why do policymakers sometimes have trouble ridding the economy of it?
One reason is that reducing inflation is often thought to cause a temporary rise in unemployment.
The curve that illustrates this tradeoff between inflation and unemployment is called the Phillips curve, after the economist who first examined this relationship.


Why do we face this short-run tradeoff?
According to a common explanation, it arises because some prices are slow to adjust.
Suppose, for example, that the government reduces the quantity of money in the economy.
In the long run, the only result of this policy change will be a fall in the overall level of prices.
Yet not all prices will adjust immediately.
It may take several years before all firms issue new catalogs, all unions make wage concessions, and all restaurants print new menus.
That is, prices are said to be sticky in the short run.


Because prices are sticky, various types of government policy have short-run effects that differ from their long-run effects.
When the government reduces the quantity of money, for instance, it reduces the amount that people spend.
Lower spending, together with prices that are stuck too high, reduces the quantity of goods and services that firms sell.
Lower sales, in turn, cause firms to lay off workers.
Thus, the reduction in the quantity of money raises unemployment temporarily until prices have fully adjusted to the change.



