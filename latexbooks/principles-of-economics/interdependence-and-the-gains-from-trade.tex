
\chapter{Interdependence and the gains from trade}

\section{The principle of comparative advantage}

\subsection{Absolute advantage}

Economists use the term \keyword{absolute advantage} when comparing the productivity of one person, firm, or nation to that of another.
The producer that requires a smaller quantity of inputs to produce a good is said to have an absolute advantage in producing that good.

\subsection{Opportunity cost and comparative advantage}

The opportunity cost of some item is what we give up to get that item.
Economists use the term \keyword{comparative advantage} when describing the opportunity cost of two producers.
The producer who has the smaller opportinity cost of producing a good is said to have a comparative advantage in producing that good.

\subsection{Comparative advantage and trade}

Differences in opportunity cost and comparative advantage create the gains from trade.
When each person specializes in producing the good for which he or she has a comparative advantage, total production in the economy rises, and this increase in the size of the economic pie can be used to make everyone better off.
In other words, as long as two people have different opportunity costs, each can benefit from trade by obtaining a good at a price lower than his or her opportunity cost of that good.



Trade can benifit everyone in society because it allows people to specialize in activities in which they have a comparative advantage.


