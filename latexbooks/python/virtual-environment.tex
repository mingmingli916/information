
\chapter{Virtual environments}
\label{cha:virtual-environments}

\section{With Python}
\label{sec:with-python}

You can create a virtual environment with the \funcword{python3} command:
\begin{lstlisting}
python3 -m venv virtual-environment-name
\end{lstlisting}

The \argument{-m venv} option runs the \argument{venv} package from the standard library as a standalone script, passing the desired name as an argument.

When you want to start using a virtual environment, you have to ``activate'' it.
If you are using a Linux or macOS computer, you can activate the virtual environment with this command:
\begin{lstlisting}
source venv/bin/activate
\end{lstlisting}

Where \argument{venv} is the virtual environment name you just created.

In the virtual environment you can type \funcword{deactivate} to leave the virtual environment.
\begin{lstlisting}
deactivate
\end{lstlisting}


\section{With conda}
\label{sec:with-conda}



%%% Local Variables:
%%% mode: latex
%%% TeX-master: "python"
%%% End:
