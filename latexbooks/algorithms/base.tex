
\chapter{Algorithm}


\section{What is Algorithm?}
\label{sec:what-algorithm}

The following relation show what is an algorithm.
\begin{tcolorbox}
input -> \keyword{algorithm} -> output  
\end{tcolorbox}


An algorithm is a sequence of computational steps that transform the input into the output.
An algorithm describes a specific computational procedure for achieving the input/output relationship.

\section{Instance of a Problem}
\label{sec:instance-problem}


An instance of a problem is the input needed to compute a solution to the problem.

\section{Correct Algorithms}
\label{sec:correct-algorithms}


For every input instance, the algorithm halts out the correct output.
We say the algorithm is correct.


\section[Two Characteristics]{Two Characteristics of Many Algorithms}
\label{sec:two-char-many}

\begin{itemize}
\item There are many candidate solutions, but finding the one that solve or the one is best is challenge.
\item They have practical applications.
\end{itemize}

\section{Data Structure}
\label{sec:data-structure}


Data structure is a way to store and organize data in order to facilitate access and modifications.
No single data structure works well for all purposes, and it is important to know the strengths and limitations of several of them.

\section{The Core Technique}
\label{sec:core-technique}


learn the technique of algorithm design and analysis.

\section{Hard Problems}
\label{sec:hard-problems}


Like the NP-complete problem, there are problem that has no efficient solutions.
Before you delve into the real problem, take a overview of it.


\section{Algorithm Efficiency}
\label{sec:algorithm-efficiency}


Computers are not infinitely fast and memory is not free, thus the efficiency of a algorithm matters.

\section{Algorithms as a Technology}
\label{sec:algor-as-techn}


Algorithms are at the core of most technologies.



\section{Loop Invariant}
\label{sec:loop-invariant}


Loop invariant is used to help us to understand why an algorithm is correct.
The three elements of loop invariant:


\begin{description}
\item[Initialization] It is true prior to the first iteration of the   loop.
\item[Maintenance] If it is true before an iteration of the loop, it   remains true before the next iteration.
\item[Termination] When the loop terminates, the invariant gives us a useful property that helps show that the algorithm is correct.
\end{description}




\section{Analyzing Algorithms}
\label{sec:analyzing-algorithms}


Analyzing an algorithm is to predict the resources usage.
Resources include time and space (memory, communication bandwidth, computer hardware, computational time\ldots).

\section{Resource Model}

Before analyzing an algorithm, there must be a model to measure the resource cost.


\section[The Information in Algorithm]{The Information You Used in You Algorithm}

The are much information in you problem.
The general is that the more information you use, the more efficient your algorithm is.
Data structure is on way of using the information in you problem.


\section{Core Idea in Modeling}
\label{sec:core-idea-modeling}


When you do algorithm modeling, remember to show the important characteristics of algorithms and suppress the tedious details.

\section{Analysis of a Algorithm}
\label{sec:analysis-algorithm}


In general, the time grows with the size of the input, so it is traditional to describe the running time as the function of the size of its input.


\section{Worst-Case Analysis}
\label{sec:worst-case-analysis}


Because the behavior of an algorithm may be different for each possible input, we need a means for summarizing that behavior in simple, easily understood formulas.


The reason to analyze worst-case running time is:
\begin{itemize}
\item It gives an upper bound on the running time.
\item Worst case ocurrs fairly often.
\item The ``average case'' is often roughly as bad as the worst case.
\end{itemize}

\section{Abstraction}
\label{sec:abstraction}


We often use some simplifying abstractions to ease the algorithm analysis.
These abstractions are:
\begin{itemize}
\item Ignore the actual cost of each statement, using the constants $c_i$ to represent these costs.
\item Ignore the abstract costs $c_i$ ( $an^2 + bn + c$ ).
\item We only use rate of growth or order of growth of the running time ( $\Theta(n^2)$ ) (pronounced ``theta of n-squared'') instead of exact running time function. 

\end{itemize}


\section{Growth of Functions}
\label{sec:growth-functions}


Although we can sometimes determine the exact running time of an algorithm, the extra procision is not usually worth the effort of computing it.
When we look at input sizes large enought to make only the order of growth of the running time relevant, we are studying the ``asymptotic efficiency of algorithms''.





%%% Local Variables:
%%% mode: latex
%%% TeX-master: "algorithms"
%%% End:
