
\chapter{日志}

\section{需求}

明确需求。

在做高清化的平台的过程中,
根据原始图和高清图的对比,
选用的VDSR高清模型,
然后经过调参依然未达到想要的效果,
在加工间观察高清化流程和高清标准后,
其实主要的功能是降噪,
之后选用了降噪模型AutoEncoder,
才达到想要的效果,
浪费的将近一个月的时间。



\section{错误信息}

程序难免会出错误,
出现错误的那刻我们应该高兴,
错误是使我们获取知识的一种深刻的方式,
在错误中,我们竭尽全力会使用我们学习到的知识来排除错误。

出现错误,我们首先仔细的查看错误信息,
去对错误的信息去进行理解,
而不是把错误信息粘贴到浏览器去百度或者google。

错误信息是最重要的信息,
在你理解的情况下,
它好似地图中的路标,
你可以轻松的到达目的地。



\section{失实信息}

分公司回报的产量信息,
是不是有前提的?
当你要基于该信息做出决策的时候,
务必要详细询问,
确认信息是不是错误的或者失实的。


对于下属经常表达失实信息的,
不要让下属下结论,
而是让下属收集问题数据,
自己整理得出结论。


\section{量化工作}

量化自己的工作或者努力。
一方面可以上其他人看到结果,另一方面可以让自己看到进度,看到自己每天
都在进步,进而提供自己的提升自己的动力。


\section{实践}
注重实践。
用实践说话,道理我们很多人都懂,关键是能够做到的并不多。
走马观花,容易磨掉我们的积极性,我们认为的简单其实不是我们认为的那样
简单。






