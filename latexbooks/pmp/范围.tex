
\chapter{范围}

\section{概念}

亲和图是一种分组技术,用以收集需求和收集创意。


\section{错题解析}

2、 [单选] 项目经理经常收到关于某项目的变更请求,但他确保变更控制过程适当地应用中。项目经理本应采取什么措施来防止频繁的变更请求?

The project manager often receive requests for changes to a project, but he ensures that the change control process is being applied appropriately. What should the project manager have done to prevent frequent change requests?

A:确保项目范围获得相关方批准

Ensure the project scope is approved by the stakeholder

B:在项目开始时定义项目范围

Define the project scope at the beginning of the project

C:完成责任分配矩阵(RAM)

Complete the Responsibility Assignment Matrix (RAM)

D:更新需求矩阵

Update the requirements matrix

正确答案:A

你的答案:B

解析:解析:PMBOK(6)P150,5.3-定义范围。“确保正妥善应用变更控制过程”,说明整体变更控制流程是起效的。那么问题出在哪里?出在相关方对于项目的期望不确定,所以会频繁的提出变更请求,那么此时要求项目的相关方多想一想、想清楚了再做决定有助于减少类似问题的发生。选项B,不是导致问题的根本原因。选项C,责任分配矩阵通常用于划分活动和资源之间的职责关系。选项D,对需求矩阵的分析包含在了整体变更控制流程,但现在这个流程没有出现问题,排除。


3、 [单选] 在审查一个长期项目期间,相关方对项目经理关于项目绩效已随时间转移而得到改善的主张表示不认同。他们要求提交详细的报告。项目经理应该怎么做?

During the review of a long-term project, stakeholders disagree with the project manager’s claim that project performance has been improved over time, and they ask for a detailed report. What should the project manager do?

A:完成趋势分析,并报告结果

Complete the trend analysis and escalate the results

B:确定进度绩效指数(SPI),以显示团队的工作效率

Determine the Schedule Performance Index (SPI) to show the team's productivity

C:计算完工尚需估算(ETC),以显示潜在的新完成日期

Calculate the Estimate to Complete (ETC) to show potential new finish dates

D:执行偏差分析,并报告结果

Execute a variance analysis and escalate the results

正确答案:A

你的答案:D

解析:解析:PMBOK(6)P170-5.6控制范围-趋势分析。“对项目得到改善的主张表示不认同”,而趋势分析旨在审查项目绩效随时间的变化情况,以判断绩效是正在改善还是正在恶化,符合题意。选项B,SPI体现项目当前的进度情况。选项C,ETC用于预测项目的完完工日期,包括在选项A当中。选项D,偏差分析通常用于分析项目的执行情况与计划之间的差距。



4、 [单选] 项目可交付成果已经完成,但由于成果不及预想,主要项目相关方不愿接受这些可交付成果。项目经理怎么做?

Project deliverables have been completed, but key project stakeholders are reluctant to accept them because the results are less than expected. What should the project manager do?

A:确保主要项目相关方的要求均得到满足

Ensure that the requirements of key project stakeholders are met

B:与主要项目相关方确认项目范围和可交付成果

Validate project scope and deliverables with key project stakeholders

C:继续开展项目收尾工作

Continue the project closure

D:评估项目所受影响,并相应地激活变更请求过程

Evaluate the impact on the project and activate the change request accordingly

正确答案:B

你的答案:C

解析:解析:PMBOK(6)121-5.5确认范围。“不愿意接受”是主观意见,是否能够获得验收确认要看验收标准。如果项目的可交付成果虽然不理想但是符合了验收的条件,项目相关方仍然要签字确认。选项A,要看项目相关方的要求是否在验收标准中得以明确。选项C,在选项B之后。选项D,如果项目的可交付成果未能满足验收标准,要对缺陷通过变更的方式进行补救,在选项B之后。



8、 [单选] 项目经理正在启动一个新项目,发起人和主要相关方对该项目具有特定的协议和保密要求。其中一些相关方之前并无相互合作经验,或从事过该项目领域的工作。项目经理应主要使用哪种数据收集技术来定义需求?

The project manager is initiating a new project with specific agreements and confidentiality requirements from the sponsor and key stakeholders. Some of these stakeholders have no previous experience of working with each other or have worked in the project area. What data gathering techniques should the project manager primarily use to define requirements?

A:访谈

Interviews

B:头脑风暴

Brainstorming

C:焦点小组会议

Focus group meetings

D:标杆对照

Benchmarking

正确答案:A

你的答案:B

解析:解析:PMBOK(6)P142,5.2.2.2数据收集-访谈,访谈是通过与相关方直接交谈,来获取信息的正式或非正式的方法。访谈也可用于获取机密信息。


9、 [单选] 虽然项目经理已接近在预算范围内按时完成项目,但客户对项目范围提出一项意外需求,结果发现客户想要一个更简单的解决方案。项目经理应该事先采取下列哪项不同做法?

Although the project manager was close to completing the project on time and within budget, the customer made an unexpected demand for the project scope and found that the customer wanted a simpler solution. Which of the following different practices should the project manager take beforehand?

A:与客户一起定义项目范围

Define the project scope with the customer

B:与相关方一起确认项目范围

Work with stakeholders to validate the project scope

C:与客户一起收集需求

Collect requirements with customers

D:在项目团队的帮助下控制范围

Control the scope with the help of the project team

正确答案:A

你的答案:D

解析:解析:PMBOK(6)P150,5.3定义范围。定义范围是制定项目和产品详细描述的过程。注意关键字“事先”,如果能够提前对解决方案做明确的约定,就能避免争议的发生。选项B,确认范围是对项目交付成果的验收。选项C,收集到的需求要经过筛选决策才能列入范围,换而言之,不是所有的需求都会列入范围基准。选项D,控制范围正常执行,从解决方案出现了偏差。


10、 [单选] 项目的第一批可交付成果已准备就绪,项目经理请求相关方批准这些可交付成果,其中一位相关方表示,由于某些需求尚未实施,所以尚未完成工作,项目经理认为这些需求超出范围,项目经理现在应该做什么?

The first batch of deliverables of the project are ready, and the project manager asks stakeholders to approve these deliverables. One of the stakeholders stated that the work has not been completed because some requirements have not been implemented. The project manager believes that these requirements are out of scope. What should the project manager do now?

A:与该相关方一起审查质量管理计划

Review the quality management plan with the stakeholder

B:接受该相关方的立场并实施这些需求

Accept the position of the stakeholder and implement these requirements

C:对该相关方的新需求应用变更控制过程

Apply the change control process to the new requirements of the stakeholder

D:与该相关方一起审查项目范围说明书的项目排除部分

Review the project exclusion portion of the project scope statement with the stakeholder

正确答案:D

你的答案:C

解析:解析:PMBOK(6)P161-5.4.3.1范围基准-范围说明书。在范围上存在分歧,需要查询范围基准。



11、 [单选] 在一个大型的IT实施项目中,客户在确认可交付成果期间发现了缺陷,并且没有在用户验收测试(UAT)上签字,进行进一步分析之后,项目经理将报告的缺陷识别为新需求。在这种情况下,项目经理应该怎么做?

In a large IT implementation project, the project manager identifies the reported defect as a new requirement after the customer discovers a defect during the validation of deliverables and does not sign the User Acceptance Test (UAT) for further analysis. What should the project manager do in this case?

A:与客户分享验收标准

Share acceptance criteria with customers

B:实施已识别的新需求

Implement identified new requirements

C:根据基准计划分析需求

Analyze requirements against a baseline plan

D:遵循升级上报程序

Follow the upgrade escalation procedures

正确答案:A

你的答案:C

解析:解析:PMBOK(6)P165-5.5.1.1,范围基准。题干中的关键字是:“确认可交付成果期间”“新需求”。在项目验收期间应以范围基准中的验收标准为标准核实可交付成果。选项A正确。实施新的需求需要实施变更,排除B。根据基准计划分析需求需要客户售前提交正式的变更请求,排除C。此时项目未达到失控状态,排除D。



12、 [单选] 项目经理已经非常了解项目相关方的期望,若要有效地规划,项目项目经理接下来应该采取什么行动?

The project manager is well aware of the expectations of the project stakeholders and what should the project manager do next to plan effectively?

A:根据资源可用性评估所需的能力

Assess the required capabilities based on resource availability

B:与团队一起举行开工会议

Hold an initiation meeting with the team

C:发现需要并将其分解为需求

Discover the need and break it down into requirements

D:确定项目预算和资金来源

Determine the project budget and source of funding

正确答案:C

你的答案:D

解析:解析:PMBOK(6)P141-5.2.1.3,相关方登记册。题干中的关键字是:“已经非常了解项目相关方的期望”“接下来应该采取什么行动”。所以是选择识别相关方之后最先要进行的子过程,启动过程组之后,规划过程组最先的是识别需求,所以选择C。估算活动资源和制定预算在识别需求之后,排除AD。开工会议在确定项目计划之后,规划结束,执行之前,排除B。


13、 [单选] 在制定范围管理计划期间,相关方要求将实验技术添加到项目范围内,以协助产品营销。项目经理应该做什么?

During the development of the scope management plan, the stakeholders request that experimental techniques be added to the project scope to assist in product marketing. What should the project manager do?

A:将项目储备用于支付试验技术的成本

Use project reserves to cover the cost of pilot technique

B:根据试验技术的成本增加项目预算

Increase the project budget based on the cost of the pilot technique

C:将相关方包含在范围规划会议中

Include stakeholders in scope planning meetings

D:与相关方一起审查范围基准

Review the scope base with the stakeholder

正确答案:C

你的答案:D

解析:解析:PMBOK(6)P136-5.1.2.3,会议。题干中的关键词:“执行范围管理计划期间”。题目中提到制定范围管理计划期间,所以寻找选项中规划范围管理的输入或工具,选项C正确。储备用于进度,成本和风险,排除A。此时还没有开始估算成本,排除B。制定范围管理计划时还未产生范围基准,排除D。


14、 [单选] 一个开放新产品的项目包含许多试图影响项目但利益有冲突的相关方。项目经理只完成了项目章程。正在制定项目管理计划,该计划仅提供产品规范和项目范围的高层级视图。若要构建详细的范围描述。项目经理接下来应该做什么?

A project that opens a new product contains many the stakeholders that have conflicting interests in trying to influence the project sheet. The project manager has only completed the project charter. A project management plan is being developed that provides only a high-level view of product specifications and project scope. To build a detailed scope description. What should the project manager do next?

A:创建工作分解结构(WBS)

Create a work breakdown structure (WBS)

B:实施规划范围管理过程

Implement the plan scope management process

C:采用相关方的意见制定范围说明书

Use the opinions of the stakeholder to develop a scope statement

D:在相关方参与计划中列出所有顾虑

List all concerns in the stakeholder engagement plan

正确答案:B

你的答案:D

解析:解析:PMBOK(6)P134-5.1,规划范围管理。题干中的关键字是:“只完成了项目章程”“高层次试图”。关键词说明,题设中的场景,项目经理刚进入规划阶段,那么需要制定相应的范围管理计划才能开始执行收集需求等制定范围的工作,选项B正确。工作分解结构和范围说明书都依赖范围管理计划,排除AC。相关方的顾虑应该登记在相关方登记册中,排除D。



15、 [单选] 在对非常含糊的报价邀请书(RFQ)作出答复以满足某大企业的需求后,项目经理获得了一个项目。该项目最近刚刚启动。很多细节仍不清晰,现在正在识别额外需求。项目经理该做什么?

After responding to a very vague request for quotation (RFQ) to meet the needs of a large enterprise, the project manager was given a project. The project has just started recently, and many details are still unclear and additional requirements are being identified. What should the project manager do?

A:实施紧急变更控制过程,以分析这一情况

Implement a contingency change control process to analyze the situation

B:定义范围基准,并促使客户予以批准

Define Scope baselines and prompt customer approval

C:阅读并审核合同,并促使所有文档均获签署

Read and review the contract and cause all documents to be signed

D:审核当前情况的时间表

Review the schedule for the current status

正确答案:B

你的答案:C

解析:解析:PMBOK(6)P161-5.4.3.1,范围基准。题干中的关键字:“刚刚启动”“细节仍不清晰”“正在识别额外需求”。根据关键字,项目经理正在收集需求,那么接下来的工作是定义范围和创建WBS。所以选项B正确。由于基准还未批准,所以无需变更,排除A。项目经理已经获得项目,说明章程已经批准,无需再审核合同,排除C。收集需求时,项目仅有大概时间表,无参考意义,排除D。



17、 [单选] 在一个项目中途,一名新项目经理加入团队。在审查项目管理计划后,项目经理意识到项目需求存在差距。项目经理下一步应该做什么?

In the middle of a project, a new project manager joins the team. After reviewing the project management plan, the project manager realized that there was a gap in project requirements. What should the project manager do next?

A:要求项目发起人提供所有遗漏的需求

Request the project sponsor to provide any missing requirements

B:与项目团队合作以收集所有遗漏的需求

Work with the project team to gather any missing requirements

C:与关键相关方开会,以识别与遗漏需求相关的风险

Have a meeting with key stakeholders to identify risks associated with missing requirements

D:安排一次与关键相关方的需求澄清会议

Schedule a requirement clarification meeting with key stakeholders

正确答案:D

你的答案:C

解析:解析:PMBOK(6)P170-5.6.2.1,偏差分析。题干中的关键字:“项目中途”“需求存在差距”。根据题干中的关键字,范围存在差距,需要进行识别和分析,所以和相关方安排会议澄清。选项D正确。项目发起人不负责提供需求,排除A。收集需求在规划阶段开展,此时已经在执行/监控,排除B。选项C在选项D之后。



19、 [单选] 项目经理提供项目范围说明书,但客户希望修改进程-这将显著变更范围,项目经理首先 应该做什么?

The project manager provides the project scope statement, but the customer wants to modify the process - this will significantly change the scope. What should the project manager do first?

A:变更范围以符合新规范

Change the scope to conform to new specifications

B:遵循初始规范中定义的范围

Follow the scope defined in the original specification

C:提交一项变更请求,以适应新进度

Submit a change request to accommodate the new schedule

D:更新预算以考虑新规范

Update the budget to consider the new specification

正确答案:C

你的答案:D

解析:解析:PMBOK(6)P166-5.5.3.3,变更请求。题干中的关键字:“客户希望修改进程”“这将显著变更范围”。根据题意,首先应该提交变更请求。选项C正确。更新范围需要经过变更控制流程,排除A。遵循初始规范,不积极,排除B。更新预算是变更通过后的事情,排除D。


21、 [单选] 设计团队发现项目经理创建的工作分解结构(WBS)遗漏了关键任务。若要避免这个问题,项目经理应该事先做什么?

A design team discovers that the work breakdown structure (WBS) created by the project manager is missing key tasks. What should the project manager have done to avoid this issue?

A:主题专家(SMEs)的专家判断

Seek expert judgment from subject matter experts (SMEs)

B:审查过往项目的历史信息

Review historical information from previous projects

C:创建需求跟踪矩阵

Create a requirements traceability matrix

D:制定一份石川图

Develop an ishikawa diagram

正确答案:C

你的答案:B

解析:解析:PMBOK(6)P148-5.2.3.2,需求跟踪矩阵。题干中的关键字:“创建WBS遗漏了任务”“若要避免……实现做什么”。从关键字中可知,这道题并非为了在事后找出问题的原因,而是提前做可以避免问题的方法。那么应该考虑哪个答案能够加强对范围和需求的跟踪。选项C正确。专家判断和参考历史信息并不能加强需求的跟踪,排除AB。石川图用于问题定位,不能预防。排除D。


22、 [单选] 在举行了几次会议来确定项目活动之后,项目团队仍然不赞同许多事项。项目经理应该使用什么来达成共识?

After several meetings to identify project activities, the project team still disagreed on many issues. What should the project manager use to reach consensus?

A:名义小组技术

Nominal group technique

B:亲和图

Affinity diagrams

C:大多数原则

The majority principle

D:一致同意原则

The unanimity principle

正确答案:C

你的答案:A

解析:解析:PMBOK(6)P144-5.2.2.4,大多数同意。题干中的关键字:“团队仍然不赞同许多事项”“使用什么来达成共识”。从题干中可以知道项目经理需要确定决策采用的方式,而由于不赞同许多事情,一致同意不现实,所以选项C正确。名义小组,亲和图都不是做决策的方式,排除AB。由于不赞同许多,所以一致同意原则很难执行,排除D。



24、 [单选] 项目经理与关键相关方开会,设定明确的项目期望,项目经理使用的是什么工具或技术?

The project manager meets with key stakeholders to set clear project expectations. What tools or skills is used by the project manager?

A:产品分析

The product analysis

B:引导

The guide

C:专家判断

The expert judgment

D:多标准决策分析

The Mult-icriteria Decision Analysis

正确答案:B

你的答案:A

解析:解析:PMBOK(6)P145,5.2.2.6收集需求-引导。引导与主题研讨会结合使用,把主要相关方召集在一起定义产品需求。研讨会可用于快速定义跨职能需求并协调相关方的需求差异。因为具有群体互动的特点,有效引导的研讨会有助于参与者之间建立信任、改进关系、改善沟通,从而有利于相关方达成一致意见。



9、 [单选] 项目经理被任命管理一个现有项目,需要了解项目可交付成果,项目经理应该参考下列哪一份文件?

A project manager is assigned to an existing project and needs to understand the project deliverables. The project manager should refer to which of the following?

A:项目章程

Project charter

B:项目需求规范

Project requirements specification

C:项目范围说明书

Project scope statement

D:项目进度表

Project schedule

正确答案:C

你的答案:A

解析:解析:PMBOK(6)P161-5.4.3.1范围基准。项目范围说明书包括对项目范围、主要可交付成果、假设条件和制约因素的描述。 选项A:项目章程中记录的是高层次需求。 选项B:需求规划了描述了如何收集和记录需求。 选项D:项目进度表主要描述进度上的规划。


26、 [单选] 项目经理与项目相关方和团队成员开会,审查范围管理计划、批准的章程和其他需求文档,专家判断和引导技术用于制定所需产品的详细描述。项目经理还应使用哪一项其他输入?

A project manager meets with project stakeholders and team members to review the scope management plan,the approved charter,and other requirements documentation.Expert judgement and facilitation are used to develop a detailed description of the product required. What other input should the project manager use?

A:过往项目的经验教训

Lessons learned register from previous projects

B:假设日志

Assumption log

C:相关方登记册

Stakeholder register

D:需求跟踪矩阵

Requirements traceability matrix

正确答案:B

你的答案:C

解析:解析:PMBOK(6)P152-5.3.1.3项目文件-假设日志,"制定产品描述"即为"定义范围,题目问输入,选择假设日志。假设日志识别了有关产品、项目、环境、相关方以及会影响项目和产品范围的假设条件和制约因素。 选项A:组织过程资产虽是定义范围过程的输入,但不如选项B更重要。 选项CD:不是定义范围过程的输入。


77、 [单选] 项目团队正在努力进行可交付成果的工作,以满足计划的进度。一名团队成员发现范围蔓延正在影响项目成本。项目经理应该怎么做?

A project team is working hard on deliverables to meet the planned schedule.One team member identifies that scope creep is affecting project costs. What should the project management do?

A:执行风险评估和范围变更管理程序

Perform risk assessment and scope change management procedures

B:允许范围蔓延,并与变更控制委员会(CCB)沟通以获得批准

Allow the scope creep and communicate it to the Change Control Board(CCB) for approval

C:估算对项目的影响,并将结果传达给项目相关方

Estimate the impact on the project and communicate the findings to project stakeholders

D:调查为什么会发生范围蔓延,并立即启动变更管理程序

Investigate why scope creep occurred and immediately initiate change management

正确答案:D

你的答案:A

解析:解析:PMBOK(6)P167-5.6控制范围。控制范围是监督项目和产品的范围状态,管理范围基准变更的过程。范围蔓延是失控的表现,要控制范围和进行风险评估。 选项A:要控制范围蔓延,而不是把蔓延的内容添加到范围基准中去。 选项B:蔓延将导致项目失败。 选项C:评估的结果要进入到整体变更控制流程,而不是提交给相关方(选项没有标明相关方是否是CCB)。


113、 [单选] 项目经理必须创建一个项目的工作分解结构(WBS),并分析项目范围的技术细节。项目经理应使用什么工具或技术?

The project manager must create a project's work breakdown structure (WBS) and analyze the technical details of the project scope. What tools or techniques should the project manager use?

A:头脑风暴

Brainstorming

B:亲和图

Affinity diagram

C:专家判断

Expert judgement

D:紧前关系绘图法

Precedence Diagramming Method(PDM)

正确答案:C

你的答案:D

解析:解析:PMBOK(6)P158-5.4.2.1专家判断。专家判断是一个频繁用到的工具。已经到了创建WBS的阶段,头脑风暴是不适合的,排除A选项。亲和图是一种分组技术,用以收集需求和收集创意,不适用于题干中的场景,排除B选项。紧前关系绘图法用于制定进度计划,排除D选项。


122、 [单选] 项目经理正在执行一个涉及不同业务部门的全公司项目。在一次规划会议上,项目经理注意到每个部门的具体需求不能引起其他部门的兴趣,这影响到会议的质量。 若要解决这个问题,项目经理应该怎么做?

A project manager is running a company-wide project involving different business units.During a planning session, the project manager notices that each unit's specific requirements are of no interest to other units, which affects the quality of the meeting. What should the project manager use to resolve specific this?

A:石川图和需求跟踪矩阵

ishikawa diagram and requirements traceability matrix

B:焦点小组会议和思维导图

Focus groups and mind mapping

C:引导和亲和图

Facilitation and affinity diagram

D:头脑风暴和需求跟踪矩阵

Brainstorming and requirements traceability matrix

正确答案:C

你的答案:B

解析:解析:PMBOK(6)P144-5.2.2.5数据表现-亲和图,PMBOK(6)P145-5.2.2.6人际关系与团队技能-引导。研讨会可用于快速定义跨职能需求并协调相关方的需求差异。 选项A:通常由于分析问题原因和记录需求。 选项BD:部门之间的观点已经出现了分歧,此时再使用头脑风暴更无法达成共识。



133、 [单选] 一家工程咨询公司的设计师已完成设计开发并发布设计用于生产,在核实可交付成果过程中发现设计标准发生了变化,已生产的可交付成果不符合新标准。 若要避免这个问题,项目经理应该事先实施哪个规划过程?

A designer for an engineering consulting company has completes the design and releases it for production. During the deliverable verification process, it is discovered that the design standard was changed, and that the produced deliverable fails to comply with it. What planning process should the project manager have implemented to avoid this issue?

A:项目整合管理

Project Integration Management

B:控制质量

Control Quality

C:规划范围管理

Plan Scope Management

D:规划质量管理

Plan Quality Management

正确答案:C

你的答案:D

解析:解析:PMBOK(6)P167-5.6控制范围。控制范围是监督项目和产品的范围状态,管理范围基准变更的过程。设计标准的变化没有同步到范围基准,范围管理有欠缺。 选项A:没有发现范围的变化,没有进入到整体变更控制流程。 选项B:需要质量管理计划。 选项D:需要输入范围基准,所以还是范围管理上出现了缺失导致了问题的发生。


146、 [单选] 项目经理负责管理一个处于执行阶段的项目,并希望审查项目范围以进行成本结算,项目经理发现在工作分解结构(WBS)中遗漏了一项可交付成果。项目经理下一步应该怎么做?

A project manager assumes a project during its execution stage and wants to review the project scope for cost settlement.The project manager discovers that some deliverables were missed in the work breakdown structure(WBS).What should the project manager do first?

A:询问相关方是否需要添加这些遗漏的可交付成果

Ask the stakeholders if these missed deliverables need to be added

B:请求项目管理办公室(PMO)批准这项工作的额外资金

Request that the project management office(PMO) approve additional funds for this work

C:根据100\%的规则将工作添加到WBS中,并重新计算项目总成本

Add the work to the WBS according to the 100 percent rule and recalculate the total project cost

D:要求主题专家(SMEs)检查这些可交付成果是否有必要

Ask subject matter experts(SMEs) to check whether these deliverables are necessary

正确答案:C

你的答案:A

解析:解析:PMBOK(6)P167-5.6控制范围。控制范围是监督项目和产品的范围状态,管理范围基准变更的过程。遗漏的可交付成果要添加到范围基准,并重新计算成本。 选项A:“询问相关方”错误。 选项B:是否动用资金、以及资金的来源,要根据整体变更控制流程。 选项D:由项目经理和项目管理团队进行评估。


198、 [单选] 项目经理与相关方一起召开一次引导式研讨会,以收集产品需求并制定需求跟踪矩阵。系统开发人员无法参加该研讨会,但请求一份会议记录以审查和确认输出。看完会议记录后,系统开发人员反馈说列出的功能需求与用户的需求无关。 若要解决这种情况,项目经理应该怎么做?

A project manager facilitate a workshop with stakeholders to gather product requirements and develop a requirements traceability matrix. The system developer was unable to attend the workshop, but requested a copy of the minutes to review and validate the output. The system developer then provides feedback that the listed functional requirements do not make sense when competed against the user requirements. What should the project manager do to address this?

A:建议系统开发人员参与下次审查会议

Advice the system developer to attend the next review meeting.

B:通知系统开发人员与用户一起审查需求

Inform the system developer to review the requirements with users.

C:更新需求跟踪矩阵以包含需求的相关关系

Update the requirements traceability matrix to include the relevant relationships of the requirements.

D:要求系统开发人员提交变更请求

Ask the system developer to submit a change request.

正确答案:C

你的答案:B

解析:解析:PMBOK(6)P148-5.2.3.2需求跟踪矩阵。需求跟踪矩阵是把产品需求从其来源连接到能满足需求的可交付成果的一种表格。系统开发人员的意见要更新到需求跟踪矩阵当中,供接下来的过程组5.3定义范围(选项AB)展开讨论。因为还没有在收集到的需求基础上形成范围基准,所以此时还不能处理变更请求,选项D排除。

