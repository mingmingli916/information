
\chapter{成本}

\section{错题}
1、 [单选] 完成进度计划中考虑的所有活动的估算成本为14,500美元。该项目的发起人需要至少90\%的成本基准确定性水平,按照蒙特卡洛模拟的结果,计算的预计成本为:概率 估算成本 40\% 10,000 美元 50\% 12,000 美元 60\% 13,000 美元 70\% 15,000 美元 80\% 18,000 美元 90\% 20,000 美元 100\% 21,000美元 若要满足发起人的要求,应急储备应该是多少?

The estimated cost of completing all activities considered in the schedule is \$14,500. The sponsor of the project needs at least 90\% of the cost baseline determination level and, based on the results of the Monte Carlo simulation, the estimated cost calculated is: probability estimated cost 40\%; \$10,000 50\%; \$12,000 60\%; \$13,000 70\%; \$15,000 80\%; \$18,000 90\%; \$20,000 100\%; \$21,000. What should be the contingency reserve to meet the sponsor's requirements?

A:5,500美元

\$5,500

B:6,500美元

\$6,500

C:7,000美元

\$7,000

D:8,200美元

\$8,200

正确答案:A

你的答案:C

解析:解析:PMBOK(6)P254-7.3.3.1成本基准。按照90\%概率的成本基准,20000美元,其中各活动的累加值是14500美元,所以应急储备要达到20000-14500=5500美元。


2、 [单选] 在项目执行期间,大家注意到实际支出与基准成本估算相比过高。项目经理下一步应该做什么?

During the execution of the project, it was noted that actual expenditures were too high compared to baseline cost estimates. What should the project manager do next?

A:重新审视项目范围并缩减高成本事项

Re-examine the project scope and reduce high-cost matters

B:查看假设日志以了解哪些成本假设是错误的

Review the assumption log to see which cost assumptions are wrong

C:查看历史数据以获得新的成本估算

View historical data for new cost estimates

D:更新成本预测并与项目发起人讨论选项

Update the cost forecast and discuss options with the project sponsor

正确答案:B

你的答案:A


题干关键词“实际支出超出成本基准”。 发生了问题,首先要分析问题,然后才是解决问题,优选B。 选项ACD都是在B之后可能采取的行动之一。



5、 [单选] 一个项目正在经历成本超支,并且预测表明这种成本超支可能会增加。项目经理应该做什么?

A project is experiencing cost overrun, and the forecast shows that this cost overrun may increase. What should the project manager do?

A:要求项目发起人批准使用应急储备

Request the project sponsor to approve the use of the contingency reserve

B:请求项目发起人增加管理储备

Request the project sponsor to increase the management reserve

C:查看商业论证以确定项目的盈利状态,并在必要时建议变更

Review the business case to determine the profitability of the project and recommend changes to if necessary

D:参考项目范围以确定项目的盈利状态,并在必要时建议变更

Refer to the project scope to determine the profitability of the project and recommend changes if necessary

正确答案:C

你的答案:D

解析: 题干关键词“成本超支可能会增加”。 本题用排除法。 选项A和B错误,因为应急储备和管理储备也包含在项目预算之内; 选项D错误,因为项目范围无法提供盈利状态的信息; 所以选C。



6、 [单选] 某经验丰富的项目经理最近被指派管理某经历破产正在复苏的城市的多个项目。该城市预算紧张。在计算非预期预算以便为项目的现金流突发情况作出调整时,项目经理规划预算时应将哪个成本要素考虑在内?

An experienced project manager was recently assigned to manage a number of projects in a city recovering from bankruptcy. The city is on a tight budget. Which cost factor should the project manager take into account when planning the budget when calculating the unexpected budget to make adjustments for the project's cash flow contingencies?

A:实际成本(AC)

Actual cost (AC)

B:挣值(EV)

Earned value (EV)

C:计划价值(PV)

Planned value (PV)

D:成本偏差 (CV)

Cost variance (CV)

正确答案:D

你的答案:C

解析:解析:PMBOK(6)P262-成本偏差。题干关键词“现金流突发情况”。为了应付现金流突发状况,有必要考虑做储备。做储备的前提是要知道挣值与实际成本之间的差距,才好做储备规划。成本偏差CV就是指出了实际绩效与成本支出之间的关系,故选D。ABC都不能直接体现“实际”和“预算”之间的差距,排除。


8、 [单选] 一位重要的相关方坚持使用替代执行方法,预计会增加项目成本,项目经理应该做什么?

What should the project manager do if a significant stakeholder insists on using alternative execution methods that are expected to increase project costs?

A:向相关方解释无法变更执行方法的原因

Request the stakeholders to explain why you cannot change the execution method

B:将潜在的方法变更升级上报给项目发起人以获得批准

Report potential method changes and upgrades to the project sponsor for approval

C:忽略该请求,并继续执行计划的执行方法

Ignore the request and continue with the planned execution method

D:评估该相关方的方法,看看他们的期望是否能够得到满足

Evaluate the methods of the stakeholders to see if their expectations can be met

正确答案:D

你的答案:B

解析:解析:PMBOK(6)P269-7.4.3.3变更请求。 题干关键词“替代执行方法”。 相关方其实是提出了变更,那么就要评估这个变更请求会带来什么影响,选D。 ABC都不符合整体变更控制流程。



9、 [单选] 项目执行六个月后,项目经理确定成本绩效指数(CPI)为 0.9,且趋势分析显示CPI显下降趋势。项目经理下一步应该怎么做?

After six months of project execution, the project manager determined that the cost performance index (CPI) was 0.9, and the trend analysis showed a significant downward trend in CPI. What should the project manager do next?

A:提交重定成本基准的变更请求

Submit a change request to reschedule the cost baseline

B:请求额外的资源

Request additional resources

C:对项目进度赶工

Work on the project schedule

D:使用管理储备

Utilization management reserve

正确答案:A

你的答案:C

解析:解析:PMAOK(6)P269-7.4.3.3变更请求。题干关键词“CPI显下降趋势”。项目经理分析项目绩效后,可能会就成本基准和进度基准,或项目管理计划的其他组成部分提出变更请求。CPI<1说明成本已经超支,CPI显下降趋势说明实际成本将进一步偏离成本基线。选项A正确,通过变更控制程序重新设定成本基线;选项BCD都将进一步使成本超支,进一步偏离成本基线,错误。


10、 [单选] 在项目执行中,项目经理发现了严重成本超支问题。在进行根本原因分析后,项目经理确定批准的预算与原始估算不一致。若要避免这个问题,项目经理应该事先做什么?

During project execution, the project manager discovered significant cost overruns. After a root cause analysis, the project manager determines that the approved budget is inconsistent with the original estimate. To avoid this issue, what should the project manager do beforehand?

A:注意在每次挣值审查期间的趋势,并在第一次出现超支迹象时重新估算

Note trends during each earned value review and re-estimate at the first sign of over-expenditure

B:预测成本超支,并请求提供更高的应急储备以解决潜在的差距

Anticipate cost overruns and request higher contingency reserves to address potential gaps

C:在创建成本管理计划期间识别这种差异,并增加预算

Identify these variations during the creation of the cost management plan and increase the budget

D:减少团队估算的小时数以与原始项目预算保持一致

Reduce the team's estimated hours in line with the original project budget

正确答案:A

你的答案:C

解析:解析:PMBOK(6)P261-挣值分析。 题干关键词“避免”、“事先”。 此题定位在7.4控制成本,应该在控制成本的过程中比较实际成本与成本基线的差距并做好预测,如果发现超支,应采取相应对策,故选A。 选项B错误,因为事先预测成本超支没有依据; 选项C错误,成本管理计划描述如何规划、安排和控制项目成本,并没有识别差异的内容; 选项D错误,没有通过整体变更控制流程。


12、 [单选] 一个正在执行的项目的成本绩效指数(CPI)为1.25,进度绩效指数(SPI)为0.8,计划价值(PV)为10,000美元,完工预算(BAC)为100,000美元。为了让项目按计划完成,必须保持的效率是多少?

The cost performance index (CPI) of an ongoing project is 1.25, the schedule performance index (SPI) is 0.8, the planned value (PV) is $10,000, and the budget at completion (BAC) is $100,000. What is the efficiency that must be maintained in order for the project to be completed as planned?

A:0.728

0.728

B:0.983

0.983

C:1.017

1.017

D:1.563

1.563

正确答案:B

你的答案:C

解析:解析:PMBOK(6)P267-表7-1挣值计算汇总表。 题干关键词“按计划完成”。 本题已知CPI=1.25,SPI=0.8,PV=10000,BAC=100000,问的是TCPI是多少? 需要先求得EV和AC: EV=SPI*PV=0.8*10000=8000, AC=EV/CPI=8000/1.25=6400. 由于要求按计划完成,所以TCPI=(BAC-EV)/(BAC-AC)=(100000-8000)/(100000-6400)=0.983,故选B。


13、 [单选] 一个项目预算为6000万美元,预计需要24个月才能完成。12个月后,该项目完成了 60\%,并使用了 3500美元。那么预算和进度的状态如何?

A project with a budget of \$60 million is expected to take 24 months to complete. Twelve months later, the project was 60\% complete and used \$3,500. What is the status of budget and progress?

A:符合预算,并超前于进度

On budget and ahead of schedule

B:超出预算,但超前于进度

Over the budget and ahead of schedule

C:符合预算和进度

On budget and on schedule

D:落后于进度,并超出预算

Behind schedule and over the budget

正确答案:A

你的答案:B

解析:解析:PMBOK(6)P267-表7-1挣值计算汇总表。 由题干已知,BAC=6000万,24个月完工。 12个月后,EV=60\%*6000万=3600万,PV=50\%*6000=3000万,AC=3500万, 则SPI=EV/PV=3600/3000>1,代表进度超前; CPI=EV/AC=3600/3500>1,代表成本有结余; 综上,选A。




