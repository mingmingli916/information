
\chapter{资源}

\section{错题分析}

2、 [单选] 某项目的进度计划已根据受项目范围驱动的资源承诺确定基准,但由于业务优先级造成资源稀释,该项目正发生渐进式延迟。项目经理应该做什么来纠正这一问题?

A project's schedule has been baselined against project-scope-driven resource commitments, but the project is being delayed incrementally due to the dilution of business priorities into resources. What should the project manager do to correct the issue?

A:遵循风险管理过程

Follow the risk management process

B:遵循相关方参与过程

Follow the stakeholder engagement process

C:遵循变更管理计划

Follow the change management plan

D:遵循沟通管理计划

Follow the communications management plan

正确答案:C

你的答案:A

解析:解析:PMBOK(6)P350-9.5.3.1管理团队-变更请求。被动的“资源稀释”会造成多大的影响?4个选项里面整体变更管理流程可以对此进行评估,特别是对可能造成的连锁反应进行分析。选项A,风险管理过程侧重不确定性事件的管理,资源被稀释已经发生了,所以排除。选项B,相关方参与是针对相关方进行管理,情景中的焦点是资源管理。选项D,沟通管理计划针对的是信息如何传递。

6、 [单选] 项目团队已成功交付了项目,该成功一定程度上归功于有效的团队建设。下列哪一项对该有效的团队建设有最大贡献?

The project team has successfully delivered the project, and this success is partly attributed to effective team building. Which of the following has the greatest contribution to effective team building?

A:有效的组织过程

A valid group process

B:灵活的变更控制程序

Flexible change control procedures

C:有规律的团队会议

Regular team meetings

D:挑战性的项目目标

Challenging project objectives

正确答案:C

你的答案:A

解析:解析:PMBOK(6)P341,9.4.2.4人际关系与团队技能-团队建设。团队建设是通过举办各种活动,强化团队的社交关系,打造积极合作的工作环境。团队建设活动既可以是状态审查会上的五分钟议程,也可以是为改善人际关系而设计的、在非工作场所专门举办的专业提升活动。

10、 [单选] 客户和硬件工程团队对用于衡量质量的控制图现值意见不一致。项目经理与两个相关方开会并听取他们的意见,为了继续该项目,项目经理觉得使用双方意见的平均值。这属于下列哪一项的实例?

The customers and hardware engineering teams disagree on the present value of the control chart used to measure quality. The project manager meets with two stakeholders and listens to their opinions. In order to continue the project, the project manager believes that the average of b the present values provided by both stakeholders should be used. Which of the following examples does this belong to?

A:缓解

Ease

B:妥协

Compromise

C:解决问题

Problem solving

D:协作

Collaboration

正确答案:B

你的答案:A

解析:解析:PMBOK(6)P348-9.5.2.1人际关系与团队技能-冲突管理。“适用双方意见的平均值”,冲突的双方都做出了让步,属于妥协策略。

11、 [单选] 一位项目团队成员决定优先处理职能性工作,而不是项目中被分配的任务,项目经理如何确保项目和所有任务保持正轨?

How does the project manager ensure that the project and all tasks are on track when a project team member decides to prioritize functional work rather than tasks assigned in the project?

A:与该团队成员讨论动机,以更好地了解其工作量

Discuss motivations with the team member to better understand their workload

B:与该团队成员的职能经理会面,建议将项目任务纳入绩效考核

Have a meeting with the team member's functional manager and recommend that project tasks be included in the performance review

C:为该团队成员提供薪酬激励,以完成所分配的项目工作

Provide compensation incentives to the team member to complete the assigned project work

D:将该团队成员的不合格情况升级上报给其职能经理并请求其他资源

Escalate the team member's nonconformity to their functional manager and request additional resources

正确答案:B

你的答案:A

解析:解析:PMBOK(6)P350-影响力。 题干关键词“优先处理职能性工作”。 矩阵环境中,团队成员实际由职能经理管理,职能经理具有评估绩效等对员工的企业实际权力。项目经理没有实际管理权力,在题目中当资源在项目和行政工作存在冲突并且优先处理行政工作时,需要与其职能经理(资源相关方)协商来解决该问题。选项B,因在矩阵企业,对资源滞后处理项目工作,应与资源上级即职能经理协商,以促进完成项目工作,选项中“将项目工作纳入绩效考核”是合适的方案。 选项A错误,这不是工作量的问题,而是工作优先级的问题; 选项C错误,认可和奖励应该在团队成员有优良表现时施行; 选项D错误,采取的措施极端且消极。


16、 [单选] 首席执行官(CEO)发布说明,为了改进各项目之间的资源分配,明年将对当前的职能型组织结构作出调整,转而采用一种矩阵结构。当项目经理就该决定对当前项目的影响询问人力资源经理时,人力资源经理说现有项目的资源分配不会有变。根据这一回答,项目经理仍应审查以下哪项?

The Chief Executive Officer (CEO) announced that in order to improve the allocation of resources between projects, next year, the current functional organizational structure will be adjusted to adopt a matrix structure. When the project manager asked the human resource manager about the impact of the decision on the current project, the human resource manager said that the resource allocation of the existing project would not change. Based on this answer, which of the following should still be reviewed by the project manager?

A:风险管理计划

The risk management plan

B:责任分配矩阵(RAM)

The Responsibility Assignment Matrix (RAM)

C:资源分解结构

The resource breakdown structure

D:相关方参与计划

The Stakeholder engagement plan

正确答案:B

你的答案:C

解析:解析:PMBOK(6)P317-责任分配矩阵。 题干关键词“组织结构”、“资源分配不会有变”。 组织结构发生变化,项目经理关心的是工作包、活动与团队成员之间的关系是否发生了变化,因此要去审查RAM,选B。 选项AD属于知识领域定位错误; 选项C是按资源类别和类型进行展现的层级列表,不能说明人与工作包之间的关系。


17、 [单选] 某项目团队的成员来自多个职能部门,生于不同的时代,这导致他们各有不同的观点,思想和想法。项目经理应如何确保项目得到有效执行?

The members of a project team come from multiple functional departments and are born in different times, which leads to their different viewpoints, thoughts and ideas. How should the project manager ensure that the project is effectively executed?

A:帮助营造通力协作的决策环境,以便团队能够集体负责开展工作

Help create a collaborative decision-making environment where the team can work collectively

B:当团队成员之间存在分歧时,采用举手表决这种决策技术

Use a show of hands as the decision-making technique when there is a disagreement among team members

C:根据项目管理计划执行项目,并指派团队成员开展项目进度计划中的任务

Execute the project according to the project management plan and assign team members to carry out the tasks in the project schedule

D:咨询采购专业人士,以识别潜在问题和团队所需知识资本

Consult purchasing professionals to identify potential issues and the knowledge capital required by the team

正确答案:A

你的答案:D

解析:解析:PMBOK(6)P386-人际交往、政治意识。 题干关键词“来自于多个职能部门”、“生于不同的时代”“不同观点、思想、想法”。 本题采用排除法为宜。 选项B错误,题干不涉及决策; 选项C错误,题干与进度计划无关; 选项D错误,题干与采购无关,也不需要做有关知识资本的工作。 综合来看,A是最佳选项。

我的解释:举手表决太具体绝对,排除B。重点是人家关系,不是进度,排除C。D侧重知识资本,而A测试集体,A更合适。



18、 [单选] 某项目处于执行阶段,项目团队已经经过规范阶段,现处于成熟阶段。该团队出色地完成大量工作、项目经理希望奖励项目团队成员,但不确定这样做是否会影响团队的积极性。项目经理该做什么?

A project is in the execution phase, and the project team has passed the standardization phase and is now in the mature phase. The team has done a lot of work well, and the project manager wants to reward project team members, but is not sure whether doing so will affect the enthusiasm of the team. What should the project manager do?

A:在项目整个运行过程中给予团队认可,而非等项目完成才这样做

Give the team credit throughout the project, rather than waiting until the project is complete

B:意识到此时奖励项目团队可能会影响他们的工作动力,因此要等待适当的时机

Realize that rewarding the project team at this point may affect their motivation, so wait for the right moment

C:信任项目团队,并宣布将执行阶段结束后奖励最佳团队成员

Trust the project team and announce that the best team members will be rewarded at the closure phase of the execution phase

D:注意项目团队成员的表现,等进入收尾阶段后对相关团队成员给予奖励

Pay attention to the performance of project team members, and reward relevant team members after they enter the closure phase

正确答案:A

你的答案:C

解析:解析:PMAOK(6)P341-认可与奖励。 题干关键词“奖励”。 项目经理应该在整个项目生命周期中尽可能地给予表彰,而不是等到项目完成时,故选A。 选项BCD都不能及时地给予团队认可和表彰,错误。

19、 [单选] 管理某公司所有资源的职能部门经理决定将一位资源从一个项目调到另一个项目,但却未告知这两个项目经理。这两位项目经理应该做什么?

The functional manager who manages all of a company's resources decides to transfer one resource from one project to another without informing both project managers. What should the two project managers do?

A:请求该职能部门经理停止实施该变更

Request the functional manager to stop implementing the change

B:要求该职能部门经理提出变更请求

Request a change request from the functional manager

C:向项目管理办公室(PMO)上报该问题

Escalate the issue to the Project Management Office (PMO)

D:要求与该职能部门经理开会确定解决方案

Request a meeting with the functional manager to determine a solution

正确答案:D

你的答案:B

解析:解析:PMBOK(6)P356-9.6.2.2问题解决。 职能经理管理公司所有资源,在发生调换成员后,应实现和职能经理开会,正面沟通情况以解决问题。ABC都是直接解决,是不合适的,选项AB在没有和职能经理沟通的情况下要求职能经理停止变更或者提变更请求,并不能有效解决问题。其中C直接上报领导不合适,没有尽到项目经理的责任。

22、 [单选] 由于对项目方向的意见有冲突,一位团队成员辞职,项目经理应该做什么?

What should the project manager do if a team member resigns due to conflicting opinions about the direction of the project?

A:说服该资源留下来,因为他们对项目团队的成功非常宝贵

Convince the resource to stay, as they are invaluable to the success of the project team

B:立即通知项目发起人,并招募替代人员,以尽可能减少项目偏移

Notify the project sponsor immediately and recruit replacement personnel to minimize project offshoring

C:修改项目团队分配文件,并更新资源管理计划

Modify project team assignment documents and update resource management plan

D:更新资源管理计划,并修改项目进度计划以反映资源短缺

Update the resource management plan and modify the project schedule to reflect resource shortages

正确答案:C

你的答案:D

解析:解析:PMBOK(6)P358-资源管理计划。 本题用排除法为宜。 选项A错误,项目方向有意见的资源一般很难继续合作; 选项B错误,不是项目经理不能解决的问题,不需要上报发起人; 选项C正确; 选项D错误,修改项目进度计划需要走变更流程。

我的解释:CD都有更新资源管理计划,说D需要走变更流程而排除不合适。侧重点是资源和工作,修改进度计划,并不能推导出这个。


