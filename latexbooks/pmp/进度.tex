
\chapter{进度}

\section{常用概念及经验}

资源平衡的主要目的是是在满足交工日期的情况下,尽可能的平衡项目各资源间的负荷,到达资源的均匀和平衡使用。资源平衡的结果往往会导致项目的预期进度比根据关键路径计算的项目周期长。

资源优化包含了资源平衡。


\section{错题}

2、 [单选] 项目经理正在估算其组织之前未从事过的一类项目的成本.应该使用什么方法来估算成本?

The project manager is estimating the cost of a type of project that his organization has not undertaken before. What method should be used to estimate the cost?

A:参数估算

Parametric estimating

B:自下而上估算

Bottom-up estimating

C:自上而下估算

Top-down estimating

D:三点估算

Three-point estimating

正确答案:B

你的答案:D

解析:解析:PMBOK(6)P202,6.4.2.5-自下而上估算。“从未从事过的一类项目”,没有可以用以参照的项目,排除参数估算和类比估算。选项D,三点估算在估算的过程中兼顾风险因素,情景中没有强调类似的场景,排除。



4、 [单选] 在进行绩效评审后,项目经理识别到关键路径上五个工作包的其中一个可能会延迟两周时间。若要确保项目按进度执行,项目经理应该做什么?

In conducting the report, the project manager identified one of the five work packages on the critical path that could be delayed by two weeks. What should the project manager do to ensure that the project is on schedule?

A:通知相关方,并请求更多时间来处理该工作包

Notify the stakeholders and request more time to process the work package

B:请求额外的资源来处理受影响的工作包

Request additional resources to handle the affected work packages

C:分析该工作包以确定是否可以使用一些浮动时间

Analyze the work package to determine if some floating time is available

D:调整受影响工作包的进度管理计划

Adjust the schedule management plan for the affected work packages

正确答案:B

你的答案:C

解析:解析:PMDOK(6)P215,6.5.2.6-进度压缩。选项A,焦点是工期出现了延迟,再请求更多的时间,问题更加严重。选项C,“关键路径”,浮动时间为0。选项D,进度管理计划是针对整个项目的,而不是针对工作包,错误。


6、 [单选] 由于监管要求,项目的上线日期是固定的,项目经理在规划阶段运行了关键路径这种方法,结果显示该项目预计会比规定时间晚两个月完成。项目经理该做什么来纠正这种情况?

Due to regulatory requirements, the project's on-line date is fixed, and the project manager runs a critical path approach during the planning phase, which shows that the project is expected to be completed two months later than the required time. What should the project manager do to correct the project?

A:更新成本管理计划,为该项目分配更多资源

Update the cost management plan to allocate more resources to the current project

B:包含进度计划储备或紧急情况,并分析新的关键路径

Contain schedule reserves or contingencies, and analyzes new critical paths

C:进行反向排程,以重新定义所需的时间和资源

Reverse scheduling to redefine the time and resources required

D:通过剔除某些活动和缩短期限来缩减项目范围

Eliminate the elimination of certain activities and reduce deadlines to reduce the project scope

正确答案:C

你的答案:B

解析:解析:PMBOK(6)P209-6.5.2.1进度网络分析。“上线的日期是固定的”,所以要以上线日期进行分析,根据分析的结果调整项目管理计划,譬如缩小项目范围、追加资源投入等等。选项ABD在选项C之后。



8、 [单选] 项目经理必须确定项目的完成日期。虽然大多数活动都是已知的,但由于产品交付日期未知,某些工作包无法安排。项目经理应该使用什么来向相关方提供完工日期估算?

The project manager must determine the project finish date. Although most activities are known, some work packages cannot be scheduled because the product delivery date is unknown. What should the project manager use to provide the stakeholder with an estimate of the finish date?

A:滚动式规划

Rolling wave planning

B:蒙特卡洛模拟

Monte Carlo simulation

C:分解

Decomposition

D:计划评审技术(PERT)

Program Evaluation and Review Technique (PERT)

正确答案:B

你的答案:D

解析:解析:PMBOK(6)P213-6.5.2.4数据分析-模拟。这题易选错成A。“必须确定项目的完工日期”,采用滚动式规划就不合适了,排除AC选项。此时需要借助蒙特卡洛技术,对项目完成日期的可能分布进行预测。



16、 [单选] 负责跟踪项目文件的团队成员被指派到另一个项目,项目经理现在正在寻求项目文件,以便帮助制定进度计划。项目经理应使用哪些项目文件?

The team member responsible for tracking project documentation has been assigned to another project, and the project manager is now seeking project documentation. To help make schedule plans, what project documents should the project manager use?

A:经验教训登记册、里程碑列表、项目团队任务分配表

Lessons learned register, milestones list, project team task assignment list

B:里程碑列表、资源需求和项目章程

List of milestones, resource requirements, and project charter

C:范围基准、里程碑列表和活动清单

Scope baselines, milestone lists, and activity lists

D:活动清单、假设日志和进度基准

Activity lists, assumption logs, and schedule baselines

正确答案:C

你的答案:B

解析:解析:PMBOK(6)P207-6.5.1制定进度计划:输入。 题干关键词“帮助制定进度计划”。 选项A错误,项目团队任务分配表,即责任分配矩阵,具体到每个成员对每个工作包的职责。而项目团队派工单指的是每个成员在项目中的角色和职责,并未与每个工作包关联起来。责任分配矩阵不是6.5的输入; 选项B错误,项目章程不是6.5的输入; 选项C正确; 选项D错误,进度基准是6.5的输出,而非输入。



17、 [单选] 在规划某项目的进度计划时,项目经理注意到,项目章程中定义了一个不切实际的时间范围,而且某些里程碑预期会发生迟延,项目经理接下来该做什么?

When planning the schedule for a project, the project manager notices that the project charter defines an unrealistic timeframe and that certain milestones are expected to be delayed. What should the project manager do next?

A:遵守项目章程中规定的时间范围,因为这是一个业务需求

Comply with the time range specified in the project charter, as this is a business requirement

B:与有关的相关方开会解决这一偏差

Have a meeting with stakeholders to resolve this variance

C:调整时间范围,并寻求项目发起人批准

Adjust the time range and seek approval from the project sponsor

D:将该风险记录在风险日志中,并定期审查

Document this risk in the risk log and reviewed periodically

正确答案:B

你的答案:C

解析:解析:PMBOK(6)P207-6.5 需要在整个项目期间不断修订和维护项目进度模型,确保进度计划一直切实可行。 题干关键词“规划项目进度时”。 在进度基准被批准前,可以进行进度计划的更新,并无需遵循正式的变更流程。在这个时候,可以直接设法解决这一偏差,故选B; 选项A错误,项目章程不是6.5的输入; 选项C错误,项目经理无权修改项目章程中的时间范围; 选项D错误,这不是一个风险,而是一个确定会发生里程碑延期的问题。



21、 [单选] 一位经验丰富的项目经理加入团队,对一个陷入困境的项目执行健康检查。健康检查报告确定所有任务是同时执行的,并且从项目开始时就不断出现问题。若要提供更好的项目结果,原先的项目经理应该事先做什么?

An experienced project manager joins the team to execute a health check on a troubled project. The health check report confirmed that all tasks were performed at the same time, and issues continued to occur from the beginning of the project. To provide better project results, what should the former project manager have done beforehand?

A:审查详细的项目和阶段可交付成果

Review detailed project and phase deliverables

B:定义项目的制约因素和限制条件

Define project constraints and limitations

C:记录已识别的风险和假设条件

Document identified risks and assumptions

D:制定更详细的项目进度计划

Develop a more detailed project schedule

正确答案:D

你的答案:C

解析:解析:PMBOK(6)P205-6.5制定进度计划。 题干关键词“所有任务同时执行”。 由于所有任务同时执行并不断出现问题,可见原先的项目进度计划并没有创建合适的进度模型,故选D。 选项ABC在D之前,问“事先做什么”,选离问题时间点最近的那个选项。


60、 [单选] 项目经理正在为一个新项目制定项目进度计划,根据项目经理的经验,该项目需要两年才能完成。然而,公司总监要求该项目在18个月内完成。项目经理应该怎么做?

A project manager has prepared project schedule for a new project. According to the project manager’s experience, the project will take two years to complete. However, the company director requests that the project is to be completed in 18 months. What should the project manager do?

A:根据总监的要求修订项目进度计划

Revise the project schedule based on the director's request.

B:使用资源平衡,来平均项目团队成员的任务

Use resource leveling to even out project team member tasks.

C:减少项目范围以满足修订的项目进度计划

Reduce the project scope to meet the revised project schedule.

D:保持项目进度计划不变,但显示这种变更对其他制约因素的影响

Keep the project schedule unchanged, but show the impact this change would have on other constraints.

正确答案:A

你的答案:C

解析:解析:PMBOK(6)P244-6.5.1.2项目文件-假设日志。假设日志所记录的假设条件和制约因素可能造成影响项目进度的单个项目风险。公司总监要求项目在18个月内完成属于“制约因素”,基于这个制约因素来制定进度计划,选择A。 选项B:资源平衡是用来平衡资源负载,情景中的焦点是工期紧张,资源平衡会使项目周期更长。 选项C:必要时再通过缩减项目范围来满足进度计划要求。 选项D:保持项目进度计划不变,与工期要求不符。

64、 [单选] 在为一个有预算限制的项目生成状态报告时,项目经理发现该项目比进度计划落后一周。若要将项目拉回正轨,项目经理应该怎么做?

While generating the status report for a budget-constrained project, the project manager identifies that the project is one week behind schedule. What should the project manager do to bring the project back on track?

A:重新分配关键路径活动的团队成员

Reallocate team members on critical-path activities.

B:向项目发起人要求额外的时间

Request additional time from the project sponsor.

C:请求项目管理办公室(PMO)增加团队成员。

Ask the project management office (PMO) for additional team members.

D:执行资源优化

Perform resource optimization.

正确答案:A

你的答案:D

解析:解析:PMBOK(6)P215-6.5.2.6进度压缩-赶工。通过增加资源来压缩进度工期。选项C,团队成员是由职能部门经理提供的。选项B要求增加额外的时间,不解决项目进度滞后的问题。 选项D:资源优化包含了资源平衡。资源平衡会使项目进度延长,不可选。


68、 [单选] 项目经理通过将工作包分解到活动中去,识别并记录产生项目可交付成果的具体行动, 结果将产生哪一份文件?

A project manager identifies and documents the specific actions that produce project deliverables by breaking down work packages into activities. What document will be produced as a result?

A:资源分解结构(RBS)

Resource breakdown structure (RBS)

B:活动资源需求

Activity resource requirements

C:里程碑清单

Milestone list

D:活动持续时间估算

Activity duration estimates

正确答案:C

你的答案:D

解析:解析:PMBOK(6)P186-6.2.3.3里程碑清单。将工作包分解到活动,我们定位到定义活动子过程。定义活动输出里程碑清单。 选项A:是选项D的输入 选项B:是6.5制定进度计划的输入 选项D:在定义活动之后。


72、 [单选] 在施工现场可被验收之前,正为一个关键项目活动寻求获得当地主管部门的批准,项目经理在为项目制定进度计划时应该做什么?

A key project activity is to seek approval from local authorities before a construction site can be accepted .What should the project manager do when scheduling the project?

A:避免将该活动放在关键路径上

Avoid putting the activity on the critical path.

B:执行确定和整合依赖关系

Perform dependency determination and integration.

C:获得专家判断

Obtain expert judgment.

D:在风险登记册中添加一个新风险

Add a new risk to the risk register.

正确答案:B

你的答案:A

解析:解析:PMBOK(6)P191-6.3.2.2确定和整合依赖关系 。“当地主管部门的批准”属于外部依赖关系,外部依赖关系是项目活动与非项目活动之间的依赖关系,这些依赖关系往往不在项目团队的控制范围内。 选项A:要获得批准,即使不在关键路径上,浮动时间不足,同样会影响项目工期。 选项D:作为一个风险进行跟踪,但题干问的是“在制定进度计划时应该做什么”

86、 [单选] 在批准一位团队成员的紧急请假请求之前,项目经理需要确保计划的项目活动不会被延迟。项目经理首先应该审查哪份文件?

Before approving a team member's request for an urgent leave, a project manager needs to ensure that scheduled activities will not be delaye What should the project manager first review?

A:责任分配矩阵(RAM)

Responsibility assignment matrix(RAM)

B:资源日历

Resource calendars

C:资源分解结构(RBS)

Resource breakdown structure(RBS)

D:项目进度计划

Project schedule plan

正确答案:D

你的答案:A

解析:解析:PMBOK(6)P222-6.6控制进度。资源日历发生变化,要结合进度基准评估团队成员请假的影响。当然,评估不局限于进度基准。 选项ABC:注意,情景中强调了“确保计划的项目活动不会延迟”,所以要首先审查项目进度计划。


170、 [单选] 项目经理正在多个国家领导一项产品部署工作,最终部署时间非常紧迫,项目经理必须快速准备估算,项目经理应该使用什么估算技术?

A project manager is leading a product deployment effort in several countries. The final deployment has a tight schedule and the project manager must prepare an estimate quickly.what estimating technique should the project manager use?

A:类比估算

Analogous estimating

B:参数估算

Parametric estimating

C:自下而上估算

Bottom-up estimating

D:三点估算法

Three-point estimating

正确答案:A

你的答案:D

解析:解析:PMBOK(6)P200-6.4.2.2类比估算。类比估算是一种使用相似活动或项目的历史数据,来估算当前活动或项目的持续时间或成本的技术。相对于其他估算技术,类比估算通常成本较低、耗时较少。类比估算是4个选项中用时最少的。

我当初没有选择类比估算的原因是,题目中没有提到类似的项目。