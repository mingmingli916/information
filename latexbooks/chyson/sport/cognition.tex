
\chapter{认知}

\section{内在与外现}

内在和外现是相统一的。

我们尽可能的去做标准的动作,
因为标准的动作更能引发固定肌肉群的发力,
同时搭配着我们的意识集中在固定的肉群,
能更好的训练的肌肉群。

同样的,我们尽可能的去训练肌肉群的力量,
有力的肌肉群使我们的动作更标准。

内在和外现有递进着去训练。
比如跑步,
我们知道,跑步的时候要尽量前脚掌去着地,
为此我们要进行一些列固定肌肉群的训练,
如贴墙蹲来训练腿部的力量,
做核心训练来增强腹部的力量,
如果在肌肉力量不足的情况下,
仅仅去追求外现形态的标准,
是不切现实和效率不高的。

固定的标准姿势,激发固定的肌肉群,
有力的肌肉群,让我们掌握更标准更难的姿势。