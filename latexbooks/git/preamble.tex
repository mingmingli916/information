%%%%%%%%%% packages and setups %%%%%%%%%%
\usepackage{xspace}
\usepackage{url}
\usepackage{booktabs}           % provide beautiful table lines
\setlength{\abovetopsep}{3pt}  % 10pt space would be added below the caption and above the top line of the table.
\usepackage{tcolorbox}
\usepackage[a4paper, inner=1.5cm, outer=3cm, top=2cm, bottom=3cm, bindingoffset=1cm]{geometry}
\usepackage[onehalfspacing]{setspace}
\usepackage{fancyhdr}           % fancy header
\fancyhf{}                      % clear the headers and footers
% \leftmark is used by the book class to store the chapter title together with the chapter number. 
% LE stands for left-even and means that this chapter title will be put on the left side of the header on even-numbered pages.
\fancyhead[LE]{\leftmark}       
% \rightmark is used by the book class to store the section title together with its number. 
% RO stands for right-odd and means that this section heading shall be displayed on right side of the header on odd-numbered pages.
\fancyhead[RO]{\nouppercase{\rightmark}} 
\fancyfoot[LE,RO]{\thepage}     % \thepage prints the page number.
% All those commands are used to modify a page style provided by fancyhdr;
% this style is called fancy. We had to tell LaTeX to use this style and we did it through \pagestyle{fancy}.
\pagestyle{fancy}
\usepackage{paralist}           % can produce compact list
\usepackage{enumitem}           % user change enumerate, itemize and description labels
% \setlist sets properties valid for all types of lists .
% Here we specified nolistsep to achieve very compact lists analogous % to the compact paralist environment.
\setlist{nolistsep}
% \setitemize modifies properties of bulleted lists .
\setitemize[1]{label=$\varheartsuit$}
% \setenumerate sets properties valid for numbered lists .
\setenumerate[1]{label=\bfseries\arabic*,font=\sffamily{}}
\usepackage{array}
\setlength{\heavyrulewidth}{2pt} % set top bottom line width
\usepackage{multirow}
\usepackage[font=large,labelfont=bf,margin=1cm]{caption}
\usepackage[section]{placeins}  % can be used to prevent cross section floating
% This package introduces font expansion to tweak the justification and uses hanging punctuation
% to improve the optical appearance of the margins. This may reduce the need of hyphenation and improves the "grayness" of the output.
\usepackage{microtype}
\usepackage[utf8]{inputenc}     % utf8 symbols not just latin symbols
\usepackage[T1]{fontenc}        % TeX macros are translated into special characters. 
\usepackage{graphicx}           % used to insert figures
\graphicspath{{images/}}
\usepackage{pdfpages}           % can insert pdf into current docuement
\usepackage{varioref}           % intelligent referencing
\usepackage{xr}                 % external referencing
\usepackage{hyperref}           % hyperlink capability
\hypersetup{
  colorlinks=true,
  linkcolor=red,
  pdfauthor={Mingming Li},
  % pdftitle={},
  % pdfsubject={},
  % pdfkeywords={}
}
\usepackage{float}              % provide H option to floating environment
\usepackage{index}              % index
\makeindex{}
\usepackage{lmodern}            % font package
% automatically add bibliography, index, TOC, LOF, and LOT to the table of contents.
% tocbibind conficts with minitoc
\usepackage{tocbibind}          
\usepackage{titlesec}           % modify headings
% specify layout and font of the chapter headings
\titleformat{\chapter}[display]
{\normalfont\sffamily\Large\bfseries\centering}
{\chaptertitlename\ \thechapter}{0pt}{\Huge}
% section heading
\titleformat{\section}
{\normalfont\sffamily\large\bfseries\centering}
{\thesection}{1em}{}
% section heading
\titleformat{\subsection}
{\normalfont\sffamily\large\bfseries\centering}
{\thesubsection}{1em}{}
% adjust the chapter headings spacing
\titlespacing*{\chapter}{0pt}{30pt}{20pt}
\usepackage{color}
\usepackage{xcolor}
\definecolor{tex-comment-color}{rgb}{0.38,0.59,0.03}
\definecolor{tex-background-color}{rgb}{0.94,0.94,0.91}
\usepackage{textcomp}
\usepackage{listings}
\lstset{
  % language=elisp,
  basicstyle=\small\ttfamily,   % print whole listing small
  keywordstyle=\color{blue},     
  identifierstyle=,                       % nothing happens
  commentstyle=\color{tex-comment-color}, % self defined comment color
  stringstyle=\ttfamily,        % typewriter type for strings
  showstringspaces=false,       % no special string spaces
  numbers=left,                 % show line number on left
  numberstyle=\tiny,            % tiny line number
  stepnumber=1,                 % every 1 step to print a number
  numbersep=10pt,               % number separate from the code
  % frame=trBL,
  % frameround=fttt,
  backgroundcolor=\color{tex-background-color},
  breaklines=true,
  upquote=true,
  literate={’}{\textquoteright{}}1 {‘}{\textquoteleft{}}1  
}
\usepackage{amsfonts}           % math fonts
\usepackage{bbm}                % math fonts
\usepackage{dsfont}             % math fonts
\usepackage{eufrak}             % math fonts
\usepackage{amsmath}            % multiline equations
% Options: Sonny, Lenny, Glenn, Conny, Rejne, Bjarne, Bjornstrup
\usepackage[Lenny]{fncychap}
\usepackage{longtable}          % make multi-page table
\usepackage[indent, skip=10pt plus1pt]{parskip}
\usepackage{indentfirst}        % indent paragraph
\setlength{\parindent}{2em}
\usepackage{xsavebox}           % lrbox, savebox et al environments
\newsavebox{\lstbox}
\usepackage{tablefootnote}      % show footnote in table
\usepackage{txfonts}            % provides heart symbol
\usepackage{pmboxdraw}          % provide special symbol support in listings
\usepackage{fontawesome}        % this package provides some social icons


%%%%%%%%%% user defined style %%%%%%%%%%
\renewcommand{\footnoterule}{\noindent\smash{\rule[3pt]{\textwidth}{1pt}}}
% renamed the figures and the list heading 
\renewcommand{\figurename}{Figures}
\renewcommand{\listfigurename}{List of Figures}
% renamed the table and the list heading 
\renewcommand{\tablename}{Tables}
\renewcommand{\listtablename}{List of Tables}



%%%%%%%%%% user defined commands %%%%%%%%%%
% color: blue brown cyan gray green lime magenta olive orange pink purple red teal violet yellow
\newcommand{\keyword}[1]{{\color{cyan}\textbf{#1}}}
\newcommand{\argument}[1]{{\color{olive}\texttt{#1}}}
\newcommand{\warningword}[1]{{\color{orange}\textbf{#1}}}
\newcommand{\funcword}[1]{{\color{brown}\textbf{#1}}}

\newcommand{\head}[1]{\textnormal{\textbf{#1}}}
\newcommand{\normal}[1]{\multicolumn{1}{l}{#1}}

\newcommand{\auctex}{AUC\TeX\xspace}

\newcommand{\myfigure}[2][0.6]{\includegraphics[width=#1\textwidth]{#2}}

%%% Local Variables:
%%% mode: latex
%%% Tex-master: "git"
%%% End: