
\chapter{LEAD}


LEAD工具和止念法基于这样的观点:
促使一个人做出反应和产生感觉的是这个人对事件的看法而不是事件本身。


L=Listen,倾听自己的逆境反应:
这是高逆商反应还是低逆商反应?
在哪个纬度得分最高或者最低?

E=Explore,探究自己对结果的担当:
我应该对结果的哪些部分担起责任?
我不应该对哪些部分担责?

A=Analyze,分析证据:
有什么证据可以表明我无法掌握?
有什么证据可以表明困境一定会蔓延到生活的其他方面?
有什么证据可以表明此次困境会持续过长时间?

D=Do,做点事情:
我还需要什么信息?
我可以做什么来获得对形势的一点点掌控感?
我可以用什么来限制困境的影响范围?
我可以做什么来限制当前困境的持续时间?


