
\chapter{前言}


语言的主要作用是交流。如果我们独自生活在这个世界上,我们其实是不需要语
言的,我们对世界的认知以六感(图像,声音,味道,气味,触感和综合历史五
感记忆和现在场的影响形成的对未来的感觉)的形式存储在我们的头脑中。

语言不是帮助我们认知世界的,但语言的发展有利于大脑的开发,人是因为其社
会属性才称之为人的,没有社会属性,更确切的称之为动物。


因为语言的主要作用是交流,用于交流的,其实是听说,读写是为了扩大交流的
范围。历史上也是先出现的语言,后出现的文字,我们小时候也是,我们先学会
的说话,之后学会的写字。




法语的学习分为:听说读写。


\section{听}


我们听到一段声音,我们尝试去理解这段声音,这个过程类似决策树。
我们要做的第一个决策是声音的划分,我们首先将声音划分为多个音节,这是一个有反馈的划分,
我们在头脑中寻找和第一个音节相对应的词语,当我们没有找到的情况下,我们
尝试增加一个或者多个,然后再去寻找对应的词语,对应的词语可能有很多,我
们需要从中筛选出一个选择,帮助做出这个决策是语境(表情,下文,语气等
等),所以听力的基础是声音和词语的联系,需要根据声音筛选词语,词汇量的
多寡影响决策树分支的完备性。

一段声音中,每个音节的权重并不一定是相同的,一般权重大的是名词,动词,
代词和形容词所对应的声音,它们承载主体的信息。


我们去理解一段话,刚开始是经过思考的,有意识的,经过不断的重复和联系,
最终会建立不经过思考的,无意识的声音和词语的链接。

听的训练:强化声音和词语的链接,这个过程是通过不断重复实现的。


\section{说}

说一定程度上是身体的记忆,记忆的部分为口,舌,肺,腹。
当我们想要表达某个意思,我们根据头脑中存储的词语,去链接对应的发音,
结果我们的身体发出对应的声音。

我们首先需要身体记忆每个音标的发音,并形成身体记忆。下一步是记忆音标组
合成音节,最后是音节之间的过渡。我们说的时候磕磕巴巴,有一下原因:
\begin{itemize}
\item 我们不知道某个词语。
\item 我们知道某个词语,但词语和声音的链接不牢固,需要话时间进行链接。
\item 我们的链接很牢固,比如一听就知道,但说的不顺,因为未形成身体记忆,
  需要化时间去调节肌肉来发出对应的声音。
\end{itemize}

说的训练:强化身体记忆和词语的链接,这个过程是通过不同重复实现的。



\section{读}

读是看到一个符号,知道它的意思。读的基础是符号对词语意思的链接。

汉字是二维空间上的符号,是一种象形文字,法语(英语)是一维空间上的符号。记忆汉字可以
用偏旁部首来帮助记忆,然后因为一维的缘故,记忆法语单词就抽象了很多,
少了想象的空间。有两个解决途径,1. 人为刻意加入想象力。 2. 使用前缀,
词根,后缀和组合次法,类比汉字中的偏旁部首,只是少了形象部分。



\section{写}

写和读类似,也是符号和意思的链接,但这个比读困难,因为读是在现有符号下,
对符号和意思链接的回忆,而读则是没有任何帮助下,对符号和意思链接的回忆。





\section{语法}

在有链接的基础后,思维的表达涉及到表达的习惯,每个国家的人思维表达的习
惯都不同,进而形成了不同的语法,学习语法其实是学习语言的思维方式。







%%% Local Variables:
%%% mode: latex
%%% TeX-master: "french"
%%% End:
