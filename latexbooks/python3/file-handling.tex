
\chapter{File handling}

There are several file formats to choose when you are doing serialization and deserialization, for example:
\begin{itemize}
\item binary 
\item test
\item XML
\end{itemize}



Which is the best file format?

The question is too context-dependent to have a single definitive answer,
especially since there are pros and cons for each format and for each way of handling them.



Binary formats are usually very \keyword{fast} to save and load and they can be very \keyword{compact}.
Binary data doesn’t need parsing since each data type is stored using its \keyword{natural representation}.
Binary data is \keyword{not human readable or editable}, and without knowing the format in detail
it is not possible to create separate tools to work with binary data.


Text formats are \keyword{human readable and editable}.
Text formats can be \keyword{tricky to parse} and
it is not always easy to give good error messages if a text file’s format is broken.



XML formats are \keyword{human readable and editable}.
XML formats can be processed using separate tools.
Parsing XML is straightforward and some parsers have good error reporting.
XML parsers can be slow, so reading very large XML files can take a lot more time than reading an equivalent binary or text file.
XML includes metadata and this can make XML more portable than text files.



Some programs use an XML file format for all the data they handle, whereas others use XML as a convenient import/export format.
The ability to import and export XML is useful and is always worth considering even if a program’s main format is a text or binary format.


