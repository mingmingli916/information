
\chapter{Debugging, testing, and profiling}

Writing programs is a mixture of art, craft, and science, and because it is done by \keyword{humans}, mistakes are made.


Mistackes fall into sevaral categories:
\begin{description}
\item[syntax error] The program can not run.
\item[logical error] The program runs, but some aspect of its behavior is not what we intended or expected.
\item[poor performance] This is almost always due to a poor choice of algorithm or data structure or both.
\end{description}

\section{Debugging}


\subsection{Dealing with syntax errors}

\begin{lstlisting}
for i in range(10)
    print(1)

#   File "/Users/mike/PycharmProjects/python3/c9_debugging_testing_profiling/t.py", line 10
#     for i in range(10)
#                      ^
# SyntaxError: invalid syntax  
\end{lstlisting}

If we try to run a program that has a syntax error, Python will stop execution and print the filename, line number, and offending line, with a caret(\^{}) underneath indicating exactly where the error was detected.


\begin{lstlisting}
try:
    if True:
        print(''
except Exception as err:
    print(err)
    
#   File "/Users/mike/PycharmProjects/python3/c9_debugging_testing_profiling/t.py", line 21
#     except Exception as err:
#     ^
# SyntaxError: invalid syntax  
\end{lstlisting}

There is no syntax error in the line indicated, so both the line number and the caret’s position are wrong.
We have omite a parenthese, but Python didn't realize this until it reach the \verb|except| keyword on the following line.




\subsection{Dealing with runtime errors}

\begin{lstlisting}
def div():
    1 / 0


if __name__ == '__main__':
    div()

# Traceback (most recent call last):
#   File "/Users/mike/PycharmProjects/python3/c9_debugging_testing_profiling/e3.py", line 17, in <module>
#     div()
#   File "/Users/mike/PycharmProjects/python3/c9_debugging_testing_profiling/e3.py", line 13, in div
#     1 / 0
# ZeroDivisionError: division by zero  
\end{lstlisting}

If an unhandled exception occurs at runtime, Python will stop executing our program and print a traceback.
Tracebacks should be read from their last line back toward their first line.
The last line specifies the unhandled exception that occurred.
Above the line, the filename, line number, and function name, followed by the line that caused the exception, are shown.


\subsection{Scientific debugging}

To be able to kill a bug we must be able to do the following.
\begin{enumerate}
\item Reproduce the bug.
\item Locate the bug.
\item Fix the bug.
\item Test the fix.
\end{enumerate}


Reproducing the bug is sometimes easy -- it always occurs on every run; and sometimes hard -- it occurs intermittently.
In either case we should try to reduce the bug’s dependencies, that is, find the smallest input and the least amount of processing that can still produce the bug.


The scientific method of finding and fixing the bug has three steps:
\begin{enumerate}
\item Think up an explanation -- a hypothesis -- that reasonably accounts for the bug.
\item Create an experiment to test the hypothesis.
\item Run the experiment.
\end{enumerate}


\section{Unit testing}

A key point of TDD (Test Driven Development) is that when we want to add a feature, we \keyword{frist} write a test for it.


Python's standard library provides two unit testing modules, \verb|doctest| and \verb|unittest|.
Creating doctests is straightforward:
We write the tests in the module, function, class, and methods’ docstrings, and for modules, we simply add three lines at the end of the module:
\begin{lstlisting}
if __name__ == "__main__":
    import doctest
    doctest.testmod()
\end{lstlisting}

To exercise the program's doctests there there are two approaches:
\begin{enumerate}
\item Import the \verb|doctest| module and then run the program -- for example, at the console, \verb|python -m doctest yourprogram.py|.
\item Create a separate test program using the \verb|unittest| module.
\end{enumerate}




\section{Profiling}

There are some programming habits that are good for performance:
\begin{itemize}
\item Prefer tuples to lists when read-only sequence is needed.
\item Use generators rather than creating large tuples or lists to iterate over.
\item Use Python’s built-in data structures -- \verb|dicts|, \verb|lists|, and \verb|tuples| -- rather than custom data structures implemented in Python, since the built-in ones are all very highly optimized.
\item When creating large strings out of lots of small strings, instead of concatenating the small strings, accumulate them all in a list, and join the list of strings into a single string at the end.
\item If an object (including a function or method) is accessed a large number of times using attribute access (e.g., when accessing a function in a module), or from a data structure, it may be better to create and use a local variable that refers to the object to provide faster access.
\end{itemize}




The \verb|cProfile| module (or the \verb|profile| module) can be sued to compare the performance of functions and methods.
And it also shows precisely what is being called and how long each call takes.


\begin{lstlisting}
import cProfile
import math


def log(x, y):
    return math.log(x, y)


code = """
for i in range(10000):
    log(10, 2)
"""
cProfile.run(code)
\end{lstlisting}


\begin{verbatim}
         20003 function calls in 0.006 seconds

   Ordered by: standard name

   ncalls  tottime  percall  cumtime  percall filename:lineno(function)
        1    0.002    0.002    0.006    0.006 <string>:2(<module>)
    10000    0.002    0.000    0.004    0.000 cprofile_.py:5(log)
        1    0.000    0.000    0.006    0.006 {built-in method builtins.exec}
    10000    0.002    0.000    0.002    0.000 {built-in method math.log}
        1    0.000    0.000    0.000    0.000 {method 'disable' of '_lsprof.Profiler' objects}
\end{verbatim}

The \verb|ncalls| (``number of calls'') column lists the number of calls to the specified function.
The \verb|tottime| (``total time'') column lists the total time spent in the function, but excluding time spent inside functions called by the function.
The first \verb|percall| column lists the average time of each all to the function (\verb|tottime| // \verb|ncalls|).
The \verb|cumtime| (``cumulative time'') column lists the time spent in the function and includes the time spent inside functions called by the function.
The second \verb|percall| column lists the average time of each call to the function, including functions called by it.




The \verb|cProfile| module allows us to profile code without instrumenting it.
The command line to use is \verb|python -m cProfile program_or_module.py|.

\verb|MyModule.py|:
\begin{lstlisting}
import math


def log(x, y):
    return math.log(x, y)


for i in range(10000):
    log(10, 2)
  
\end{lstlisting}

We can save the complement profile data and analyze it using the \verb|pstats| module.

\begin{lstlisting}

(base) mike@Mikes-MacBook-Pro c9_debugging_testing_profiling % python -m cProfile -o profile.dat MyModule.py
(base) mike@Mikes-MacBook-Pro c9_debugging_testing_profiling % python -m pstats                             
Welcome to the profile statistics browser.
% read profile.dat
profile.dat% callers log
   Random listing order was used
   List reduced from 65 to 2 due to restriction <'log'>

Function                    was called by...
                                ncalls  tottime  cumtime
{built-in method math.log}  <-   10000    0.001    0.001  MyModule.py:4(log)
MyModule.py:4(log)          <-   10000    0.002    0.003  MyModule.py:1(<module>)


profile.dat% callees log
   Random listing order was used
   List reduced from 65 to 2 due to restriction <'log'>

Function                    called...
                                ncalls  tottime  cumtime
{built-in method math.log}  -> 
MyModule.py:4(log)          ->   10000    0.001    0.001  {built-in method math.log}


profile.dat% quit
Goodbye.
  
\end{lstlisting}

