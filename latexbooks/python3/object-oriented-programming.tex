
\chapter{Object-oriented programming}

\section{Costom classes}

\subsection{Attributes and methods}


\begin{tcolorbox}
\begin{verbatim}
class className:
    suite

class className(base_classes):
    suite
\end{verbatim}
\end{tcolorbox}

Just like \verb|def| statements, \verb|class| is a statement, so we can create classes dynamically if we want to.
Class instances are created by calling the class with any necessary arguments.


\begin{lstlisting}
#!/usr/bin/env python3
# Copyright (c) 2021-02

import math


class Point:
    def __init__(self, x=0, y=0):
        """
        A 2D cartesian coordinate
        :param x:
        :param y:
        """
        self.x = x
        self.y = y

    def distance_from_origin(self):
        return math.hypot(self.x, self.y)

    def __eq__(self, other):
        return self.x == other.x and self.y == other.y

    def __repr__(self):
        return f'Point({self.x!r}, {self.y!r})'

    def __str__(self):
        return f'({self.x!r}, {self.y!r})'

\end{lstlisting}


Python automatically supplies the first argument in method calls -- it is an object reference to the object itself.
We must include this argument in the parameter list, and by convention the parameter is called \verb|self|.
All object attributes (data and method attributes) must be qualified by \verb|self|.



To create an object, two steps are necessary:
\begin{enumerate}
\item a raw or uninitialized object must be created (\verb|__new__()|)
\item the object must be initialized, ready for use (\verb|__init__()|)
\end{enumerate}



If we call a method on an object and the object's class does not have an implementation of that method,
Python will automatically go through the object's base classes, and their base classes, and so on, until it finds the method --
and if the method is not found an \verb|AtributeError| exception is raised.



Calling \verb|super().__init__()| is to call the base class's \verb|__init__()| method.
For classes that directly inherit \verb|object| there is no need to do this and we call base class methods only when necessary --
for example, when creating classes that are designed to be subclassed, or
when creating classes that don't directly inherit \verb|object|.



If we want to avoid inappropriate comparisons, we can this:
\begin{lstlisting}
    def __eq__(self, other):
        if not isinstance(other, Point):
            return NotImplemented
        return self.x == other.x and self.y == other.y
\end{lstlisting}
In this case, if \verb|NotImplemented| is returned, Python will then try calling \verb|other.__eq__(self)| to see
whether the \verb|other| type supports the comparion with the \verb|Point| type, and
if there is no such method or if that method alse returns \verb|NotImplemented|, Python will give up and
raise a \verb|TypeError| exception.
Only the following methods may return \verb|NotImplemented|:
\begin{itemize}
\item \verb|__lt__(self, other)|
\item \verb|__le__(self, other)|
\item \verb|__eq__(self, other)|
\item \verb|__ne__(self, other)|
\item \verb|__ge__(self, other)|
\item \verb|__gt__(self, other)|
\end{itemize}




\begin{tcolorbox}
  Poweful \verb|eval()|: (\verb|eval(expression)|)
  \begin{lstlisting}
    p = Shape.Point(3, 9)
    repr(p)                                              # returns: 'Point(3, 9)'
    q = eval(p.__module__ + "." + repr(p))        
    repr(q)                                              # returns: 'Point(3, 9)'
  \end{lstlisting}


  \begin{lstlisting}
    a0 = 0
    a1 = 1
    a2 = 2
    a3 = 3
    
    for i in range(4):
        print(eval(f'a{i} * {i}'))
  \end{lstlisting}

  A more powerful function is \verb|exec()|,
  it can accept python code not only python expression.
  \begin{lstlisting}
    for i in range(3):
        exec(f'a{i} = {i}')
        exec(f'print(a{i})')
    exec('me = "Mike Chyson"')
    print(me)

    exec('''
    def say_hello():
        print('hello')
    ''')
    say_hello()
  \end{lstlisting}

\end{tcolorbox}



\subsection{Inheritance and plymorphism}

\begin{lstlisting}
class Circle(Point):
    def __init__(self, radius, x=0, y=0):
        super().__init__(x, y)
        self.radius = radius

    def edge_distance_from_origin(self):
        return abs(self.distance_from_origin() - self.radius)

    def area(self):
        return math.pi * (self.radius ** 2)

    def circumference(self):
        return 2 * math.pi * self.radius

    def __eq__(self, other):
        return self.radius == other.radius and super().__eq__(other)

    def __repr__(self):
        return f'Circle({self.radius!r}, {self.x!r}, {self.y!r})'

    def __str__(self):
        return repr(self)
\end{lstlisting}



\subsection{Using properties to control attribute access}

\begin{tcolorbox}
A property is an item of object data that is accessed like an instance variable
but where the accesses are handled methods behind the scenes.
\end{tcolorbox}


\begin{lstlisting}
class Circle(Point):
    def __init__(self, radius, x=0, y=0):
        super().__init__(x, y)
        self.radius = radius

    @property
    def edge_distance_from_origin(self):
        return abs(self.distance_from_origin - self.radius)

    @property
    def area(self):
        return math.pi * (self.radius ** 2)

    @property
    def circumference(self):
        return 2 * math.pi * self.radius

    @roperty
    def radius(self):
        return self.__radius

    @property.setter
    def radius(self, radius):
        assert radius > 0, 'radius must be nonzero and non-negative'
        self.__radius = radius  
\end{lstlisting}

\begin{tcolorbox}
  A decorator is a function that takes a function or method as its argument and returns a ``decorated'' version,
  that is, a version of the function or method that is modified in some way.
  A decorator is indicated by preceding its name with an at symbol (@). 
\end{tcolorbox}



The \verb|property()| decorator function is built-in and takes up to four arguments:
\begin{itemize}
\item a getter function
\item a setter function 
\item a deleter function
\item a docstring
\end{itemize}

The effect of using \verb|@property| is the same as calling the \verb|property()| function with just one argument, the getter function.
We could have created the area property like this:
\begin{lstlisting}
  def area(self):
      return math.pi * (self.radius ** 2)
  area = property(area)
\end{lstlisting}
We rarely use this syntax, since using a decorator is shorter and clearer.





To turn an attribute into a readable/writable property we must create a private attribute where the data is actually held and supply getter and setter methods.
\begin{lstlisting}
    @property
    def radius(self):
        return self.__radius

    @property.setter
    def radius(self, radius):
        assert radius > 0, 'radius must be nonzero and non-negative'
        self.__radius = radius   
\end{lstlisting}


Every property that is created has a \verb|getter|, \verb|setter|, and \verb|deleter| attribute,
so once the radius property is created using \verb|@property|, the \verb|radius.getter|, \verb|radius.setter|, and
\verb|radius.deleter| attributes become available.
The \verb|radius.getter| is set to the getter method by the \verb|@property| decorator.
The other two are set up by Python so that they do nothing (so the attribute cannot be written to or deleted),
unless they are used as decorators, in which case they in effect replace themselves with the method they are used to decorate.


The \verb|Circle|'s initializer, \verb|Circle.__init__()|, includes the statement \verb|self.radius = radius|;
this will call the \verb|radius| properity's setter.



