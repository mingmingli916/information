\documentclass{ctexbook}
\usepackage{xeCJK}              
\usepackage[english]{babel}
\usepackage[utf8]{inputenc}
\usepackage[T1]{fontenc}
\usepackage{lmodern}
\usepackage{microtype}
\usepackage{natbib}
%% \usepackage{tocbibind}          
\usepackage{amsmath}
\usepackage{amsthm}
\usepackage[colorlinks=true,linkcolor=red]{hyperref}

\usepackage{color,xcolor}

\usepackage{indentfirst}
\setlength{\parindent}{2em}

\usepackage[onehalfspacing]{setspace}

\usepackage{hyperref}

\usepackage{pdfpages}
\usepackage{graphicx}

\usepackage{booktabs}
\usepackage{tcolorbox}

%% user defined command
\newcommand{\keyword}[1]{\textbf{#1}}
\newcommand{\lcmd}[1]{\texttt{#1}}
\newcommand{\head}[1]{\textnormal{\textbf{#1}}}
\usepackage{datetime}
\renewcommand{\today}{\number\year 年 \number\month 月 \number\day 日}

\usepackage{float}
\hypersetup{
  pdfauthor={\pdfauthor},   
  pdftitle={\pdftitle},     
  pdfsubject={\pdfsubject}, 
  pdfkeywords={\pdfkeywords}
}
% \includeonly{}
\begin{document}
\begin{titlepage}
  \raggedleft
  {\Large 作者:\\\author[1in]}    
  {\large 此书关于\\}
  {\Huge\scshape \bookname\\[.2in]}      
  {\large \bookoneword\\}    
  \vfill
  {\itshape 首次编辑日期:\timeandcompany \\}
  {\itshape 最近编辑日期:\today}
\end{titlepage}
\frontmatter
\chapter*{Dedication}

我学习项目管理的目的有三个:
\begin{itemize}
\item 考取PMP证书。
\item 开阔自己的视野,从更广的角度去思考问题。
\item 指导自己的生活,将项目管理的知识应用于自己的生活中。
\end{itemize}

\tableofcontents
\mainmatter

\part{机遇}

\chapter{马太效应:英超球员的优势积累}

个性并非个人成功的决定因素。成功人士并非白手起价,他们其实是以某种形式获得了家庭的荫庇和支持。那些最终变得卓尔不群的人看似完全靠个人的奋斗,其实不然。事实上,他们一直得益于某些隐藏的先天优势,或是非凡的机缘,抑或某一文化的特殊优势 -- 这使他们学的快,干的多,可以以普通人难以企及的方式认识世界。

马太效应: 凡是有的,还要加给他,叫他有余;没有的,连他所有的,也要夺过来。

在社会学领域,所谓的成功就是“优势积累”的结果。


\chapter{10000小事定律:生于1955年的乔布斯和比尔 盖茨}

到底是否存在与生俱来的天赋?

很明显,答案是肯定的。但问题是,随着研究是深入,心理学家们发现,天赋的作用其实很小,而后天努力的作用其实很大。


埃里克森的研究中最引人注目的结论是:
第一,根本没有“与生俱来的天才” -- 花比别人少的时间就能达到比别人高的成就;
第二,也不存在“劳苦命” -- 一个人的努力程度比别人高却无法比别人更优秀。


一个人在学习的过程中,要完美掌握某项复杂技能,就要一遍又一遍地艰苦练习,而练习的时常必须达到一个最小临界值。事实上,研究者就练习时长给出了一个神奇的临界值:10000小时。

完全出自莫扎特之手,并被奉为他的第一部经典作品的,是第九号钢琴协奏曲,然而,这部作品是他在21岁时创作的。那时候莫扎特作曲已有10个年头了。
Harold C. Schonberg认为莫扎特实际上是“大器晚成”,因为他经历了整整20年的作曲生涯,才创作出最伟大的作品。


使他们(比尔乔伊,比尔盖茨,甲壳虫乐队)如此出色的不是他们非凡的才能,而是他们非凡的机遇。

那些获得特殊机遇眷顾的人总能努力工作,并胜任使命;与机遇相伴的人总能取得非凡寻常的成就。他们的成功并不仅仅是自己努力的成果,更是独特的成长环境促成的结果。


\chapter{智商与机遇:“特曼人”的谬误}

智商与成功只在一定程度上相互关联,一旦某人的智商超过120分,此时更高的智商并不意味着会同比转化成更多的现实优势。

一个篮球运动员只需要身高足够高,超过职业队的身高门槛就可以了。同样,智商也是如此,智商门槛同样存在。

仅靠智商很难区分两个聪明孩子。


既然智力因素仅在某些程度上发挥作用,那么超越这以程度以后,当智力发挥不了作用的时候,另外一些因素就开始发挥作用了。这又有点像打篮球:一旦你的身高足够高,人们就开始关注你的速度、球感、灵活性,以及球技和投篮的准确性了。

\chapter{社交与家庭:天才懒根的忧伤}

有这样一种特殊技能,它能让你说服教授把课从上午调到下午,能让你在辩解一宗谋杀案时振振有词。心理学家Robert Sternberg称之为“实践智力”。Sternberg的实践智力包括“知道该向什么人说什么话,该在什么时候说,怎样说才能达到最好的效果”。这种技能更像是一种程序化概念:知道如何做某事,而不需要为什么知道,也不需要解释为什么。这种技能本质上是一种实践能力:这不是关于如何辩解的知识,而是帮你正确了解形势从而获得你想要得到的东西的知识。

“权利”意识:他们认为自己有权提出自己的特殊要求,有权参与制度互动。他们在各种情景中更加自如,愿意分享信息,并希望赢得别人的关注。。。通过互动来满足自己的偏好。
“权利”意识能使他在未来更好地适应社会。

一直以来,兰根的努力都依靠个人奋斗,然而没有哪个摇滚明星、职业球员或是软件业亿万富翁是仅依靠自己的努力就能最终获得成功的。

\chapter{最佳时代:乔 弗洛姆的律师生涯}

成功总有其原因。成功人士不可能独自走向成功,他们总是特定地点和特定环境的产物。

现实的确相当不公。然而,对于时常身处逆境的“异类”,这些不利因素却常常最终成为他们的机遇之源。

弗洛姆不是克服了逆境,而是原先的逆境忽然之间变成了机遇。

自主性、复合性、付出和回报的关联性,以上三点是任何一份称得上令人满意的工作都应该具备的属性。能赚多少钱并不是使我们快乐的最终源泉,使我们快乐的源泉是这份工作在多大程度上能让我们实现自我。只有艰苦的从事没有意义的工作才称得上是艰苦工作。

成功不是随机事件。成功是一系列可预知的,强而有力优势环境和机遇带来的结果。


\part{文化传承}

\chapter{文化差异:小镇哈伦}

一种被社会学家称为”荣誉文化“的东西使暴力蔓延到这里的每一个角落。

”荣誉文化“植根于高地或者富庶地区的边缘地带。这种解释的逻辑是,当人们居住在多岩石的山坡地带时,由于那里的土地很难耕种,人们大多会依靠放牧生活。他们一直生活在牲口被偷,整个生活被毁的恐惧之中。所以他们养成了好斗的性情:他们必须通过自己的言行表明自己不是弱者,要对危及他们的哪怕是最轻微的挑战予以最坚决的反击 -- 这就是”荣誉文化“的含义。

文化直接决定了我们看待世界的方式和行为模式,其作用如此巨大,以至于没有它,我们将无法认识世界。


\chapter{权利距离指数:韩国飞机失事率}

缓和性语气:低调处理所说内容以取悦听众。

\begin{tcolorbox}
六种不同程度的缓和性:
1. 命令:“右转30度。”这是对某事情特别关切时最直接、最清晰的表达,缓和性为零。
2. 责任性陈述:“我认为我们需要向右调整航向。”这里使用了“我们”,要求的精确度大为降低,语气稍微缓和。
3. 建议:“让我们躲过坏天气。”这是不明确陈述,意思是“我们一起面对”。
4. 询问:“你准备往哪边调整航向?”这种句式比建议更弱化,因为说话者表明飞机不在自己的掌控之下。
5. 偏好:“我觉得我们向左或右偏向转比较好。”
6. 暗示:“前方25英里处天气可能不太好。”这种语气是所有语气中最缓和的一种。
\end{tcolorbox}



“暗示”是最难解读和最容易被拒绝的对话方式。


权利距离是指人们对待比自己更高等级阶层的态度,特别是指对权威的重视和尊重的程度。

在权利距离指数底的国家,掌权者会因手中的权利而感觉不好意思,并试图淡化其重要性。

我们取得事业成功的能力与我们的文化背景紧密相连。

飞机失事的三个典型前提:轻微的技术故障、坏天气、疲惫的飞行员。

西方人的沟通方式在语言学上被称为“以说话这为导向的”,意思是,说话者有责任将意思清晰明白地表达出来。

韩国文化中的沟通方式,和亚洲许多国家一样,是“以聆听者为导向“的。也就是说,能否搞清除说话的意思,取决于聆听者自己。


他们对待工作非常积极,愿意主动承担责任,不再需要别人指示他们做什么。

在通向职业成功的道路上,除了个人努力,我们所处的文化、历史和外部环境对我们是否成功也具有决定性意义。

我们每个人都处于某种文化环境中,而文化环境又是集优点、弱点、素质、倾向等各种属性于一身的综合体,为什么人们通常不愿正视这一事实?我们不应割裂我们的行为与成长环境之间的关系。

\chapter{内在优势:亚洲人精神}

人类大脑存储数字的记忆周期是2秒钟,也就是说,人们很容易记住2秒钟内能读完的东西。

英语中的数字系统是高度不规则的,导致说英语的人对数学的感念不那么清晰明了。


如果让一个7岁说英语的孩子心算37(thirty-seven)加22(twenty-two),他首先必须把文字转化成阿拉伯数字(37+22),只有这样他才方便计算:2+7=9,30+20=50,结果就是59.
如果让同龄的亚洲孩子计算37加22,对于他们来说,计算公式就嵌在语句中,因此他们不用转换就能得出答案:59。

亚洲语言的数字系统含义清晰。
西方学生需要死记硬背的定西,东方学生对其却有清晰的概念。


在数学学习方面,亚洲人具有某种“内才优势”,这种“内在优势”是一种不同寻常的优势。


态度比能力更重要。只有你愿意,你就能驾驭数学。

一年忙到头,吃穿不用愁。


\chapter{扭转风气:玛丽塔之幸}

”异类“是那些获得特殊机遇之人,是那些耐心等待之人,当机遇来到时就当仁不让地把握住的人。


\chapter{结尾}


他们的成功是历史与环境的产物,是机遇与积累的结晶。
说到底,所谓的”异类“从来就不是什么异类。






\backmatter
\listoftables
\listoffigures
\bibliographystyle{plainnat}
\bibliography{tex}
\end{document}
