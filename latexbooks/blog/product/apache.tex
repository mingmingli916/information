
\chapter{Deploy Django project with Apache}

You can run Django project separately and it works.
Then why did you deploy Django project to Apache or Nginx?
Becuase Django is intended only for use while developing.
It is in the business of making Web frameworks, not Web servers.
Deploying Django with Apache and \keyword{mod\_wsgi}\footnote{wsgi: web server gateway interface} is a tried and tested way to get Django into production.



\section{Apache}

\subsection{Install Apache httpd}

The operating system is CentOS 8.
\lstset{language=Sh}
\begin{lstlisting}
  dnf install httpd
\end{lstlisting}

\subsection{Start httpd}

\begin{lstlisting}
  systemctl start httpd
\end{lstlisting}

\subsection{Stop httpd}

\begin{lstlisting}
  systemctl stop httpd
\end{lstlisting}

\subsection{Enable httpd on operationg system start}

\begin{lstlisting}
  systemctl enable httpd
\end{lstlisting}



\section{mod\_wsgi}

The \verb|mod_wsgi| package implements a simple to use Apache module which can host any Python web application which supports the Python WSGI specification.

To install mod\_wsgi in CentOS 8:
\begin{lstlisting}
  dnf search mod_wsgi
  dnf install python3-mod_wsgi
\end{lstlisting}


\section{Permission}

Give apache the right to read, write and execute the files in your django project
\begin{lstlisting}
  sudo setfacl -m u:ec2-user:rwx /var
  sudo setfacl -m u:ec2-user:rwx /var/www
\end{lstlisting}

Give apache the right to read, write and execute the files in your django project
\begin{lstlisting}
  setfacl -R -m u:apache:rwx /var/www/mysite
\end{lstlisting}

Note: Sometime you need to give apache permission to you home directory.

%% \subsection{http permission}
%% Alter the configuration file ``/etc/httpd/conf/httpd.conf''.
Add execute perssion to \verb|wsgi.py| file:
\begin{lstlisting}
  chmod +x mysite/mysite/wsgi.py
\end{lstlisting}

\section{Configurate Apache}
Create a new file \verb|/etc/httpd/conf.d/mingmingli.net.conf|:

\lstset{language=Python}
\begin{lstlisting}

<VirtualHost *:80>
    DocumentRoot "/var/www/html"
    ServerName mingmingli.net
    ServerAlias www.mingmingli.net


</VirtualHost>
    
# you can visit the static file system  
Alias /static /var/www/mysite/static 
<Directory /var/www/mysite/static>
    Require all granted # permission
</Directory>

Alias /media /var/www/mysite/media
<Directory /var/www/mysite/media>
    Require all granted # permission
</Directory>

# permission to access wsgi.py file
<Directory /var/www/mysite/mysite> 
    <Files wsgi.py>
        Require all granted
    </Files>
</Directory>


WSGIDaemonProcess mingmingli.net python-path=/var/www/mysite:/home/ec2-user/anaconda3/lib/python3.9/site-packages
WSGIProcessGroup mingmingli.net
WSGIScriptAlias / /var/www/mysite/mysite/wsgi.py
\end{lstlisting}

Note:
You shoud your apache permission to access the packages in your home.
\begin{lstlisting}
  chmod +x /home/ec2-user
\end{lstlisting}

Don't  use  setfacl command to give  rwx permission to /home/ec2-user.
If you do that, the website works  but you cannot  login  ec2  with ssh  anymore with the  error:
\begin{lstlisting}
  ec2-user@ec2-3-129-43-26.us-east-2.compute.amazonaws.com: Permission denied (publickey,gssapi-keyex,gssapi-with-mic).
\end{lstlisting}


Otherwise, you may get a error with
\begin{lstlisting}
  from django.core.wsgi import get_wsgi_application
  ModuleNotFoundError: No module named 'django'
\end{lstlisting}



Change \verb|/etc/httpd/conf/httpd.conf|:
\begin{lstlisting}
  <Directory />
    AllowOverride none
    # Require all denied
    Require all granted
</Directory>
\end{lstlisting}

\section{Collect static files}

In order to let django can find the static files,
add the following configuration into \verb|blog/setting.py|:
\begin{lstlisting}
STATIC_ROOT = os.path.join(BASE_DIR, "static")
\end{lstlisting}

Then run the following command:
\lstset{language=Sh}
\begin{lstlisting}
python manage.py collectstatic  
\end{lstlisting}
This command collects all the installed application static files into ``static'' directory in blog.



