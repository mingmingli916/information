
\chapter{Translation}
The goal of translation is to provide several lanugage for end uers.

The process is:
\begin{enumerate}
\item Mark the content you want to translation.
\item Create lanugage files.
\item Alter the messages files and compile.
\end{enumerate}

About how to mark the content to translate, refer \url{https://docs.djangoproject.com/en/4.0/topics/i18n/translation/}.

\section{Enable  Translation}

Alter the settings.py in project root directory (the directory containg the manage.py file).

\begin{lstlisting}
LANGUAGE_CODE = 'en-us'

TIME_ZONE = 'UTC'

USE_I18N = True

USE_TZ = True

LOCALE_PATHS = [os.path.join(BASE_DIR, 'locale')]

LANGUAGES = [
    ('en', 'English'),
    ('fr', 'Français'),
]
\end{lstlisting}

\section{Mark}

\begin{lstlisting}


<link rel="stylesheet" type="text/css" href="">
<div class="center">

    <div class="pagination">
    <span class="step-links">
        
            <a href="?page=1">&laquo; </a>
            <a href="?page={{ page_obj.previous_page_number }}"> </a>
        

        
            
                <a class="active" href="?page={{ i }}">{{ i }}</a>
            
                <a href="?page={{ i }}">{{ i }}</a>
            
        

        
            <a href="?page={{ page_obj.next_page_number }}"></a>
            <a href="?page={{ page_obj.paginator.num_pages }}"> &raquo;</a>
        

        <a class="active"> {{ page_obj.number }}/{{ page_obj.paginator.num_pages }}</a>

    </span>
    </div>

</div>
\end{lstlisting}

\section{URL patterns}

Enable \verb|LocalMiddleware| in setting.py (project root directory):
\begin{lstlisting}
MIDDLEWARE = [
    'django.middleware.security.SecurityMiddleware',
    'django.contrib.sessions.middleware.SessionMiddleware',
    'django.middleware.common.CommonMiddleware',
    'django.middleware.csrf.CsrfViewMiddleware',
    'django.contrib.auth.middleware.AuthenticationMiddleware',
    'django.contrib.messages.middleware.MessageMiddleware',
    'django.middleware.clickjacking.XFrameOptionsMiddleware',
    'django.middleware.locale.LocaleMiddleware'
]
\end{lstlisting}


Edit the urls.py file:
\begin{lstlisting}


urlpatterns = [
    path('i18n/', include('django.conf.urls.i18n')),
    path('admin/', admin.site.urls),
    path('', include('home.urls')),
    path('books/', include('books.urls')),
    path('articles/', include('articles.urls')),
    path('od/', include('od.urls'))
]

urlpatterns += i18n_patterns(*urlpatterns)
\end{lstlisting}


\section{Create language files}

First, create a locale directory in the project root directory (the directory containing the manage.py file)
\begin{lstlisting}
  mkdir  locale
\end{lstlisting}


Second, run the command to create message files in the project root directory.
\begin{lstlisting}
  python manage.py makemessages -l fr
\end{lstlisting}

fr stands for French.


\section{Compile message files}

After you create your message file – and each time you make changes to it – you’ll need to compile it into a more efficient form, for use by gettext.




\begin{lstlisting}
python manage.py compilemessages
\end{lstlisting}




\section{Switch language on webpage}

\begin{lstlisting}





    
  
            <form action="" method="post">
                
                <select name="language">
                    
                        <option value="{{ language.code }}" selected>
                            {{ language.name_local }} ({{ language.code }})
                        </option>
                    
                </select>
                <input type="submit" value=>
            </form>  
\end{lstlisting}



\section{Model translation}

doc: \url{https://django-modeltranslation.readthedocs.io/en/latest/index.html}

The modeltranslation application is used to translate dynamic content of existing Django models to an arbitrary number of languages without having to change the original model classes. It uses a registration approach (comparable to Django's admin app) to be able to add translations to existing or new projects and is fully integrated into the Django admin backend.

\subsection{Install}


\begin{lstlisting}
pip install django-modeltranslation  
\end{lstlisting}


\subsection{Setup}


1. Add modeltranslation to the \verb|INSTALLED_APPS| variable of your project’s settings.py.

\begin{lstlisting}
INSTALLED_APPS = [
    'modeltranslation',
    'django.contrib.admin',
    'django.contrib.auth',
    'django.contrib.contenttypes',
    'django.contrib.sessions',
    'django.contrib.messages',
    'django.contrib.staticfiles',
    'home.apps.HomeConfig',
    'books.apps.BooksConfig',
    'articles.apps.ArticlesConfig',
    'od.apps.OdConfig',
]  
\end{lstlisting}

Make sure the ``modeltranslation'' is before the ``django.contrib.admin''.

2. Set \verb|USE_I18N = True| in settings.py.

3. Set LANGUAGES in setting.py. The first language is treated as the default lanugage

\begin{lstlisting}
gettext = lambda s: s
LANGUAGES = [
    ('en', gettext('English')),
    ('fr', gettext('French')),
]
\end{lstlisting}

\begin{tcolorbox}
The gettext lambda function is not a feature of modeltranslation, but rather required for Django to be able to (statically) translate the verbose names of the languages using the standard i18n solution.
\end{tcolorbox}




4. Create a translation.py in your app directory and register TranslationOptions for every model you want to translate.

5. Sync the database using ``python manage.py makemigrations'' and ``python manage.py migrate''.


\subsection{Registing models for translation}

Registering models and their fields for translation requires the following steps:

1. Create a ``translation.py'' in your app directory.
2. Create a translation option class for every model to translate.
3. Register the model and the translation option class at ``modeltranslation.translator.translator''.

\begin{lstlisting}
(base) li@Mingmings-MacBook-Pro mysite % tree articles 
articles
├── __init__.py
├── __pycache__
│   ├── __init__.cpython-38.pyc
│   ├── admin.cpython-38.pyc
│   ├── apps.cpython-38.pyc
│   ├── models.cpython-38.pyc
│   ├── urls.cpython-38.pyc
│   └── views.cpython-38.pyc
├── admin.py
├── apps.py
├── migrations
│   ├── 0001_initial.py
│   ├── 0002_remove_article_url_article_article_and_more.py
│   ├── 0003_rename_article_article_html.py
│   ├── __init__.py
│   └── __pycache__
│       ├── 0001_initial.cpython-38.pyc
│       ├── 0002_remove_article_url_article_article_and_more.cpython-38.pyc
│       ├── 0003_rename_article_article_html.cpython-38.pyc
│       └── __init__.cpython-38.pyc
├── models.py
├── static
│   └── articles
│       └── index.css
├── templates
│   └── articles
│       └── index.html
├── tests.py
├── translation.py
├── urls.py
└── views.py

7 directories, 24 files  
\end{lstlisting}


In ``translation.py'':
\begin{lstlisting}
from modeltranslation.translator import register, TranslationOptions
from .models import Article


@register(Article)
class ArticleTranslationOptions(TranslationOptions):
    fields = ('name', 'author', 'keywords', 'date', 'introduction')

\end{lstlisting}

Notice:
If you have only one field to be translated, write as following:
\begin{lstlisting}
@register(Type)
class TypeTranslationOptions(TranslationOptions):
    fields = ('title',)  # Notice here  
\end{lstlisting}


\subsection{Add translation to administration}

In you app's admin.py file:
\begin{lstlisting}
from .models import Article
from modeltranslation.admin import TranslationAdmin


# Register your models here.
class ArticleAdmin(admin.ModelAdmin):
    list_display = ('name', 'keywords', 'date', 'introduction')
    list_filter = ['date']
    search_fields = ['name', 'keywords', 'introduction']


class MyTranslatedArticleAdmin(ArticleAdmin, TranslationAdmin):
    pass


admin.site.register(Article, MyTranslatedArticleAdmin)
  
\end{lstlisting}


\subsection{Manage commands}



If you add translation to an existed project, run the command:
\begin{lstlisting}

python manage.py update_translation_fields  
\end{lstlisting}



You should run this command after adding a new language to your settings.LANGUAGES or a new field to the TranslationOptions of a registered model.

\begin{lstlisting}
python manage.py sync_translation_fields  
\end{lstlisting}



