
\chapter{Numbers}
\label{cha:numbers}

GNU Emacs support two numeric data types: \keyword{integers} and \keyword{floating-point numbers}.

\section{Iteger Basics}
\label{sec:iteger-basics}

The Lisp reader reads an integer as nonempty sequence of decimal digits with optional initial sign and optional final period.
\begin{lstlisting}
1
1.
+1
-1
0
-0
\end{lstlisting}

The syntax for integers in bases other than 10 consists of \argument{\#} followed by a radix indication followed by one or more digits.
\begin{lstlisting}
;; binary
#b101100                                ; 44
;; octal
#o54                                    ; 44
;; hex
#x2c                                    ; 44
;; #radix r integer, radix=24
#24r1k                                  ; 44
\end{lstlisting}

Many of the functions described in this chapter accept markers for arguments in place of numbers.
Since the actual arguments to such functions may be either numbers or markers, we often give these arguments the name \argument{number-or-marker}.
When the argument value is a marker, its position value is used and its buffer is ignored.


In Emacs Lisp, text characters are represented by integers.
Any integer between zero and the value of \lstinline|(max-char)|, inclusive, is considered to be valid as a character.


Integers in Emacs Lisp are not limited to the machine word size.
Under the hood, though, there are two kinds of integers: smaller ones, called \keyword{fixnums}, and larger ones, called \keyword{bignums}.

\begin{lstlisting}
most-positive-fixnum                    ; 2305843009213693951
most-negative-fixnum                    ; -2305843009213693952
;; Maximum number N of bits in safely-calculated integers.
integer-width                           ; 65536
\end{lstlisting}


\section{Floating-Point Basics}
\label{sec:float-point-basics}

The range of floating-point numbers is the same as the range of the C data type \argument{double} on the machine you are using.

The read syntax for floating-point numbers requires either a decimal point, an exponent, or both.
Optional signs (\argument{+} or \argument{-}) precede the number and its exponent.
\begin{lstlisting}
;; 1500
1500.0
+15e2
15.0e+2
+1500000e-3
.15e4
\end{lstlisting}
Emacs Lisp requires at least one digit after a decimal point in a floating-point number that does not have an exponent.
\argument{1500.} is an integer.

\begin{lstlisting}
;; read syntaxes for special floating-point values.
1.0e+INF                                ; 1.0e+INF
-1.0e+INF                               ; -1.0e+INF
0.0e+NaN                                ; 0.0e+NaN
-0.0e+NaN                               ; -0.0e+NaN
\end{lstlisting}

The following functions are specialized for handling floating-point numbers:
\begin{itemize}[itemsep=10pt]
\item \lstinline|(isnan x)|\\
  This predicate returns \argument{t} if its floating-point argument is a \argument{NaN}, \argument{nil} otherwise.
\item \lstinline|(frexp x)|\\
  This function returns a cons cell \argument{(s . e)}, where \argument{s} and \argument{e} are respectively the significand and exponent of the floating-point number \argument{x}.
\item \lstinline|(ldexp s e)|\\
  Given a numeric significand \argument{s} and an integer exponent \argument{e}, this function returns the floating point number \argument{\(s2^{e}\)}.
\item \lstinline|(copysign x1 x2)|\\
  This function copies the sign of \argument{x2} to the value of \argument{x1}, and returns the result.
  \argument{x1} and \argument{x2} must be floating point.
\item \lstinline|(logb x)|\\
  This function returns the binary exponent of \argument{x}.
\end{itemize}

\section{Type Predicates for Numbers}
\label{sec:type-pred-numb}

\begin{itemize}
\item bignump 
\item fixnump 
\item floatp 
\item integerp
\item numberp 
\item natnump
\item zerop 
\end{itemize}

natnum stands for natural number.

\section{Comparison of Numbers}
\label{sec:comparison-numbers}

\begin{itemize}[itemsep=10pt]
\item \lstinline|(= number-or-marker &rest number-or-markers)|\\
  This function tests whether all its arguments are numerically equal, and returns \argument{t} if so, \argument{nil} otherwise.
\item \lstinline|(eql value1 value2)|\\
  This function acts like \argument{eq} except when both arguments are numbers.
  It compares numbers by type and numeric value.
  Floating-point values with the same sign, exponent and fraction are \argument{eql}.
  This differs from numeric comparison: \lstinline|(eql 0.0 -0.0)| returns \argument{nil} and \lstinline|(eql 0.0e+NaN 0.0e+NaN)| returns \argument{t}, whereas \argument{=} does the opposite.
\item \lstinline|(/= number-or-marker1 number-or-marker2)|\\
  This function tests whether its arguments are numerically equal, and returns \argument{t} if they are not, and \argument{nil} if they are.
\item \lstinline|(< number-or-marker &rest number-or-markers)|\\
  This function tests whether each argument is strictly less than the following argument.
  It returns \argument{t} if so, \argument{nil} otherwise.
\item \lstinline|(<= number-or-marker &rest number-or-markers)|\\
  This function tests whether each argument is less than or equal to the following argument.
  It returns \argument{t} if so, \argument{nil} otherwise.
\item \lstinline|(> number-or-marker &rest number-or-markers)|\\
  This function tests whether each argument is strictly greater than the following argument.
  It returns \argument{t} if so, \argument{nil} otherwise.  
\item \lstinline|>= number-or-marker &rest number-or-markers|\\
  This function tests whether each argument is greater than or equal to the following argument.
  It returns \argument{t} if so, \argument{nil} otherwise.  
\item \lstinline|(max number-or-marker &rest numbers-or-markers)|\\
  This function returns the largest of its arguments.
\item \lstinline|min number-or-marker &rest numbers-or-markers|\\
  This function returns the smallest of its arguments.
\item \lstinline|(abs number)|\\
  This function returns the absolute value of number.
\end{itemize}

\section{Numeric Conversions}
\label{sec:numeric-conversions}

\begin{itemize}[itemsep=10pt]
\item \lstinline|(float number)|\\
  This returns \argument{number} converted to floating point.
  If \argument{number} is already floating point, \argument{float} returns it unchanged.
  
\item \lstinline|(truncate number &optional divisor)|\\
  This returns \argument{number}, converted to an integer by rounding towards zero.
\item \lstinline|(floor number &optional divisor)|\\
  This returns number, converted to an integer by rounding downward (towards negative infinity).
\item \lstinline|(ceiling number &optional divisor)|\\
  This returns \argument{number}, converted to an integer by rounding upward (towards positive infinity).
\item \lstinline|(round number &optional divisor)|\\
  This returns \argument{number}, converted to an integer by rounding towards the nearest integer.
\end{itemize}

\section{Arithmetic Operations} 
\label{sec:arithm-opera}

\begin{itemize}[itemsep=10pt]
\item \lstinline|(1+ number-or-marker)|\\
  This function returns number-or-marker plus 1. 
\item \lstinline|(1- numbers-or-markers)|\\
  This function returns number-or-marker minus 1.
\item \lstinline|(+ &rest numbers-or-markers)|\\
  This function adds its arguments together. When given no arguments, + returns 0.
\item \lstinline|(- &optional number-or-marker &rest more-numbers-or-markers)|\\
  The \argument{-} function serves two purposes: negation and subtraction.
  When \argument{-} has a single argument, the value is the negative of the argument.
  When there are multiple arguments, \argument{-} subtracts each of the \argument{more-numbers-or-markers} from \argument{number-or-marker}, cumulatively.
  If there are no arguments, the result is 0.
\item \lstinline|(* &rest numbers-or-markers)|\\
  This function multiplies its arguments together, and returns the product.
  When given no arguments, \argument{*} returns 1.
\item \lstinline|(/ number &rest divisors)|\\
  With one or more \argument{divisors}, this function divides \argument{number} by each divisor in \argument{divisors} in turn, and returns the quotient.
  With no \argument{divisors}, this function returns 1/\argument{number}.
  Each argument may be a number or a marker.
  If all the arguments are integers, the result is an integer, obtained by rounding the quotient towards zero after each division.
\item \lstinline|(% dividend divisor)|\\
  This function returns the integer remainder after division of \argument{dividend} by \argument{divisor}.
  The arguments must be integers or markers.
\item \lstinline|(mod dividend divisor)|\\
  This function returns the value of \argument{dividend} modulo \argument{divisor}; in other words, the remainder after division of \argument{dividend} by \argument{divisor}, but with the same sign as \argument{divisor}.
  The arguments must be numbers or markers.
  
  Unlike \argument{\%}, \argument{mod} permits floating-point arguments; it rounds the quotient downward (towards minus infinity) to an integer, and uses that quotient to compute the remainder.
  

\end{itemize}

\section{Rounding Operations}
\label{sec:rounding-operations}

The functions \argument{ffloor, fceiling, fround}, and \argument{ftruncate} take a floating-point argument and return a floating-point result whose value is a nearby integer.

\begin{lstlisting}
(floor 1.1)                             ; 1
(ffloor 1.1)                            ; 1.0
\end{lstlisting}

\section{Bitwise Opeartions on Integers}
\label{sec:bitw-opeart-iteg}

The bitwise operations in Emacs Lisp apply only to integers.

\begin{itemize}[itemsep=10pt]
\item \lstinline|(ash integer1 count)|\\
  \argument{ash} (arithmetic shift) shifts the bits in \argument{integer1} to the left \argument{count} places, or to the right if \argument{count} is negative.
  Left shifts introduce zero bits on the right; right shifts discard the rightmost bits.
\item \lstinline|(lsh integer1 count)|\\
  \argument{lsh} (logical shift) shifts the bits in \argument{integer1} to the left \argument{count} places, or to the right if \argument{count} is negative, bringing zeros into the vacated bits.
\begin{lstlisting}
;; -7 = ...111111111111111111111111111001
(ash -7 -1)                   ; -4 = ...111111111111111111111111111100
(lsh -7 -1)           ; 536870908  = ...011111111111111111111111111100
\end{lstlisting}
\item \lstinline|(logand &rest ints-or-markers)|\\
  This function returns the bitwise AND of the arguments: the nth bit is 1 in the result if, and only if, the nth bit is 1 in all the arguments.
  If \argument{logand} is not passed any argument, it returns a value of -1.
  This number is an identity element for \argument{logand} because its binary representation consists entirely of ones.
  If \argument{logand} is passed just one argument, it returns that argument.
\item \lstinline|(logior &rest ints-or-markers)|\\
  This function returns the bitwise inclusive OR of its arguments: the nth bit is 1 in the result if, and only if, the nth bit is 1 in at least one of the arguments.
  If there are no arguments, the result is 0, which is an identity element for this operation.
  If \argument{logior} is passed just one argument, it returns that argument.
\item \lstinline|(logxor &rest ints-or-markers)|\\
  This function returns the bitwise exclusive OR of its arguments: the nth bit is 1 in the result if, and only if, the nth bit is 1 in an odd number of the arguments.
  If there are no arguments, the result is 0, which is an identity element for this operation.
  If \argument{logxor} is passed just one argument, it returns that argument.
\item \lstinline|(lognot integer)|\\
  This function returns the bitwise complement of its argument: the nth bit is one in the result if, and only if, the nth bit is zero in \argument{integer}, and vice-versa. 
\item \lstinline|(logcount integer)|\\
  This function returns the \keyword{Hamming weight} of \argument{integer}: the number of ones in the binary representation of \argument{integer}.
  If \argument{integer} is negative, it returns the number of zero bits in its two’s complement binary representation.
  The result is always nonnegative.
\end{itemize}

\section{Standard Mathematical Functions}
\label{sec:stand-math-funct}

These mathematical functions allow integers as well as floating-point numbers as arguments.
\begin{itemize}
\item \lstinline|(sin arg)|
\item \lstinline|(cos arg)|
\item \lstinline|(tan arg)|
\item \lstinline|(asin arg)|
\item \lstinline|(acos arg)|
\item \lstinline|(atan y &optional x)|
\item \lstinline|(exp arg)|
\item \lstinline|(log arg &optional base)|
\item \lstinline|(expt x y)|
\item \lstinline|(sqrt arg)|
\item \lstinline|float-e|
\item \lstinline|float-pi|
\end{itemize}

\section{Random Numbers}
\label{sec:random-numbers}

A deterministic computer program cannot generate true random numbers.
For most purposes, \keyword{pseudo-random numbers} suffice.
A series of pseudo-random numbers is generated in a deterministic fashion.
The numbers are not truly random, but they have certain properties that mimic a random series.
For example, all possible values occur equally often in a pseudo-random series.



Pseudo-random numbers are generated from a \keyword{seed} value.
Starting from any given seed, the random function always generates the same sequence of numbers.
By default, Emacs initializes the random seed at startup, in such a way that the sequence of values of random (with overwhelming likelihood) differs in each Emacs run.


Sometimes you want the random number sequence to be repeatable.
For example, when debugging a program whose behavior depends on the random number sequence, it is helpful to get the same behavior in each program run.
To make the sequence repeat, execute \lstinline|(random "")|.
This sets the seed to a constant value for your particular Emacs executable (though it may differ for other Emacs builds).
You can use other strings to choose various seed values.

\begin{lstlisting}
(random &optional limit)
\end{lstlisting}
This function returns a pseudo-random integer.
Repeated calls return a series of pseudo-random integers.

If \argument{limit} is a positive integer, the value is chosen to be nonnegative and less than \argument{limit}.
Otherwise, the value might be any fixnum.
If \argument{limit} is \argument{t}, it means to choose a new seed as if Emacs were restarting, typically from the system entropy.
On systems lacking entropy pools, choose the seed from less-random volatile data such as the current time.
If \argument{limit} is a string, it means to choose a new seed based on the string’s contents.


\section{Summary}
\label{sec:summary-1}

\begin{figure}[H]
  \centering
  \myfigure[0.8]{numbers}
  \caption{Numbers}
\end{figure}

%%% Local Variables:
%%% mode: latex
%%% TeX-master: "elisp"
%%% End:
