
\chapter{Sequences, Arrays, and Vectors}
\label{cha:sequ-arrays-vect}

\begin{figure}[H]
  \centering
  \myfigure{sequence}
  \caption{Sequences}
  \label{fig:sequences}
\end{figure}


\section{Sequences}
\label{sec:sequences}


\begin{itemize}
\item \lstinline|(sequencep object)|\\
  This function returns \argument{t} if \argument{object} is a sequence, \argument{nil} otherwise.
\item \lstinline|(length sequence)|\\
  This function returns the number of elements in sequence.
\begin{lstlisting}
(length '(1 2 3))                       ; 3
\end{lstlisting}
\item \lstinline|(length< sequence length)|\\
  Return non-\argument{nil} if \argument{sequence} is shorter than \argument{length}.
  This may be more efficient than computing the length of sequence if \argument{sequence} is a long list.
\item \lstinline|(length> sequence length)|
\item \lstinline|(length= sequence length)|
\item \lstinline|(elt sequence index)|\\
  This function returns the element of sequence indexed by index.
\begin{lstlisting}
(elt '(1 2 3 4) 2)                      ; 3
\end{lstlisting}
\item \lstinline|(copy-sequence seqr)|\\
  This function returns a copy of \argument{seqr}, which should be either a sequence or a record.
\item \lstinline|(reverse sequence)|\\
  This function creates a new sequence whose elements are the elements of \argument{sequence}, but in reverse order.
  The original argument \argument{sequence} is not altered. 
\item \lstinline|(nreverse sequence)|\\
  This function reverses the order of the elements of \argument{sequence}.
  Unlike \funcword{reverse} the original \argument{sequence} may be modified.
\item \lstinline|(sort sequence predicate)|\\
  This function sorts \argument{sequence} stably.
  Note that this function doesn’t work for all sequences; it may be used only for lists and vectors.
\item \lstinline|(seqp)|\\
  This function returns non-nil if object is a sequence (a list or array), or any additional type of sequence defined via \argument{seq.el} generic functions.
  This is an extensible variant of \funcword{sequencep}.
\item \lstinline|(seq-drop sequence n)|\\
  This function returns all but the first \argument{n} (an integer) elements of \argument{sequence}.
  If \argument{n} is negative or zero, the result is \argument{sequence}.
\begin{lstlisting}
(seq-drop [1 2 3 4 5 6] 3)              ; [4 5 6]
\end{lstlisting}
\item \lstinline|(seq-take sequence n)|\\
  This function returns the first \argument{n} (an integer) elements of \argument{sequence}.
  If \argument{n} is negative or zero, the result is nil.
\begin{lstlisting}
(seq-take '(1 2 3 4) 3)                 ; (1 2 3)
\end{lstlisting}
\item \lstinline|(seq-take-while predicate sequence)|\\
  This function returns the members of \argument{sequence} in order, stopping before the first one for which \argument{predicate} returns nil.
\begin{lstlisting}
(seq-take-while (lambda (elt) (> elt 0)) '(1 2 3 -1 -2))
; (1 2 3)
\end{lstlisting}
\item \lstinline|(seq-drop-while predicate sequence)|
\item \lstinline|(seq-do function sequence)|\\
  This function applies function to each element of \argument{sequence} in turn (presumably for side effects), and returns \argument{sequence}.
\item \lstinline|(seq-map function sequence)|\\
  This function returns the result of applying \argument{function} to each element of \argument{sequence}.
  The returned value is a list.
\begin{lstlisting}
(seq-map #'1+ '(2 4 6))                 ; (3 5 7)
\end{lstlisting}
\item \lstinline|(seq-map-indexed function sequence)|\\
  This function returns the result of applying \argument{function} to each element of \argument{sequence} and its index within seq.
  The returned value is a list.
\begin{lstlisting}
(seq-map-indexed (lambda (elt idx)
                   (list idx elt))
                 '(a b c))
;;  ((0 a) (1 b) (2 c))
\end{lstlisting}
\item \lstinline|(seq-mapn function &rest sequences)|\\
  This function returns the result of applying function to each element of sequences.
\begin{lstlisting}
(seq-mapn #'+ '(2 4 6) '(20 40 60))
;;  (22 44 66)
\end{lstlisting}
\item \lstinline|(seq-filter predicate sequence)|\\
  This function returns a list of all the elements in \argument{sequence} for which \argument{predicate} returns non-nil.
\begin{lstlisting}
(seq-filter (lambda (elt) (> elt 0)) [1 -1 3 -3 5])
;;  (1 3 5)
\end{lstlisting}
\item \lstinline|(seq-remove predicate sequence)|\\
  This function returns a list of all the elements in \argument{sequence} for which \argument{predicate} returns nil.
\begin{lstlisting}
(seq-remove (lambda (elt) (> elt 0)) [1 -1 3 -3 5])
;;  (-1 -3)
\end{lstlisting}
  
\item \lstinline|(seq-reduce function sequence initial-value)|\\
  This function returns the result of calling \argument{function} with \argument{initial-value} and the first element of \argument{sequence}, then calling \argument{function} with that result and the second element of \argument{sequence}, then with that result and the third element of \argument{sequence}, etc.
  \argument{function} should be a function of two arguments.
\begin{lstlisting}
(seq-reduce #'+ '(1 2 3 4) 5)
;;  15
\end{lstlisting}
  
\item \lstinline|(seq-some predicate sequence)|\\
  This function returns the first non-nil value returned by applying \argument{predicate} to each element of sequence in turn.
\begin{lstlisting}
(seq-some #'numberp ["abc" 1 nil])      ; t
(seq-some #'1+ [2 4 6])                 ; 3
\end{lstlisting}
  
\item \lstinline|(seq-find predicate sequence &optional default)|\\
  This function returns the first element in \argument{sequence} for which \argument{predicate} returns non-nil.
  If no element matches \argument{predicate}, the function returns \argument{default}.
\begin{lstlisting}
(seq-find #'numberp ["abc" 1 nil])      ; 1
\end{lstlisting}
\item \lstinline|(seq-every-p predicate sequence)|\\
  This function returns non-nil if applying \argument{predicate} to every element of \argument{sequence} returns non-nil.
\begin{lstlisting}
(seq-every-p #'numberp [2 4 6])         ; t
(seq-every-p #'numberp [2 4 "6"])       ; nil
\end{lstlisting}
  
\item \lstinline|(seq-empty-p sequence)|\\
  This function returns non-nil if \argument{sequence} is empty.

\item \lstinline|(seq-count predicate sequence)|\\
  This function returns the number of elements in \argument{sequence} for which \argument{predicate} returns non-nil.
\begin{lstlisting}
(seq-count (lambda (elt) (> elt 0)) [-1 2 0 3 -1]) ; 2
\end{lstlisting}
\item \lstinline|(seq-sort function sequence)|\\
  This function returns a copy of \argument{sequence} that is sorted according to \argument{function}, a function of two arguments that returns non-nil if the first argument should sort before the second.
\begin{lstlisting}
(seq-sort #'> '(1 2 3 4))               ; (4 3 2 1)
\end{lstlisting}
\item \lstinline|(seq-sort-by function predicate sequence)|\\
  This function is similar to \funcword{seq-sort}, but the elements of \argument{sequence} are transformed by applying \argument{function} on them before being sorted.
  \argument{function} is a function of one argument.
\begin{lstlisting}
(seq-sort-by #'seq-length #'> ["a" "ab" "abc"])
;;  ["abc" "ab" "a"]
\end{lstlisting}

  
\item \lstinline|(seq-contains-p sequence elt &optional function)|\\
  This function returns non-nil if at least one element in \argument{sequence} is equal to \argument{elt}.
  If the optional argument \argument{function} is non-nil, it is a function of two arguments to use instead of the default \funcword{equal}.
\begin{lstlisting}
(seq-contains-p '(symbol1 symbol2) 'symbol1) ; t
(seq-contains-p '(symbol1 symbol2) 'symbol3) ; nil
\end{lstlisting}

\item \lstinline|(seq-set-equal-p sequence1 sequence2 &optional function)|\\
  This function checks whether \argument{sequence1} and \argument{sequence2} contain the same elements, regardless of the order.
  If the optional argument \argument{function} is non-nil, it is a function of two arguments to use instead of the default \funcword{equal}.
\begin{lstlisting}
(seq-set-equal-p '(a b c) '(c b a))
;;  t
(seq-set-equal-p '(a b c) '(c b))
;;  nil
(seq-set-equal-p '("a" "b" "c") '("c" "b" "a"))
;;  t
(seq-set-equal-p '("a" "b" "c") '("c" "b" "a") #'eq)
;;  nil
\end{lstlisting}
  
\item \lstinline|(seq-position sequence elt &optional function)|\\
  This function returns the index of the first element in \argument{sequence} that is equal to \argument{elt}.
  If the optional argument \argument{function} is non-nil, it is a function of two arguments to use instead of the default \funcword{equal}.
\begin{lstlisting}
(seq-position '(a b c) 'b)              ; 1
(seq-position '(a b c) 'd)              ; nil
\end{lstlisting}

  
\item \lstinline|(seq-uniq sequence &optional function)|\\
  This function returns a list of the elements of \argument{sequence} with duplicates removed.
  If the optional argument \argument{function} is non-nil, it is a function of two arguments to use instead of the default \funcword{equal}.
\begin{lstlisting}
(seq-uniq '(1 2 2 1 3))                 ; (1 2 3)
(seq-uniq '(1 2 2.0 1.0) #'=)           ; (1 2)
(seq-uniq '(1 2 2.0 1.0))               ; (1 2 2.0 1.0)
\end{lstlisting}

  
\item \lstinline|(seq-subseq sequence start &optional end)|\\
  This function returns a subset of \argument{sequence} from \argument{start} to \argument{end}, both integers (end defaults to the last element).
  If \argument{start} or \argument{end} is negative, it counts from the end of \argument{sequence}.
\begin{lstlisting}
(seq-subseq '(1 2 3 4 5) 1)
;;  (2 3 4 5)
(seq-subseq '[1 2 3 4 5] 1 3)
;;  [2 3]
(seq-subseq '[1 2 3 4 5] -3 -1)
;;  [3 4]
\end{lstlisting}

\item \lstinline|(seq-concatenate type &rest sequences)|\\
  This function returns a sequence of type \argument{type} made of the concatenation of \argument{sequences}.
  \argument{type} may be: vector, list or string.
\begin{lstlisting}
(seq-concatenate 'list '(1 2) '(3 4) [5 6])
;;  (1 2 3 4 5 6)
(seq-concatenate 'string "Hello " "world")
;;  "Hello world"
\end{lstlisting}

  
\item \lstinline|(seq-mapcat function sequence &optional type)|\\
  This function returns the result of applying \funcword{seq-concatenate} to the result of applying \argument{function} to each element of \argument{sequence}.
  The result is a sequence of type \argument{type}, or a list if \argument{type} is nil.
\begin{lstlisting}
(seq-mapcat #'seq-reverse '((3 2 1) (6 5 4)))
;;  (1 2 3 4 5 6)
\end{lstlisting}

  
\item \lstinline|(seq-partition sequence n)|\\
  This function returns a list of the elements of \argument{sequence} grouped into sub-sequences of length \argument{n}.
  The last sequence may contain less elements than \argument{n}.
  \argument{n} must be an integer.
  If \argument{n} is a negative integer or 0, the return value is nil.
\begin{lstlisting}
(seq-partition '(0 1 2 3 4 5 6 7) 3)
;;  ((0 1 2) (3 4 5) (6 7))
\end{lstlisting}

  
\item \lstinline|(seq-union sequence1 sequence2 &optional function)|\\
  This function returns a list of the elements that appear either in \argument{sequence1} or \argument{sequence2}.
  The elements of the returned list are all unique, in the sense that no two elements there will compare equal.
  If the optional argument \argument{function} is non-nil, it should be a function of two arguments to use to compare elements, instead of the default \funcword{equal}.
\begin{lstlisting}
(seq-union [1 2 3] [3 5])
;;  (1 2 3 5)
\end{lstlisting}

\item \lstinline|(seq-intersection sequence1 sequence2 &optional function)|\\
  This function returns a list of the elements that appear both in \argument{sequence1} and \argument{sequence2}.
  If the optional argument \argument{function} is non-nil, it is a function of two arguments to use to compare elements instead of the default \funcword{equal}.
\begin{lstlisting}
(seq-intersection [2 3 4 5] [1 3 5 6 7])
;;  (3 5)
\end{lstlisting}


  
\item \lstinline|(seq-difference sequence1 sequence2 &optional function)|\\
  This function returns a list of the elements that appear in \argument{sequence1} but not in \argument{sequence2}.
  If the optional argument \argument{function} is non-nil, it is a function of two arguments to use to compare elements instead of the default \funcword{equal}.
\begin{lstlisting}
(seq-difference '(2 3 4 5) [1 3 5 6 7])
;;  (2 4)
\end{lstlisting}


\item \lstinline|(seq-group-by function sequence)|\\
  This function separates the elements of \argument{sequence} into an alist whose keys are the result of applying \argument{function} to each element of \argument{sequence}.
  Keys are compared using \funcword{equal}.
\begin{lstlisting}
(seq-group-by #'integerp '(1 2.1 3 2 3.2))
;;  ((t 1 3 2) (nil 2.1 3.2))
(seq-group-by #'car '((a 1) (b 2) (a 3) (c 4))) 
;;  ((b (b 2)) (a (a 1) (a 3)) (c (c 4)))
\end{lstlisting}

  
\item \lstinline|(seq-into sequence type)|\\
  This function converts the sequence \argument{sequence} into a sequence of type \argument{type}.
  \argument{type} can be one of the following symbols: vector, string or list.
\begin{lstlisting}
(seq-into [1 2 3] 'list)
;;  (1 2 3)
(seq-into nil 'vector)
;;  []
(seq-into "hello" 'vector)
;;  [104 101 108 108 111]
\end{lstlisting}

  
\item \lstinline|(seq-min sequence)|\\
  This function returns the smallest element of \argument{sequence}.
  The elements of \argument{sequence} must be numbers or markers.
\item \lstinline|(seq-max sequence)|
\item \lstinline|(seq-doseq (var sequence) body...)|\\
  This macro is like \funcword{dolist}, except that \argument{sequence} can be a list, vector or string.
  This is primarily useful for side-effects.

\item \lstinline|(seq-let var-sequence val-sequence body...)|\\
  This macro binds the variables defined in \argument{var-sequence} to the values that are the corresponding elements of \argument{val-sequence}.
  This is known as \keyword{destructuring binding}.
  The elements of \argument{var-sequence} can themselves include sequences, allowing for nested destructuring.

  The \argument{var-sequence} sequence can also include the \argument{\&rest} marker followed by a variable name to be bound to the rest of \argument{val-sequence}.

\begin{lstlisting}
(seq-let [first second] [1 2 3 4]
  (list first second))
;;  (1 2)
(seq-let (_ a _ b) '(1 2 3 4)
  (list a b))
;;  (2 4)
(seq-let [a [b [c]]] [1 [2 [3]]]
  (list a b c))
;;  (1 2 3)
(seq-let [a b &rest others] [1 2 3 4]
  others)
;;  [3 4]
\end{lstlisting}

  
\item \lstinline|(seq-setq var-sequence val-sequence)|\\
  This macro works similarly to \funcword{seq-let}, except that values are assigned to variables as if by \funcword{setq} instead of as in a \funcword{let} binding.
\begin{lstlisting}
(let ((a nil)
      (b nil))
  (seq-setq (_ a _ b) '(1 2 3 4))
  (list a b))
;;  (2 4)
\end{lstlisting}

  
\item \lstinline|(seq-random-elt sequence)|\\
  This function returns an element of \argument{sequence} taken at random.
\begin{lstlisting}
(seq-random-elt [1 2 3 4])              ; 3
(seq-random-elt [1 2 3 4])              ; 3
(seq-random-elt [1 2 3 4])              ; 3
(seq-random-elt [1 2 3 4])              ; 1
(seq-random-elt [1 2 3 4])              ; 2
\end{lstlisting}
\end{itemize}








%%% Local Variables:
%%% mode: latex
%%% TeX-master: "elisp"
%%% End:

\section{Arrays}
\label{sec:arrays}

Emacs defines four types of array, all one-dimensional:
\begin{itemize}
\item strings 
\item vectors
\item char-tables
\item bool-vectors
\end{itemize}

Vectors and char-tables can hold elements of any type, but strings can only hold characters, and bool-vectors can only hold t and nil.


All four kinds of array share these characteristics:
\begin{itemize}
\item All have zero-origin indexing.
\item The length of the array is fixed once you create it; you cannot change the length of an existing array.
\item For purposes of evaluation, the array is a constant—i.e., it evaluates to itself.
\item The elements of an array may be referenced or changed with the functions \funcword{aref} and \funcword{aset}, respectively 
\end{itemize}


When you create an array, other than a char-table, you must specify its length.
You cannot specify the length of a char-table, because that is determined by the range of character codes.



Here's the functions that accept all types of arrays.
\begin{itemize}
\item \lstinline|(arrayp object)|
\item \lstinline|(aref arr index)|\\
  This function returns the \argument{index}th element of the array or record \argument{arr}.
\item \lstinline|(aset array index object)|\\
  This function sets the \argument{index}th element of \argument{array} to be object.
  It returns \argument{object}.
\item \lstinline|(fillarray array object)|\\
  This function fills the array \argument{array} with \argument{object}, so that each element of \argument{array} is \argument{object}.
  It returns \argument{array}.
\begin{lstlisting}
(setq a (copy-sequence [a b c d e f g])) ; need to be mutable
;; [a b c d e f g]
(fillarray a 0)
;; [0 0 0 0 0 0 0]
a
;; [0 0 0 0 0 0 0]
\end{lstlisting}
\end{itemize}

\section{Vectors}
\label{sec:vectors}

A vector is a general-purpose array whose elements can be any Lisp objects.
Vectors are printed with square brackets surrounding the elements.
You can write vectors in the same way in Lisp input.

Here are some functions that relate to vectors:
\begin{itemize}
\item \lstinline|(vectorp object)|
\item \lstinline|(vector &rest objects)|\\
  This function creates and returns a vector whose elements are the arguments, \argument{objects}.
\begin{lstlisting}
(vector 'foo 23 [bar baz] "rats")
;; [foo 23 [bar baz] "rats"]
(vector)
;; []
\end{lstlisting}
\item \lstinline|(make-vector length object)|\\
  This function returns a new vector consisting of \argument{length} elements, each initialized to \argument{object}.
\begin{lstlisting}
(setq sleepy (make-vector 9 'Z))
;; [Z Z Z Z Z Z Z Z Z]
\end{lstlisting}
\item \lstinline|(vconcat &rest sequences)|\\
  This function returns a new vector containing all the elements of \argument{sequences}.
\begin{lstlisting}
(setq a (vconcat '(A B C) '(D E F)))
;; [A B C D E F]
(eq a (vconcat a))
;; nil
(vconcat [A B C] "aa" '(foo (6 7)))
;; [A B C 97 97 foo (6 7)]
\end{lstlisting}
\end{itemize}


\section{Char-Tables}
\label{sec:char-tables}

A char-table is much like a vector, except that it is indexed by character codes.
Any valid character code, without modifiers, can be used as an index in a char-table.
In addition, a char-table can have extra slots to hold additional data not associated with particular character codes.


Each char-table has a \keyword{subtype} (a symbol) which serves two purposes:
\begin{itemize}
\item The subtype provides an easy way to tell what the char-table is for.
  For instance, display tables are char-tables with \argument{display-table} as the subtype, and syntax tables are char-tables with \argument{syntax-table} as the subtype.
\item The subtype controls the number of extra slots in the char-table.
  This number is specified by the subtype’s char-table-extra-slots symbol property, whose value should be integer between 0 and 10.
  If the subtype has no such symbol property, the char-table has no extra slots.
\end{itemize}


A char-table can have a parent, which is another char-table.
If it does, then whenever the char-table specifies nil for a particular character c, it inherits the value specified in the parent.

A char-table can also have a default value.
If so, then (aref char-table c) returns the default value whenever the char-table does not specify any other non-nil value.


\begin{itemize}
\item \lstinline|(make-char-table subtype &optional init)|\\
  Return a newly-created char-table, with subtype \argument{subtype} (a symbol).
  Each element is initialized to \argument{init}, which defaults to nil.
  You cannot alter the subtype of a char-table after the char-table is created.

  There is no argument to specify the length of the char-table, because all char-tables have room for any valid character code as an index.

  
\item \lstinline|(char-table-p object)|
  
\item \lstinline|(char-table-subtype char-table)|\\
  This function returns the subtype symbol of \argument{char-table}.
\item \lstinline|(char-table-parent char-table)|\\
  This function returns the parent of \argument{char-table}.
\item \lstinline|(set-char-table-parent char-table new-parent)|\\
  This function sets the parent of \argument{char-table} to \argument{new-parent}.
\item \lstinline|(char-table-extra-slot char-table n)|\\
  This function returns the contents of extra slot \argument{n} (zero based) of \argument{char-table}.
\item \lstinline|(set-char-table-extra-slot char-table n value)|\\
  This function stores \argument{value} in extra slot \argument{n} (zero based) of \argument{char-table}.
\item \lstinline|(char-table-range char-table range)|\\
  This returns the value specified in \argument{char-table} for a range of characters \argument{range}.
  Here are the possibilities for range:
  \begin{itemize}
  \item nil: refers to the default value.
  \item char: refers to the element for character char.
  \item \lstinline|(from . to)|: A cons cell refers to all the characters in the inclusive range \argument{[from to]}.
  \end{itemize}
\item \lstinline|(set-char-table-range char-table range value)|\\
  This function sets the value in \argument{char-table} for a range of characters \argument{range}.
  Here are the possibilities for \argument{range}:
  \begin{itemize}
  \item nil: refers to the default value.
  \item t: refers to the whole range of character codes.
  \item char: refers to the element for character char.
  \item \lstinline|(from . to)|: A cons cell refers to all the characters in the inclusive range \argument{[from to]}.
  \end{itemize}
\item \lstinline|(map-char-table function char-table)|\\
  This function calls \argument{function} for each element of \argument{char-table} that has a non-nil value.
  The call to \argument{function} is with two arguments, a key and a value.
  The key is either a valid character or a cons cell \lstinline|(from . to)|.
  The value is what \lstinline|(char-table-range char-table key)| returns.

  The return value is always nil; to make calls to \funcword{map-char-table} useful, \argument{function} should have side effects. 

\end{itemize}

\section{Bool-vectors}
\label{sec:bool-vectors}



A bool-vector is much like a vector, except that it stores only the values t and nil.
If you try to store any non-nil value into an element of the bool-vector, the effect is to store t there.


\begin{itemize}
\item \lstinline|(make-bool-vector length initial)|\\
  Return a new bool-vector of \argument{length} elements, each one initialized to \argument{initial}.
\item \lstinline|(bool-vector &rest objects)|\\
  This function creates and returns a bool-vector whose elements are the arguments, \argument{objects}.
\item \lstinline|(bool-vector-p object)|
\item \lstinline|bool-vector-exclusive-or a b &optional c|\\
  Return \keyword{bitwise exclusive or} of bool vectors \argument{a} and \argument{b}.
  If optional argument \argument{c} is given, the result of this operation is stored into \argument{c}.
  All arguments should be bool vectors of the same length.
  
\item \lstinline|(bool-vector-union a b &optional c)|\\
  Return \keyword{bitwise or} of bool vectors \argument{a} and \argument{b}.
\item \lstinline|(bool-vector-intersection a b &optional c)|\\
  Return \keyword{bitwise and} of bool vectors \argument{a} and \argument{b}.
\item \lstinline|(bool-vector-set-difference a b &optional c)|\\
  Return \keyword{set difference} of bool vectors \argument{a} and \argument{b}.
\item \lstinline|(bool-vector-not a &optional b)|\\
  Return \keyword{set complement} of bool vector \argument{a}.
  If optional argument \argument{b} is given, the result of this operation is stored into \argument{b}.
\item \lstinline|(bool-vector-subsetp a b)|\\
  Return t if every t value in a is also t in b, nil otherwise. 
\item \lstinline|(bool-vector-count-consecutive a b i)|\\
  Return the number of consecutive elements in a equal \argument{b} starting at \argument{i}.
  \argument{a} is a bool vector, \argument{b} is t or nil, and \argument{i} is an index into a.
\item \lstinline|(bool-vector-count-population a)|\\
  Return the number of elements that are t in bool vector \argument{a}.

  The printed form represents up to 8 boolean values as a single character:
\begin{lstlisting}
(bool-vector t nil t nil)               ; #&4"^E"
(bool-vector)                           ; #&0""

;; You can use vconcat to print a bool-vector like other vectors:
(vconcat (bool-vector nil t nil t))
;; [nil t nil t]
\end{lstlisting}


  
\end{itemize}


%%% Local Variables:
%%% mode: latex
%%% TeX-master: "elisp"
%%% End:
