
\chapter{Lists}
\label{cha:lists}

A \keyword{list} represents a sequence of zero or more elements (which may be any Lisp objects).


\section{Lists and Cons Cells}
\label{sec:lists-cons-cells}

A list is a series of \keyword{cons cells} chained together, so that each cell refers to the next one.
There is one cons cell for each element of the list.
By convention, the cars of the cons cells hold the elements of the list, and the cdrs are used to chain the list (this asymmetry between car and cdr is entirely a matter of convention; at the level of cons cells, the car and cdr slots have similar properties).

Also by convention, the cdr of the last cons cell in a list is \argument{nil}.
We call such a \argument{nil}-terminated structure a \keyword{proper list}.
If the cdr of a list’s last cons cell is some value other than \argument{nil}, we call the structure a \keyword{dotted list}, since its printed representation would use dotted pair notation.
There is one other possibility: some cons cell’s cdr could point to one of the previous cons cells in the list. We call that structure a \keyword{circular list}.

Because most cons cells are used as part of lists, we refer to any structure made out of cons cells as a \keyword{list structure}.




\section{Predicates on Lists}
\label{sec:predicates-lists}

\begin{itemize}
\item \lstinline|(consp object)|\\
  This function returns \argument{t} if object is a cons cell, \argument{nil} otherwise.
  \argument{nil} is not a cons cell, although it is a list.
\item \lstinline|(atom object)|\\
  This function returns \argument{t} if object is an atom, \argument{nil} otherwise.
  All objects except cons cells are atoms.
  The symbol \argument{nil} is an atom and is also a list; it is the only Lisp object that is both.
\item \lstinline|(listp object)|\\
  This function returns \argument{t} if object is a cons cell or \argument{nil}.
  Otherwise, it returns \argument{nil}.
\item \lstinline|(nlistp object)|\\
  This function is the opposite of listp: it returns \argument{t} if object is not a list.
  Otherwise, it returns \argument{nil}.
\item \lstinline|(null object)|\\
  This function returns \argument{t} if object is \argument{nil}, and returns \argument{nil} otherwise.
\item \lstinline|(proper-list-p object)|\\
  This function returns the length of object if it is a proper list, \argument{nil} otherwise.
\end{itemize}


\section{Accessing Elements of Lists}
\label{sec:access-elem-lists}

\begin{itemize}
\item \lstinline|(car cons-cell)|\\
  This function returns the value referred to by the first slot of the cons cell \argument{cons-cell}.
  If \argument{cons-cell} is \argument{nil}, this function returns \argument{nil}.
  An error is signaled if the argument is not a cons cell or \argument{nil}.
\item \lstinline|(cdr cons-cell)|\\
  This function returns the value referred to by the second slot of the cons cell \argument{cons-cell}.
  If \argument{cons-cell} is \argument{nil}, this function returns \argument{nil}.
  An error is signaled if the argument is not a cons cell or \argument{nil}.
\item \lstinline|(car-safe object)|\\
  This function lets you take the car of a cons cell while avoiding errors for other data types.
  It returns the car of \argument{object} if \argument{object} is a cons cell, \argument{nil} otherwise.
\item \lstinline|(cdr-safe object)|\\
  This function lets you take the cdr of a cons cell while avoiding errors for other data types.
  It returns the cdr of \argument{object} if \argument{object} is a cons cell, \argument{nil} otherwise.
\item \lstinline|(pop listname)|\\
  This macro provides a convenient way to examine the car of a list, and take it off the list, all at once.
  It operates on the list stored in \argument{listname}.
  It removes the first element from the list, saves the cdr into \argument{listname}, then returns the removed element.
\item \lstinline|(nth n list)|\\
  This function returns the nth element of list.
  If the length of list is \argument{n} or less, the value is \argument{nil}.
\item \lstinline|(nthcdr n list)|\\
  This function returns the nth cdr of list. 
\item \lstinline|(last list &optional n)|\\
  This function returns the last link of list.
  The car of this link is the list’s last element.
  If \argument{list} is null, \argument{nil} is returned.
  If \argument{n} is non-\argument{nil}, the nth-to-last link is returned instead, or the whole of list if \argument{n} is bigger than list’s length.
\item \lstinline|(safe-length list)|\\
  This function returns the length of list, with no risk of either an error or an infinite loop.
  If \argument{list} is not \argument{nil} or a cons cell, safe-length returns 0.
\item \lstinline|(butlast x &optional n)|\\
  This function returns the list \argument{x} with the last element, or the last \argument{n} elements, removed.
  If \argument{n} is greater than zero it makes a copy of the list so as not to damage the original list.
  In general, \lstinline|(append (butlast x n) (last x n))| will return a list equal to \argument{x}.
\item \lstinline|(nbutlast x &optional n)|\\
  This is a version of \argument{butlast} that works by destructively modifying the cdr of the appropriate element, rather than making a copy of the list.
\end{itemize}

\section{Building Cons Cells and Lists}
\label{sec:building-cons-cells}

\begin{itemize}
\item \lstinline|(cons object1 object2)|\\
  This function is the most basic function for building new list structure.
  It creates a new cons cell, making \argument{object1} the car, and \argument{object2} the cdr.
  It then returns the new cons cell.
\begin{lstlisting}
(cons 1 '(2))                           ; (1 2)
(cons 1 '())                            ; (1)
(cons 1 2)                              ; (1 . 2)
\end{lstlisting}
\item \lstinline|(list &rest objects)|\\
  This function creates a list with \argument{objects} as its elements.
  The resulting list is always \argument{nil}-terminated.
  If no \argument{objects} are given, the empty list is returned.
\begin{lstlisting}
(list 1 2 3 4 5)                        ; (1 2 3 4 5)
(list 1 2 '(3 4 5) 'foo)                ; (1 2 (3 4 5) foo)
(list)                                  ; nil
\end{lstlisting}
\item \lstinline|(make-list length object)|\\
  This function creates a list of \argument{length} elements, in which each element is \argument{object}.
\begin{lstlisting}
(make-list 3 'pigs)                     ; (pigs pigs pigs)
\end{lstlisting}
\item \lstinline|(append &rest sequences)|\\
  This function returns a list containing all the elements of sequences.
  All arguments except the last one are copied, so none of the arguments is altered.
  The final argument is not copied or converted; it becomes the cdr of the last cons cell in the new list.
\begin{lstlisting}
(setq trees '(pine oak))                ; (pine oak)
(setq more-trees (append '(maple birch) trees)) ; (maple birch pine oak)
trees                                           ; (pine oak)
more-trees                                      ; (maple birch pine oak)
(append [a b] "cd" nil)                         ; (a b 99 100)
(append)                                        ; nil
(append '(x y) 'z)                              ; (x y . z)
\end{lstlisting}
\item \lstinline|(copy-tree tree &optional vecp)|\\
  This function returns a copy of the tree \argument{tree}.
  If \argument{tree} is a cons cell, this makes a new cons cell with the same car and cdr, then recursively copies the car and cdr in the same way.

  Normally, when \argument{tree} is anything other than a cons cell, \funcword{copy-tree} simply returns tree.
  However, if \argument{vecp} is non-nil, it copies vectors too (and operates recursively on their elements).
\item \lstinline|(flatten-tree tree)|\\
  This function returns a ``flattened'' copy of tree, that is, a list containing all the non-\argument{nil} terminal nodes, or leaves, of the tree of cons cells rooted at \argument{tree}.
\begin{lstlisting}
(flatten-tree '(1 (2 . 3) nil (4 5 (6)) 7)) ; (1 2 3 4 5 6 7)
\end{lstlisting}
\item \lstinline|(ensure-list object)|\\
  This function returns \argument{object} as a list.
  If \argument{object} is already a list, the function returns it; otherwise, the function returns a one-element list containing \argument{object}.
\item \lstinline|(number-sequence from &optional to separation)|\\
  This function returns a list of numbers starting with \argument{from} and incrementing by \argument{separation}, and ending at or just before \argument{to}.
\begin{lstlisting}
(number-sequence 1.5 6 2)               ; (1.5 3.5 5.5)
\end{lstlisting}
\end{itemize}

\section{Modifying List Variables}
\label{sec:modify-list-vari}

\begin{itemize}
\item \lstinline|(push element listname)|\\
  This macro creates a new list whose car is \argument{element} and whose cdr is the list specified by \argument{listname}, and saves that list in \argument{listname}.
\begin{lstlisting}
(setq l '(a b))                         ; (a b)
(push 'c l)                             ; (c a b)
l                                       ; (c a b)
\end{lstlisting}
\item \lstinline|(pop listname)|\\
  This macro provides a convenient way to examine the car of a list, and take it off the list, all at once.
  It removes the first element from the list, saves the cdr into \argument{listname}, then returns the removed element.
\item \lstinline|(add-to-list symbole element &optional append compare-fn)|\\
  This function sets the variable \argument{symbol} by consing \argument{element} onto the old value, if \argument{element} is not already a member of that value.
  It returns the resulting list, whether updated or not.
\begin{lstlisting}
(setq foo '(a b))                       ; (a b)
(add-to-list 'foo 'c)                   ; (c a b)
(add-to-list 'foo 'b)                   ; (c a b)
foo                                     ; (c a b)
(add-to-list 'foo 'd t)                 ; (c a b d)
\end{lstlisting}
\item \lstinline|(add-to-ordered-list symbol element &optional order)|\\
  This function sets the variable \argument{symbol} by inserting \argument{element} into the old value, which must be a list, at the position specified by \argument{order}.
  If \argument{element} is already a member of the list, its position in the list is adjusted according to \argument{order}.
  Membership is tested using \funcword{eq}.
  This function returns the resulting list, whether updated or not.
\begin{lstlisting}
(setq foo '())                          ;  nil
(add-to-ordered-list 'foo 'a 1)         ; (a)
(add-to-ordered-list 'foo 'c 3)         ; (a c)
(add-to-ordered-list 'foo 'b 2)         ; (a b c)
(add-to-ordered-list 'foo 'b 4)         ; (a c b)
(add-to-ordered-list 'foo 'd)           ; (a c b d)
(add-to-ordered-list 'foo 'e)           ; (a c b e d)
foo                                     ; (a c b e d)
\end{lstlisting}
\end{itemize}

\section{Modifying Existing List Structure}
\label{sec:modify-exist-list}

You can modify the car and cdr contents of a cons cell with the primitives \funcword{setcar} and \funcword{setcdr}.
These are destructive operations because they change existing list structure.


\begin{itemize}
\item \lstinline|(setcar cons object)|\\
  This function stores \argument{object} as the new car of \argument{cons}, replacing its previous car.
  It returns the value \argument{object}.
\begin{lstlisting}
(setq x (list 1 2))                     ; (1 2)
(setcar x 4)                            ; 4
x                                       ; (4 2)

;; Create two lists that are partly shared.
(setq x1 (list 'a 'b 'c))               ; (a b c)
(setq x2 (cons 'z (cdr x1)))            ; (z b c)
;; Replace the car of a shared link.
(setcar (cdr x1) 'foo)                  ; foo
x1                                      ; (a foo c)
x2                                      ; (z foo c)
;; Replace the car of a link that is not shared.
(setcar x1 'baz)                        ; baz
x1                                      ; (baz foo c)
x2                                      ; (z foo c)
\end{lstlisting}
  
\item \lstinline|(setcdr cons object)|\\
  This function stores \argument{object} as the new cdr of \argument{cons}, replacing its previous cdr.
  It returns the value object.
\begin{lstlisting}
(setq x (list 1 2 3))                   ; (1 2 3)
(setcdr x '(4))                         ; (4)
x                                       ; (1 4)

(setq x1 (list 'a 'b 'c))               ;  (a b c)
(setcdr x1 (cdr (cdr x1)))              ; (c)
x1                                      ; (a c)

(setq x1 (list 'a 'b 'c))               ; (a b c)
(setcdr x1 (cons 'd (cdr x1)))          ; (d b c)
x1                                      ; (a d b c)
\end{lstlisting}
  
\item \lstinline|(nconc &rest lists)|\\
  This function returns a list containing all the elements of \argument{lists}.
  Unlike \funcword{append}, the \argument{lists} are not copied.
  Instead, the last cdr of each of the \argument{lists} is changed to refer to the following list.
  The last of the \argument{lists} is not altered.

  Since the last argument of \funcword{nconc} is not itself modified, it is reasonable to use a constant list.
  However, the other arguments (all but the last) should be mutable lists.

\begin{lstlisting}
(setq x (list 1 2 3))                   ; (1 2 3)
(nconc x '(4 5))                        ; (1 2 3 4 5)
x                                       ; (1 2 3 4 5)
\end{lstlisting}
\end{itemize}

\section{Using Lists as Sets}
\label{sec:using-lists-as}

\begin{itemize}
\item \lstinline|(memq object list)|\\
  This function tests to see whether \argument{object} is a member of \argument{list}.
  If it is, \funcword{memq} returns a list starting with the first occurrence of \argument{object}.
  Otherwise, it returns \argument{nil}.
  The letter ‘q’ in \funcword{memq} says that it uses \funcword{eq} to compare \argument{object} against the elements of the \argument{list}.
\begin{lstlisting}
(memq 'b '(a b c b a))                  ; (b c b a)
\end{lstlisting}
\item \lstinline|(delq object list)|\\
  This function destructively removes all elements \funcword{eq} to \argument{object} from \argument{list}, and returns the resulting list.

  The \funcword{delq} function deletes elements from the front of the list by simply advancing down the list, and returning a sublist that starts after those elements.
  When an element to be deleted appears in the middle of the list, removing it involves changing the cdrs.
\begin{lstlisting}
(delq 'a '(a b c))                      ; (b c)

(setq sample-list (list 'a 'b 'c '(4))) ; (a b c (4))
(delq 'a sample-list)                   ; (b c (4))
sample-list                             ; (a b c (4))
(delq 'c sample-list)                   ; (a b (4))
sample-list                             ; (a b (4))
\end{lstlisting}
\item \lstinline|(remq object list)|\\
  This function returns a copy of \argument{list}, with all elements removed which are \funcword{eq} to \argument{object}.
\begin{lstlisting}
(setq sample-list (list 'a 'b 'c 'a 'b 'c)) ; (a b c a b c)
(remq 'a sample-list)                       ; (b c b c)
sample-list                                 ; (a b c a b c)
\end{lstlisting}
\item \lstinline|(memql object list)|\\
  This function tests to see whether \argument{object} is a member of \argument{list}, comparing members with \argument{object} using \funcword{eql}.
  If \argument{object} is a member, \funcword{memql} returns a list starting with its first occurrence in \argument{list}.
  Otherwise, it returns \argument{nil}.
\begin{lstlisting}
(memql 1.2 '(1.1 1.2 1.3))              ; (1.2 1.3)
(memq 1.2 '(1.1 1.2 1.3))               ; nil
\end{lstlisting}
\item \lstinline|(member object list)|\\
  This function tests to see whether \argument{object} is a member of \argument{list}, comparing members with \argument{object} using \funcword{equal}.
  If \argument{object} is a member, \funcword{member} returns a list starting with its first occurrence in \argument{list}.
  Otherwise, it returns \argument{nil}.
\begin{lstlisting}
(member '(2) '((1) (2)))                ; ((2))
(memq '(2) '((1) (2)))                  ; nil
\end{lstlisting}
\item \lstinline|(delete object sequence)|\\
  This function removes all elements \funcword{equal} to \argument{object} from \argument{sequence}, and returns the resulting sequence.
  If \argument{sequence} is a list, it \funcword{delete} likes \funcword{delq} but comparing with \funcword{equal}.
  If \argument{sequence} is a vector or string, \funcword{delete} returns a copy of \argument{sequence} with all elements \funcword{equal} to \argument{object} removed.
\begin{lstlisting}
(setq l (list '(2) '(1) '(2)))          ; ((2) (1) (2))
(delete '(2) l)                         ; ((1))
l                                       ; ((2) (1))

(setq l [(2) (1) (2)])                  ; [(2) (1) (2)]
(delete '(2) l)                         ; [(1)]
l                                       ; [(2) (1) (2)]
\end{lstlisting}
\item \lstinline|(remove object sequence)|\\
  This function is the non-destructive counterpart of \funcword{delete}.
  It returns a copy of \argument{sequence}, a list, vector, or string, with elements \funcword{equal} to \argument{}object removed.
\item \lstinline|(member-ignore-case object list)|\\
  This function is like \funcword{member}, except that \argument{object} should be a string and that it ignores differences in letter-case and text representation.
\item \lstinline|(delete-dups list)|\\
  This function destructively removes all \funcword{equal} duplicates from \argument{list}, stores the result in \argument{list} and returns \argument{it}.
  Of several \funcword{equal} occurrences of an element in \argument{list}, \funcword{delete-dups} keeps the first one.
\end{itemize}


\section{Association Lists}
\label{sec:association-lists}


An \keyword{association list}, or \keyword{alist} for short, records a mapping from keys to values.
It is a list of cons cells called \keyword{associations}: the car of each cons cell is the key, and the cdr is the associated value.

\begin{itemize}
\item \lstinline|(assoc key alist &optional testfn)|\\
  This function returns the first association for \argument{key} in \argument{alist}, comparing \argument{key} against the alist elements using \argument{testfn} if it is a function, and \funcword{equal} otherwise.
  The function returns \argument{nil} if no association in \argument{alist} has a car equal to \argument{key}, as tested by \argument{testfn}.
\begin{lstlisting}
(setq trees '((pine . cones) (oak . acorns) (maple . seeds)))
;; ((pine . cones) (oak . acorns) (maple . seeds))
(assoc 'oak trees)
;;  (oak . acorns)
(cdr (assoc 'oak trees))
;;  acorns
(assoc 'birch trees)
;;  nil
\end{lstlisting}
\item \lstinline|(assoc-string key alist &optional case-fold)|\\
  This function works like \funcword{assoc}, except that \argument{key} must be string or symbol, and comparison is done using \funcword{compare-strings}.
  Symbols are converted to strings before testing.
  If \argument{case-fold} is non-\argument{nil}, \argument{key} and the elements of \argument{alist} are converted to uppercase before comparison. 
\item \lstinline|(rassoc value alist)|\\
  This function returns the first association with value \argument{value} in \argument{alist}.
  It returns \argument{nil} if no association in \argument{alist} has a cdr \funcword{equal} to \argument{value}.
\item \lstinline|(assq key alist)|\\
  This function is like \funcword{assoc} in that it returns the first association for key in \argument{alist}, but it makes the comparison using \funcword{eq}.
\item \lstinline|(alist-get key alist &optional default remove testfn)|\\
  It finds the first association by comparing key with \argument{alist} elements, and, if found, return the value of that association.
  If no association is found, the function returns \argument{default}.
  Comparison of \argument{key} against \argument{alist} elements uses the function specified by \argument{testfn}, defaulting to \funcword{eq}.
\item \lstinline|(rassq value alist)|\\
  This function returns the first association with value \argument{value} in \argument{alist}.
  It returns \argument{nil} if no association in \argument{alist} has a cdr \funcword{eq} to value.
\item \lstinline|(assoc-default key alist &optional testfn default)|\\
  This function searches \argument{alist} for a match for \argument{key}.
  For each element of \argument{alist}, it compares the element (if it is an atom) or the element’s car (if it is a cons) against \argument{key}, by calling \argument{testfn} with two arguments: the element or its car, and \argument{key}. 

  If an \argument{alist} element matches \argument{key} by this criterion, then \funcword{assoc-default} returns a value based on this element.
  If the element is a cons, then the value is the element’s cdr.
  Otherwise, the return value is \argument{default}.
  
  If no alist element matches key, \funcword{assoc-default} returns \argument{nil}.
\item \lstinline|(copy-alist alist)|\\
  This function returns a two-level deep copy of \argument{alist}.
\item \lstinline|(assq-delete-all key alist)|\\
  This function deletes from \argument{alist} all the elements whose car is \funcword{eq} to \argument{key}, much as if you used \funcword{delq} to delete each such element one by one.
  It returns the shortened alist, and often modifies the original list structure of \argument{alist}.
  For correct results, use the return value of \funcword{assq-delete-all} rather than looking at the saved value of \argument{alist}.
\item \lstinline|(assoc-delete-all key alist &optional testfn)|\\
  This function is like \funcword{assq-delete-all} except that it accepts an optional argument \funcword{testfn}.
  If omitted or \argument{nil}, \argument{testfn} defaults to \funcword{equal}. 
\item \lstinline|(rassq-delete-all value alist)|\\
  This function deletes from \argument{alist} all the elements whose cdr is \funcword{eq} to value.
  It returns the shortened alist, and often modifies the original list structure of alist.
  \funcword{rassq-delete-all} is like \funcword{assq-delete-all} except that it compares the cdr of each alist association instead of the car.
\item \lstinline|(let-alist alist body)|\\
  Creates a binding for each symbol used as keys, prefixed with dot.
  This can be useful when accessing several items in the same association list.
\begin{lstlisting}
(setq colors '((rose . red) (lily . white) (buttercup . yellow)))
;;  ((rose . red) (lily . white) (buttercup . yellow))
(let-alist colors
  (if (eq .rose 'red)
      .lily))
;;  white
\end{lstlisting}
  The \argument{body} is inspected at compilation time, and only the symbols that appear in body with a '.' as the first character in the symbol name will be bound.
  Finding the keys is done with \funcword{assq}.
  Nested association lists is supported:
\begin{lstlisting}
(setq colors '((rose . red) (lily (belladonna . yellow) (brindisi . pink))))
;;  ((rose . red) (lily (belladonna . yellow) (brindisi . pink)))
(let-alist colors
  (if (eq .rose 'red)
      .lily.belladonna))
;;  yellow
\end{lstlisting}
\end{itemize}

\section{Property Lists}
\label{sec:property-lists}


A \keyword{property list} (\keyword{plist} for short) is a list of paired elements.
Each of the pairs associates a property name (usually a symbol) with a property or value.


\begin{lstlisting}
(pine cones numbers (1 2 3) color "blue")
\end{lstlisting}
This property list associates \argument{pine} with \argument{cones}, \argument{numbers} with \argument{(1 2 3)}, and \argument{color} with \argument{"blue"}.


\begin{itemize}
\item \lstinline|(plist-get plist property)|\\
  This returns the value of the \argument{property} property stored in the property list \argument{plist}.
  If \argument{property} is not found in the \argument{plist}, it returns \argument{nil}.
\begin{lstlisting}
(plist-get '(foo 4) 'foo)               ; 4
(plist-get '(foo 4 bad) 'foo)           ; 4
(plist-get '(foo 4 bad) 'bad)           ; nil
\end{lstlisting}
\item \lstinline|(plist-put plist property value)|\\
  This stores \argument{value} as the value of the \argument{property} property in the property list \argument{plist}.
  The function returns the modified property list.
\begin{lstlisting}
(setq my-plist (list 'bar t 'foo 4))    ; (bar t foo 4)
(setq my-plist (plist-put my-plist 'foo 69)) ; (bar t foo 69)
(setq my-plist (plist-put my-plist 'quux '(a))) ; (bar t foo 69 quux (a))
\end{lstlisting}
\item \lstinline|(lax-plist-get plist property)|\\
  Like \funcword{plist-get} except that it compares properties using \funcword{equal} instead of \funcword{eq}.
\item \lstinline|(lax-plist-put plist property value)|\\
  Like \funcword{plist-put} except that it compares properties using \funcword{equal} instead of \funcword{eq}.  
\item \lstinline|(plist-member plist property)|\\
  This returns non-\argument{nil} if \argument{plist} contains the given \argument{property}.
  Unlike \funcword{plist-get}, this allows you to distinguish between a missing property and a property with the value \argument{nil}.
\begin{lstlisting}
(plist-get '(foo 4 bad) 'bad)           ; nil
(plist-get '(foo 4 bad) 'bar)           ; nil
(plist-member '(foo 4 bad) 'bad)        ; (bad)
(plist-member '(foo 4 bad) 'bar)        ; nil
\end{lstlisting}
\end{itemize}

%%% Local Variables:
%%% mode: latex
%%% TeX-master: "elisp"
%%% End:
