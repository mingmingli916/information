
\chapter{Templates}
\label{cha:templates}

A template is a file that contains the text of a response, with placeholder variables for the dynamic parts that will be known only in the context of a request.
The process that replaces the variables with actual values and returns a final response string is called rendering.
For the task of rendering templates, Flask uses a powerful template engine called Jinja2.


\section{Rendering templates}
\label{sec:rendering-templates}

\begin{lstlisting}[caption={templates/user.html},label={lst:user}]
<h1>Hello, {{ name }}!</h1>
\end{lstlisting}

\begin{lstlisting}[caption=rendering a template]
from flask import Flask, render_template

# ...

@app.route('/user/<name>')
def user(name):
    return render_template('user.html', name=name)
\end{lstlisting}

\section{Variables}
\label{sec:variables}

The \lstinline|{{ name }}| construct used in the template shown in Listing \ref{lst:user} references a variable, a special placeholder that tells the template engine that the value that goes in that place should be obtained from data provided at the time the template is rendered.

\section{Control structures}
\label{sec:control-structures}

\begin{lstlisting}

    ...

    ...





    ...





    ...











    ...




\end{lstlisting}

In general, the code part is placed in the \lstinline||.

\section{Links}
\label{sec:links}

Flask provides the \lstinline|url_for()| helper function, which gener‐ ates URLs from the information stored in the application's URL map.
This function can also be used to link static files.

\begin{lstlisting}

    {{ super() }}
    <link rel="shortcut icon" href="{{ url_for('static', filename='favicon.ico') }}" type="image/x-icon">
    <link rel="icon" href="{{ url_for('static', filename='favicon.ico') }}" type="image/x-icon">
    <link rel="stylesheet" type="text/css" href="{{ url_for('static', filename='styles.css') }}">
    <script src="{{ url_for('static',filename='ckeditor/ckeditor.js') }}"></script>

\end{lstlisting}

%%% Local Variables:
%%% mode: latex
%%% TeX-master: "website"
%%% End:
