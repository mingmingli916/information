
\chapter{Flask Deploy}
\label{cha:deploy}

Here we used the Flask to develop the web site and used gunicorn and nginx to deploy it.



\section{Gunicorn}
\label{sec:gunicorn}

\subsection{Install Gunicorn}
\label{sec:install-gunicorn}

\begin{lstlisting}
  pip install gunicorn
\end{lstlisting}


\subsection{Creating the WSGI Entry Point}
\label{sec:application}

Create \keyword{wsgi.py} in you application.
The content of the file is as follows:

\begin{lstlisting}
from myblog import app

if __name__ == '__main__':
    app.run()
\end{lstlisting}



\subsection{Start Gunicorn}
\label{sec:start-gunicorn}

\begin{lstlisting}
  gunicorn --bind 0.0.0.0:8000 wsgi:app
\end{lstlisting}


\subsection{Build a Gunicorn Service}
\label{sec:build-gunic-serv}

Create the file \keyword{/etc/systemd/system/myblog.service}:
\begin{lstlisting}
[Unit]
Sescription=Gunicorn instance to serve myblog
After=network.target

[Service]
User=root
Group=www-data
WorkingDirectory=/var/www/myblog
Environment="PATH=/var/www/myblog/venv/bin"
ExecStart=/var/www/myblog/venv/bin/gunicorn --workers 3 --bind 0.0.0.0:8000 wsgi:app

[Install]
WantedBy=multi-user.target  
\end{lstlisting}

After the creation, you can start the service and enable it to start at the server starts up:
\begin{lstlisting}
  systemclt start myblog
  systemctl enable myblog
\end{lstlisting}



\section{Nginx}
\label{sec:nginx}


\subsection{Install Nginx}
\label{sec:install-nginx}

\begin{lstlisting}
  dnf install nginx
  # or
  apt install nginx
\end{lstlisting}

\subsection{Config}
\label{sec:config}

Create the file \keyword{/etc/nginx/sites-available/myblog}:
\begin{lstlisting}
server {
    listen 80;
    server_name mingmingli.site www.mingmingli.site;

    location / {
        include proxy_params;       
        proxy_pass http://127.0.0.1:8000/;
    }
}  
\end{lstlisting}

Then create a link:
\begin{lstlisting}
  ln -s /etc/nginx/sites-available/myblog /etc/nginx/sites-enabled/
\end{lstlisting}

You can test the configuration is correct or not:
\begin{lstlisting}
  nginx -t
\end{lstlisting}

\subsection{Tell Flask it is Behind a Proxy}

myblog.py
\begin{lstlisting}
from werkzeug.middleware.proxy_fix import ProxyFix

app.wsgi_app = ProxyFix(
    app.wsgi_app, x_for=1, x_proto=1, x_host=1, x_prefix=1
)  
\end{lstlisting}


\section{SSL}
\label{sec:ssl}

\url{https://certbot.eff.org/instructions}


%%% Local Variables:
%%% mode: latex
%%% TeX-master: "website"
%%% End:
