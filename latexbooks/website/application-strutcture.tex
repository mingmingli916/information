
\chapter{Application structure}
\label{cha:appl-struct}

\usetikzlibrary{arrows,snakes,backgrounds}
\tikzstyle{transition}=[rectangle,draw=black!50,thick, inner sep=0pt,minimum size=2cm]
\tikzstyle{pre}=[<-,shorten <=1pt,>=stealth]
\tikzstyle{post}=[->,shorten >=1pt,>=stealth]
\begin{tikzpicture}
  \draw (0,0) rectangle ++(10,10);
  \node at (5,10.5) {Application};
  \node[transition] (response) at (2,2) {response};
  \node[transition] (view) at (8,2) {view}
  edge [post] (response);
  
  \node[transition] (route) at (8,8) {route}
  edge [post] (view);

  \node[transition] (request) at (2,8) {request}
  edge [post] node [auto] {dispatch} (route);

  \node at (-2,8) {}
  edge [post] (request);

  \node at (-2,2) {}
  edge [pre] (response);

  
\end{tikzpicture}


\section{Initialization}
\label{sec:initialization}

All Flask applications must create an application instance.
The web server passes all requests it receives from clients to this object for handling, using a protocol called Web Server Gateway Interface (WSGI).
The application instance is an object of class Flask, usually created as follows:
\begin{lstlisting}
from flask import Flask
app = Flask(__name__)
\end{lstlisting}

\section{Routes and view functions}
\label{sec:rout-view-funct}

The association between a URL and the function that handles it is called a route.
\begin{lstlisting}
@app.route('/')
def index():
    return '<h1>Hello World!</h1>'
\end{lstlisting}


\section{Responses}
\label{sec:responses}

\begin{lstlisting}
from flask import make_response
@app.route('/')
def index():
    response = make_response('<h1>This document carries a cookie!</h1>')
    response.set_cookie('answer', '42')
    return response
\end{lstlisting}

\section{Development server}
\label{sec:development-server}

Flask applications include a development web server.
To start the hello.py application:
\begin{lstlisting}
flask run hello.py

# or
export FLASK_APP=hello.py
flask run

# to enable debug mode
flask run hello.py --debug
# or
export FLASK_DEBUG=1
flask run
\end{lstlisting}

\section{Extensions}
\label{sec:extensions}

To use extensions:
\begin{enumerate}
\item Install the extension, for example
\begin{lstlisting}
pip install flask-bootstrap
\end{lstlisting}
  
\item Initialize the extension.
\begin{lstlisting}
from flask_bootstrap import Bootstrap
# ...
bootstrap = Bootstrap(app)
\end{lstlisting}
  
  Here's another method to initialize the extension:
\begin{lstlisting}
bootstrap = Bootstrap()
bootstrap.init_app(app)
\end{lstlisting}
\item Use it in the template or in the code.
\begin{lstlisting}

\end{lstlisting}
\end{enumerate}

%%% Local Variables:
%%% mode: latex
%%% TeX-master: "website"
%%% End:
