
\chapter{感悟}


\section{计划}

\keyword{计划}就是想在某个时间节点或则之前完成某件事。
计划的目的是实现某愿景。
內在的驱动力是愿景实现后的美好。

\section{专精与必需}


必需是不可或缺的事物。
专精是你的技术或则知识水平高于大多数人。
不同的人有不同的专精与必需。


专精是因为我们的精力和时间是有限的,
在有限的时间和精力内,
我们要只有专注某些事情才会比其他人做的更好。


\section{质与量}


量变引起质变。

\section{自学}


自学是一种有效获取知识的方式。
自学具有主动性。



\section{知识}


知识是一种思维认知,它是超脱于现实的,它即可以符合某种现实,也可以不符合某种现实。
随着知识的积累,我们提高了对特定知识理解能力的加强。


知识的不对等是我们致富的法宝。


\section{责任}

由于“趋利避害”的进化特性,承担责任需要我们的主观能动性。


\section{借口}

借口其实是一种自我价值的保护,避免借口的最好方法就是自信。

\section{能力}

能力是我们能做某些事,
能力分为先天的和后天的。


\section{努力}


努力是为了达到自己的\keyword{目标}而进行的脑力和体力的付出。

\section{困难}

困难有两个属性:门槛属性和时间属性。

\section{记忆}

记忆分为记和忆。
负责记忆的主要器官是大脑,也有一部分肌肉记忆。
记忆遵循遗忘定律。
右脑的记忆量巨大,牢固。


记忆是学习的基础。


\section{学习}

真正的学习态度。
责任,兴趣,荣誉。。。
通过实践来学习。

真正的学习态度是什么样的呢?
\begin{itemize}
\item 充满了兴趣
\item 遇到困难会主动去寻找资料
\item 想要了解底层的原理
\item 经常练习
\item 定期对知识进行总结
\end{itemize}



学习的方式是什么?怎么提高学习的效率?
比如,遇到不会的词,要记录下来,还要经常复习。



我们可以通过惩罚或者奖励来促进学习。

\section{信息}

发现信息的能力至关重要。 比如,发现kaggle这个平台可以提高自己机器学习水平。

发现他人的需求可以开发出相应的软件来谋取福利。


\section{数据}

数据很重要,它展示了你的历史。

但更重要的是分析。 通过分析,我们可以知道优缺点,今儿提高自己。

\section{心态}

拥有良好的心态,你就成功了一半。

\section{准则}

应用某些准则前,你需要问自己几个问题:
\begin{itemize}
\item 这些准则是否适用于我的情况。
\item 如果违反准则,利是否大于弊。
\end{itemize}

\section{观测}

我们感知到的并不是真实的,现实的观测是在排除更多的不可能。

\section{方法论}

我们遇到问题后,解决问题的一系列步骤。

\section{行为分析}

回想处理事情的行为,有哪些需要改善的?

\section{词汇}

词汇是需要的基础。


\section{基本功}


在经过100\%准确率的打字训练后,我明显发现打字的感觉更好了。
这就是基本功对效率的提升。

\section{10000小时定律}
\label{sec:10000}

通过10000个小时的训练,你可以成为某个领域的专家。

\section{数据}

数据很重要,它展示了你的历史。

但更重要的是分析。
通过分析,我们可以知道优缺点,今儿提高自己。

\section{实践}

实践是巩固知识的最好方法。
实践是提高能力的最有效的方法。


\section{概念}

人与人能力的不同是概念的不同。


\section{观察}

观察为记忆提供良好的素材。

%%% Local Variables:
%%% mode: latex
%%% TeX-master: "mingmingli"
%%% End:
