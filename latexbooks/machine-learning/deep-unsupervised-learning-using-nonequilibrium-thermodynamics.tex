
\chapter{Deep Unsupervised Learning using Nonequilibrium Thermodynamics\protect{\cite{js2015deep}}}
\label{cha:deep-unsup-learn}


The algorithm consists of two trajectories: forward (inference) diffusion process and inverse (generative) diffusion process.
The forward diffusion process converts complex data distribution into a simple, tractable distribution and the inverse diffusion process learn a finite-time reversal of the forward diffusion process which define a generative model distribution.

\section{Algorithm}
\label{sec:algorithm}



\subsection{Forward trajectory}
\label{sec:forward-trajectory}
\begin{equation}
  \label{eq:18}
  \pi(\mathbf{y})=\int d \mathbf{y}^{\prime} T_\pi\left(\mathbf{y} \mid \mathbf{y}^{\prime} ; \beta\right) \pi\left(\mathbf{y}^{\prime}\right)
\end{equation}

Where \(\pi(\mathbf{y})\) is a well behaved (analyti- cally tractable) distribution.
\(T_\pi\left(\mathbf{y} \mid \mathbf{y}^{\prime} ; \beta\right)\) is a Markov diffusion kernel.
\(\beta\) is the diffusion rate.

\begin{equation}
  \label{eq:29}
  q\left(\mathbf{x}^{(t)} \mid \mathbf{x}^{(t-1)}\right)=T_\pi\left(\mathbf{x}^{(t)} \mid \mathbf{x}^{(t-1)} ; \beta_t\right)
\end{equation}

The forward trajectory, corresponding to starting at the data distribution and performing \(T\) steps of diffusion is
\begin{equation}
  \label{eq:30}
  q\left(\mathbf{x}^{(0 \cdots T)}\right)=q\left(\mathbf{x}^{(0)}\right) \prod_{t=1}^T q\left(\mathbf{x}^{(t)} \mid \mathbf{x}^{(t-1)}\right)
\end{equation}

Where \(q\left(\mathbf{x}^{(0)}\right)\) is the initial data distribution.

\subsection{Reverse Trajectory}
\label{sec:reverse-trajectory}
The generative distribution will be trained to describe the same trajectory, but in reverse,
\begin{equation}
  \label{eq:31}
  p\left(\mathbf{x}^{(T)}\right)=\pi\left(\mathbf{x}^{(T)}\right)
\end{equation}
\begin{equation}
  \label{eq:32}
  p\left(\mathbf{x}^{(0 \cdots T)}\right)=p\left(\mathbf{x}^{(T)}\right) \prod_{t=1}^T p\left(\mathbf{x}^{(t-1)} \mid \mathbf{x}^{(t)}\right)
\end{equation}

For both Gaussian and binomial diffusion, for continuous diffusion (limit of small step size $\beta$ ) the reversal of the diffusion process has the identical functional form as the forward process.
Since $q\left(\mathbf{x}^{(t)} \mid \mathbf{x}^{(t-1)}\right)$ is a Gaussian (binomial) distribution, and if $\beta_t$ is small, then $q\left(\mathbf{x}^{(t-1)} \mid \mathbf{x}^{(t)}\right)$ will also be a Gaussian (binomial) distribution.
The longer the trajectory the smaller the diffusion rate $\beta$ can be made.

During learning only the mean and covariance for a Gaussian diffusion kernel, or the bit flip probability for a binomial kernel, need be estimated.


\section{Model probability}
\label{sec:model-probability}

The probability the generative model assigns to the data is
\begin{equation}
  \label{eq:33}
p\left(\mathbf{x}^{(0)}\right)=\int d \mathbf{x}^{(1 \cdots T)} p\left(\mathbf{x}^{(0 \cdots T)}\right)  
\end{equation}

Naively this integral is intractable – but taking a cue from annealed importance sampling and the Jarzynski equality, we instead evaluate the relative probability of the forward and reverse trajectories, averaged over forward trajectories
\begin{equation}
  \label{eq:34}
\begin{aligned}
& p\left(\mathbf{x}^{(0)}\right)=\int d \mathbf{x}^{(1 \cdots T)} p\left(\mathbf{x}^{(0 \cdots T)}\right) \frac{q\left(\mathbf{x}^{(1 \cdots T)} \mid \mathbf{x}^{(0)}\right)}{q\left(\mathbf{x}^{(1 \cdots T)} \mid \mathbf{x}^{(0)}\right)} \\
&=\int d \mathbf{x}^{(1 \cdots T)} q\left(\mathbf{x}^{(1 \cdots T)} \mid \mathbf{x}^{(0)}\right) \frac{p\left(\mathbf{x}^{(0 \cdots T)}\right)}{q\left(\mathbf{x}^{(1 \cdots T)} \mid \mathbf{x}^{(0)}\right)} \\
&=\int d \mathbf{x}^{(1 \cdots T)} q\left(\mathbf{x}^{(1 \cdots T)} \mid \mathbf{x}^{(0)}\right) \cdot p\left(\mathbf{x}^{(T)}\right) \prod_{t=1}^T \frac{p\left(\mathbf{x}^{(t-1)} \mid \mathbf{x}^{(t)}\right)}{q\left(\mathbf{x}^{(t)} \mid \mathbf{x}^{(t-1)}\right)}
\end{aligned}  
\end{equation}


\section{Training}
\label{sec:training}

Training amounts to maximizing the model log likelihood,
\begin{equation}
  \label{eq:35}
\begin{aligned}
L= & \int d \mathbf{x}^{(0)} q\left(\mathbf{x}^{(0)}\right) \log p\left(\mathbf{x}^{(0)}\right) \\
= & \int d \mathbf{x}^{(0)} q\left(\mathbf{x}^{(0)}\right) \cdot 
 \log \left[\begin{array}{c}
\int d \mathbf{x}^{(1 \cdots T)} q\left(\mathbf{x}^{(1 \cdots T)} \mid \mathbf{x}^{(0)}\right) . \\
p\left(\mathbf{x}^{(T)}\right) \prod_{t=1}^T \frac{p\left(\mathbf{x}^{(t-1)} \mid \mathbf{x}^{(t)}\right)}{q\left(\mathbf{x}^{(t)} \mid \mathbf{x}^{(t-1)}\right)}
\end{array}\right],
\end{aligned}
\end{equation}


which has a lower bound provided by Jensen's inequality,
\begin{equation}
  \label{eq:37}
L \geq  \int d \mathbf{x}^{(0 \cdots T)} q\left(\mathbf{x}^{(0 \cdots T)}\right) \cdot  \log \left[p\left(\mathbf{x}^{(T)}\right) \prod_{t=1}^T \frac{p\left(\mathbf{x}^{(t-1)} \mid \mathbf{x}^{(t)}\right)}{q\left(\mathbf{x}^{(t)} \mid \mathbf{x}^{(t-1)}\right)}\right] .  
\end{equation}




for our diffusion trajectories this reduces to,
\begin{equation}
  \label{eq:36}
L  \geq K 
\end{equation}
\begin{equation}
  \label{eq:39}
  \begin{aligned}
K= & -\sum_{t=2}^T \int d \mathbf{x}^{(0)} d \mathbf{x}^{(t)} q\left(\mathbf{x}^{(0)}, \mathbf{x}^{(t)}\right) \\
& D_{K L}\left(q\left(\mathbf{x}^{(t-1)} \mid \mathbf{x}^{(t)}, \mathbf{x}^{(0)}\right)|| p\left(\mathbf{x}^{(t-1)} \mid \mathbf{x}^{(t)}\right)\right) \\
+ & H_q\left(\mathbf{X}^{(T)} \mid \mathbf{X}^{(0)}\right)-H_q\left(\mathbf{X}^{(1)} \mid \mathbf{X}^{(0)}\right)-H_p\left(\mathbf{X}^{(T)}\right)
  \end{aligned}
\end{equation}

where the entropies and KL divergences can be analytically computed. The derivation of this bound parallels the derivation of the log likelihood bound in variational Bayesian methods.

If the forward and reverse trajectories are identical, corresponding to a quasi-static process, then the inequality in Equation \ref{eq:36} becomes an equality.

Training consists of finding the reverse Markov transitions which maximize this lower bound on the log likelihood,
\begin{equation}
  \label{eq:38}
\hat{p}\left(\mathbf{x}^{(t-1)} \mid \mathbf{x}^{(t)}\right)=\underset{p\left(\mathbf{x}^{(t-1)} \mid \mathbf{x}^{(t)}\right)}{\operatorname{argmax}} K .  
\end{equation}


%%% Local Variables:
%%% mode: latex
%%% TeX-master: "machine-learning"
%%% End:

