
\chapter{信箱记忆法}

很多人一下子记住了大量的学习资料后,在复习的时候非常吃力,甚至有非常严重的遗忘情况出现。
原因就在于处理记忆大量的资料时,没有把经过记忆处理后的材料放进大脑里面一个特殊的箱子里,从而导致信息杂乱无章,为了避 免这种情况出现,运用的时候能迅速找到记忆资料所存放的位置,运用信箱记忆法是非常关键的一步。

运用信箱记忆法的三个关键要点:
\begin{enumerate}
\item 设立信箱\\
  在运用信箱记忆法的时候,设立的信箱一定要是实际的物品及地点等,这点非常关键。
  我们必须选择一些比较熟悉的物品、地点、身体部位等来作为设立的信箱。
  只有是熟悉的,大脑才能快速呈现出信箱的形象、大小等。
  在设立信箱的时候一定要根据记忆材料的数量设立相应的信箱, 比如,有20项资料,就设立20个信箱,有50项资料,就设立50个信箱。
\item 设立的信箱要按照一定的顺序\\
  在采用信箱记忆法的时候,顺序是保证我们能高效记忆和快速回忆复习的关键。
  如果没有顺序,虽然也能把资料全部完成记忆,但复习的时候就会非常凌乱,而且运用的时候常常会导致找不到信箱的情况,非常不方便资料的提取。
  一般情况下按照两种顺序设立。
  一种是从左到右或者是从右到左,一种是从上到下或者是从下到上。
  如果按照其中任意一种顺序设立信箱的时候中间有重复出现的物品的话,只需把第一个设立为信箱就好,其他的都忽略掉,否则在记忆材料的时候因为有相同的信箱,很容易发生记忆混淆的现象。
\item 信箱一定在记忆材料的前面\\
  采用信箱记忆法的时候,信箱一定采用链式记忆锁链或者串联在记忆资料的前面,这样的好处就是能快速找出信箱里面所“装载”的资料。
  只要想到第几个信箱,这个信箱里面所“装载”的资料也就轻松提取出来了。
  如果把信箱锁链或者串联在记忆资料的中间,就会很容易出现找不到信箱无法提取出记忆的资料。
\end{enumerate}

如果你是为了长期记忆或者是要在短时期之内记住大量的学习资料,你所要设立的信箱种类应该是多种多样的,一般单一地设立几十个同类的信箱无法满足你记忆海量的资料。
为了设立更多种信箱来协助自己记忆大量的资料,方便而又常用的方法 就是采取位置信箱、身体信箱、生肖信箱、人物信箱等。
一句话就是,将信息关联到自己熟悉的场景。





%%% Local Variables:
%%% mode: latex
%%% TeX-master: "memory"
%%% End:
