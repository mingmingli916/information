
\chapter{记忆之门}



\section{良好的心态}

\keyword{自信}影响着一个人记忆力的发挥。
当一个人自信心增强的时候,精力往往非常旺盛, 情绪非常乐观,大脑细胞的活动能力大大增强,从而使大脑智能思维不断奔腾流动,想象天马行空,给予大脑全新的能量。
所以,就算是你丢掉了一切,也不要丢掉心灵中最宝贵的财富,那就是自信。
你一定要在\keyword{潜意识}里告诉自己,“我可以!”,潜意识对人的影响力量是显意识的3万倍以上。




\section{克服遗忘}

因为存在遗忘定律,最大化记忆的最好方法就是定期复习。


\section{增强记忆和思维能力的三大黄金思维模式}

\begin{itemize}
\item 图像\\
  直观又有冲击力。
\item 比喻\\
  陌生变熟悉,深奥变浅显,抽象变具体,使事物生动形象具体可感。
\item 联系\\
  不再孤立,有线索可以搜寻到相关事物。
\end{itemize}



%%% Local Variables:
%%% mode: latex
%%% TeX-master: "memory."
%%% End:
