
\chapter{链式记忆法}

链式记忆分为两种,一种叫链式环扣法,另一种叫链式串联法。
都是通过创造性的联想思维,找到材料与材料之间最好的\keyword{链接点},形成一条记忆链条,相互联系,从而更快、更简单、更轻松而且更加牢固地记住大量的资料。
链式记忆法非常适用于记忆\keyword{较多的、相互没有联系}的资料,只需要在记忆的时候将资料与资料相互按顺序链接起来就好。



\section{记忆资料的转换}

在学习中,为了提升记忆的速度和效率,不管遇到的是形象 材料还是抽象材料,在记忆的时候,都要进行转换。
运用正确的转换法则,我们就能把记忆材料给大脑以一个\keyword{直观、 鲜明、稳定和整体的感知},大脑在接收到这些感知的时候就会迅速联想发散,引发人的情绪色彩,产生跳跃式的想象,形成形象思维,深深地印在脑海中。
通过这样的处理后,我们记忆起来就容易多了。


\section{形象材料的转换}

例如:骆驼-面包,雨伞-牛奶,司令-蝴蝶。
两个驼峰里藏了两个面包。
天上下牛奶,我打着伞。
司令屁股上有个蝴蝶胎记。
总之,越夸张,越生动,给大脑的冲击力就越强,我们记忆起来就越容易。
联想的方式越多, 联想的画面越清晰越具体,我们记忆的效率就越好。


\section{抽象材料的转换}

例如:时间、原理、抽象、运气。

抽象材料的转化可以使用如下四种方法:
\begin{itemize}
\item 找代表物体\\
  为“时间”找到一些能代替它的形象物品,如闹钟、手表、沙漏、更夫、日晷、钟楼、手 机、电脑、启明星、影子等。
\item 运用谐音\\
  对“原理”进行谐音转换,可以转换成圆圆的梨 子“圆梨”,同样也可以转化成果园里面的“园里”,还可以转换成院子里 面的“院里”,更夸张点的还可以转换成一个女影视演员的名字“袁立”, 如果你身边正好有一个叫“袁丽”的人,只要你能想到她(他),也是个非常棒的转换了。

  注意两点,1. 不是所有的记忆材料都可以用谐音转换。
  2. 运用谐音记忆了以后,在进行复习和回忆的时候一定要通过谐音还原成原来的材料。
\item 字面展开\\
  只通过字面,不去管材料本身的真实意思,运用一些奇异、特别的想法,把没有生命的词语进行拟人化,或者直接通过字面把有关的人、事、物等串连到一起,让其扩展成生动的形象、奇异的场面或诙谐有趣的情节画面等。
  
  对“抽象”进行字面展开,可以在脑海中得到 这样一幅画面:一个人拿着鞭子在抽打大象。
\item 场景想象\\
  利用右脑对于图片感知和快速反应的功能, 把我们所要记忆的那些枯燥乏味的材料想象成可爱、生动、形象的场景,并在脑海中完美展现出来。
  让场景赋予每个要转换的材料以生命力,在不知不觉中就建立了场景与记忆材料之间的直接联系,在回忆或者运用的时候直接提取大脑里面的场景,就能快速还原回原来的材料。

  由“运气”想到“中了500万大奖后”那兴奋的场景
\end{itemize}



\section{链式环扣记忆法}

如果想熟练地运用这个记忆方法,并应用于工作、生活和学习当中的话,必须掌握链式环扣记忆法的三个关键步骤。
\begin{enumerate}
\item 左右脑的切换\\
  利用右脑的创造能力,把要记忆的资料转换成具体的图像,图像的转换越清晰、越生动越好。
  在进行文字与图像转换的时候,一定要运用形象资料和抽象资料的转换原则,这样就可以抛开左脑逻辑的干扰,让我们更加清晰地感知稳定而具体的图像,如果闭上眼睛,甚至可以让自己在大脑中直接看到构思的图像。
  生动的形象,让资料与资料迅速找到联结点,而且还会累积出无穷多的感性形象,让联结点产生无穷多的创新性,更能给自己的思维带来无穷多的变化。

  例如,毛巾、纸飞机,椅子,猪八戒。
  柔软的白色毛巾很温暖。
  天空中飞着各式各样、各种颜色的纸飞机。
  一把雕刻着九条龙的金黄色椅子。
  一个挺着大肚子、扛着金耙、正在跟妖精恶战的猪八戒。
\item 锁链联结\\
  在锁链联结的步骤里面,唯一的要求就是,锁链的时候资料与资料之间联结的部分越是夸张夸大、越是反逻辑越好。
  图像越是清晰,锁链的时候才能做到一字不漏,顺序不乱,资料一项都不少,锁链联结的效果也就越好。
  不用去管锁链的时候是不是符合逻辑,是不是符合常识,是不是合理,只要能把相邻的两项资料有效地联结起来,方便我们记忆就可以了,记得住,我们的目的就达到了。

  例如,毛巾 — 纸飞机 :用一条红色的毛巾绑着一个五颜六色的纸飞机。
  纸飞机 — 椅子:纸飞机从天上掉下来把椅子砸了个洞。
  椅子 — 猪八戒:一把黄色的椅子上坐着肥胖的猪八戒。
\item 回想记忆\\
  锁链联结完成后,为了记得更牢固、更加清晰准确,必须对所有锁链的资料从头到尾按照第二个步骤里面联结的图像在大脑里回想一遍, 千万不要在回想记忆的时候又把资料再按照另外的图像锁链联结一遍, 这样不仅会浪费时间,还会造成顺序的混乱。
  如果在回想的时候,有些地方回想不起来了,就试着将那些回忆不起来的图像刻画得更加清晰具体和稳定些。
  当所有的联结都能回想得非常完整,转换成原来的资料, 只需要还原就可以了。
\end{enumerate}



\section{链式串联记忆法}

链式串联记忆法其实是链式环扣记忆法的升级版。
链式串联法就是直接把我们要记忆的材料串联成一个整体故事或者影像场景,故此,链式串联法也被有些人称为虚构故事法或影像记忆法。
虚构的这个故事或者影像场景越是离奇、诙谐、幽默,越是容易记忆牢固,就算是抽象资料也不用担心记不住,只需要虚构出的故事或者影像场景能把所有的资料都按照顺序串联起来。



链式串联记忆法的三个关键技巧:
\begin{enumerate}
\item 五感并用\\
  人天生就具备超强的记忆渠道——五大感官,也就是视觉、听觉、嗅觉、味觉和触觉。
  只要调动五大感官共同记忆资料,实现长期记忆都是非常容易的事情。

  在链式串联法中,我们必须充分调动五大感官,把共同接收到的信息组合在一起,形成一个有效的整体。
  在应用链式串联法的时候调动的感官越多,就越容易找到学习的最佳状态,在记忆资料的时候,速度就会越快,记忆的牢固度就越高,就越能体会到学习的乐趣。

  例如,中国古代史的各个阶段。
  \begin{tcolorbox}
    原始社会、夏、商、西周、春秋、战国、秦、汉、三国、两晋、南北朝、隋、唐、五代十国、辽、北宋、金、南宋、元、明、清
  \end{tcolorbox}
  现在调动五官,
  原始社会:想到一群扛着石斧、披着兽皮的原始人,日出而作,日落而息,晚上睡在山洞里面,洞口熊熊燃烧着一堆火,远处还传来一阵又一阵野兽的叫声。
  夏:夏天。想到炎炎夏日,太阳火辣辣地照着大地,树上的知了不知疲倦地鸣叫。
  商:商人。想到一群赶着骆驼行走的商人,骆驼的铃铛发出清脆的铃声。
  西周:谐音成稀粥。想到集市上很多人争抢一碗热气腾腾的稀粥。
  春秋:从春天到秋天。想到树枝上的花开了,到了秋天树枝上结满了果实。
  战国:一个国家的名称。想到战国时期,群雄争霸。
  秦、汉:人名秦汉。想到一个长得英俊潇洒、风度翩翩的台湾演员。
  三国:谐音成三口锅。想到自己家门口摆放着三口黑漆漆的大铁锅。
  两晋:谐音成两斤。想到铁做的锅只有两斤重。
  南北朝:地名。想到一个地点的名字就叫南北朝。
  隋、唐:隋谐音成随,唐谐音成糖。想到一个人随身的衣服口袋里面装满了糖果。
  五代十国:想到五个口袋十口锅随意地丢在地上。
  辽:谐音成聊。想到一大群人坐在一起拉家常。
  北宋:谐音成白送。想到一个人把东西白白地送给人家,一分钱也不要。
  金:想到一大堆金光灿灿的黄金堆在自己的家里。
  南宋:谐音成难送。想到把自己家里最值钱的东西送给人家,送了几次都没有人接受。
  元、明、清:想象一个穿着官袍的人,原(元)来是明朝的清官。
\item 组成故事\\
  我们都喜欢听故事,绝大多数的人只要听完别人讲的故事后都能轻易地记住。
  这是因为故事有情节有意义并且有趣,让人对不能直接接触到的东西产生好奇和向往,这种好奇和向往能对我们的大脑产生刺激, 从而形成一个牢固的信息点,只要有外界的信息引导,这个信息点就能在我们的大脑里呈现出清晰的图像情景,并让人用语言和肢体动作表达出来。


  我们出生在\argument{原始社会} 的\argument{夏} 天,看到一群\argument{商} 人在吃\argument{稀粥}(西周) ,从\argument{春} 天一直吃到\argument{秋} 天,最后受不了了就逃跑到了\argument{战国} ,遇到了一个叫\argument{秦汉} 的人,他背着\argument{三口锅}(三国) , 一共有\argument{两斤}(两晋)重,他背着三口锅到了\argument{南北朝}这个地方,\argument{随}(隋) 身还带了许多\argument{糖}(唐) ,这些糖用\argument{五个口袋十口锅}(五代十国)装着,他很喜欢跟人家\argument{聊}(辽)天,聊得很高兴的时候,就把糖\argument{白送}(北宋) 给别人,但人家不要白送的,要用\argument{金}子来交换,他就感叹这个糖怎么这么\argument{难送}(南宋)出去,最后一打听,\argument{原}(元) 来人家是\argument{明}朝的\argument{清}官。
\item 复述还原\\
  为了达成长期记忆,最好是在组成故事后要多复述几遍,尤其是遇到比较生疏的资料时,复述几遍是相当必要的。
  只有把重点资料记准确了,记牢固了,在以后要使用的时候才能清晰地从大脑里面提取出来。
  由于在组成故事的过程当中,运用了词语的转换原则,因此,必须将转换部分的资料还原成原来的信息。
\end{enumerate}

\section{链式记忆法的注意事项}

我们在采用链式记忆的时候一定要使用“我自己”这个场景,而不是“我”的第一人称。只有使用“我自己”才会让自己感受到的故事更加真实。
在链式记忆中,要使用“我自己”的场景,还有一个原因,就是当你真正地把自己植入故事时,就会对本来虚拟的故事充满情感。
在链式记忆时,如果你能把“假装”、“好像”、“假设”、“如果是”等这些隔离自己情感的词语去掉的话,大脑在记忆的时候才会真正地感受到真实。


\section{记忆}

\begin{tcolorbox}
  \argument{链条(链式记忆法)}带动\argument{发动机(转换,电能转动能)},我背着飞行器\argument{戴着表(代表)},在\argument{长颈(场景)}鹿身上留下\argument{紫(字面展开)鞋印(谐音)}。
\end{tcolorbox}


%%% Local Variables:
%%% mode: latex
%%% TeX-master: "memory"
%%% End:
