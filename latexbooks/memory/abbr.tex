
\chapter{缩编记忆法}

缩编记忆法就是我们把需要记 忆的大量长段资料进行通读过后,进行分析、整理出资料的基本要素以及本质,然后排列组合成简单、短小、精炼、便于记忆的熟悉句子。 当要提取出记忆的材料时,只需要把组合成的句子还原回去就可以了。


缩编记忆法的关键要点:
\begin{enumerate}
\item 通读材料\\
  通读要记忆的资料,可以让我们准确地把握材料的内容,了解材料一些生字、生词、语句等的大概意思,让大脑思维产生丰富合理的想象,让材料的内容在心中、在眼前活起来。
\item 找出代表\\
  知道了资料的大概意思,就需要抓住资料的个性及特征找出能够代表其内容的一些关键的字、词。
  在找代表的时候,一定离不开观察、辨别和发掘这三个步骤。
  观察可以让我们从整体到局部都能找出代表这些资料的关键字和关键词。
  辨别可以让代表这些资料内容中极其相似和容易混淆的关键字和关键词一目了然。
  发掘针对的是那些不容易找出关键字和关键词的资料,必须要依靠人为地给予一些关键字或者关键词,以方便记忆。
  最主要的一点就是,找出的关键词或者关键字要易于组合。
\item 转化组合\\
  组合的时候,一定要让这些关键词或者关键字产生联系,可以产生的联系有因果联系、顺序联系、有先到后的动态联系、也可以是由近及远的静态联系等。
  通过这些联系组合成我们熟悉的定理、法则、公式、人、事、地、物或者是短小、精炼、诙谐、幽默、夸张的笑话、句子、谚语 等,为了方便记忆,你还可以把关键字或者关键词进行谐音转化。
\item 复习还原\\
  还原的时候,一定要注意资料的整体性和正确性。
  复习可使用351351复习法。
\end{enumerate}



例如:
\begin{tcolorbox}
  影响气候的主要因素:洋流、地形、海陆分布、大气环流、纬度
\end{tcolorbox}

\begin{enumerate}
\item 通读一遍就知道了影响气候一共有5个最主要的因素。
\item 从5个因素里面各选择出一个关键字,洋流选择“洋”字,地形选择“地”字,海陆分布选择“海”字,大气环流选择“大”字,纬度选择“纬”字。
\item 把选出的5个关键字如果直接组合起来就是“洋地海大纬”。
  有没有更好的组合方式呢?当然有,只要对其中的“纬”字谐音转化后,重新排列一下这几个字的顺序我们就得到了一个非常棒的句子 —— 伟(纬)大地海洋。
\item 影响气候的主要因素直接记忆一句话“伟大地海洋”就可以了。
\end{enumerate}

还可以用缩略法来创歌诀。
例如:
两湖两广两河山 (湖南、湖北,广东、广西,河南、河北,山 东、山西)
五江云贵福吉安(江苏、浙江、江西、黑龙江、新疆,云南,贵州,福建,吉林,安徽)
西四二宁青甘陕(西藏,四川,宁夏、辽宁,青海,甘肃,陕西)
海南内台北上天(海南岛,内蒙古,台湾,北京,上海,天津)

%%% Local Variables:
%%% mode: latex
%%% TeX-master: "memory"
%%% End:
