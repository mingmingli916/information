\documentclass{book}
\usepackage[a4paper,inner=1.5cm,outer=3cm,top=2cm,bottom=3cm,bindingoffset=1cm]{geometry}
\usepackage{hyperref}
\usepackage{xeCJK}
\usepackage{listings}
\usepackage{xcolor}
\usepackage{graphicx}

\begin{document}
\title{Latex Cookbook}
\author{Mike Chyson}
\maketitle
\tableofcontents{}
\chapter{Ubuntu下Latex中文的使用}

\section{包的安装}
% \lstset{language=Sh}
\begin{lstlisting}[frame=single]
# 安装texlive
apt search texlive
apt install texlive

# 这样就可以使用latex了,
# 但是为了能编译中文,需要安装中文包
apt search cjk
apt install latex-cjk-all

# 安装额外的附加包
apt install texlive-latex-extra

# 为了使用xelatex命令
apt install texlive-xetex
\end{lstlisting}
至此,支持中文的latex环境已经可以正常使用。

\section{中文包的使用}
将如下内容写入文件test.tex,然后运行命令
\begin{lstlisting}
xelatex test.tex
\end{lstlisting}
即可生成pdf文件。

\lstset{language=Tex}
\begin{lstlisting}[frame=single]
\usepackage{xeCJK}  % 使用中文包
% \setCJKmainfont{WenQuanYi Micro Hei}  % 可选项,配置字体
\begin{document}
我是中国人。
\end{document}
\end{lstlisting}

\chapter{Mac LaTex}
\section{Install}
% \lstset{language=Sh}
\begin{lstlisting}
  brew cask install mactex
\end{lstlisting}
\chapter{Latex IDE}


\section{Texmaker集成环境}
Texmaker是latex的一种集成开发环境。你可以使用自己喜欢的集成环境,或者文本编辑器。比如Emacs + AUCTex。


\begin{lstlisting}[frame=single]
# 安装
apt install texmaker

# 使用
texmaker
\end{lstlisting}
Options $\rightarrow$ Configure Texmaker
\begin{figure}[h]
\centering
\includegraphics[width=\textwidth]{/home/hack/Pictures/texmaker.png}
\caption{texmakter编译环境配置}
\end{figure}

\section{Emacs + AUCTex}
\end{document}
%%% Local Variables:
%%% mode: latex
%%% TeX-master: t
%%% End:

%%% Local Variables:
%%% mode: latex
%%% TeX-master: t
%%% End:
